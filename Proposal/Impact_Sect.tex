\section{Impact design}

\subsection{Local Impact Assessment}

The food production optimization solution targets multiple dimensions of sustainable development across different time horizons. The anticipated impacts span nutritional security, environmental sustainability, and economic prosperity for agricultural communities. The impact framework follows a results chain from immediate outputs through short-term, mid-term, and long-term outcomes, with specific indicators and measurement approaches aligned to SDG targets.

\subsubsection{Outputs}

The quantum optimization system delivers four primary outputs: (1) diversified crop plans minimizing environmental impact, measured by the percentage of food groups below normalized LCA environmental impact midpoints; (2) crop plans maximizing nutritional scores, tracking food groups above Nutrient Value Score (NVS) thresholds; (3) crop plans maximizing affordability, ensuring diverse food groups remain accessible to target populations; and (4) a digital mapping and optimization platform, with deployment tracked through stakeholder usage metrics. These outputs directly support SDG 2 (Zero Hunger) Target 2.a on agricultural research investment and SDG 9 (Industry, Innovation and Infrastructure) Target 9.a on sustainable infrastructure in developing countries. Analysis of crop plans provides the primary means of measurement for optimization outputs.

\subsubsection{Short-Term Impacts (0-2 years)}

Short-term impacts focus on the immediate implementation of optimized crop plans and platform adoption. The primary indicator is the actual planting of diversified crops meeting optimization criteria—specifically, the percentage of food groups from quantum-optimized plans that are actually cultivated by participating farmers. Planting plan analysis combined with baseline surveys track whether optimized crop selections that minimize environmental impact, maximize affordability, and maximize nutritional scores translate from digital plans to on-ground implementation.

\textbf{SDG 2 (Zero Hunger):} The solution delivers optimized crop planning outputs that balance nutritional value, affordability, and environmental impact. Through quantum-classical hybrid optimization, the system produces planting assignments across diversified food groups that can be implemented immediately by farming communities. This addresses SDG Target 2.3 (doubling agricultural productivity of small-scale producers) and 2.4 (ensuring sustainable food production systems).

\textbf{SDG 1 (No Poverty):} By optimizing crop selection toward higher-value, market-appropriate foods, the system enables immediate income increases for small-scale farmers. Household income surveys demonstrate measurable improvements in earnings from optimized crop sales compared to traditional planting patterns. The proportion of the farming population living below the national poverty line serves as the key indicator, disaggregated by sex and age through census surveys. This directly supports Target 1.2 (reducing poverty by at least half).

\textbf{SDG 6 (Clean Water and Sanitation):} Optimized crop plans prioritize water-efficient species selection. Changes in water-use efficiency over time are measured through comparison of crop plan analysis with previous crop split patterns, addressing Target 6.4 (substantially increasing water-use efficiency across all sectors).

\textbf{SDG 4 (Quality Education):} Ensuring equitable access to the optimization platform requires building ICT skills among rural farming communities. The proportion of youth and adults with information and communications technology skills is tracked through school vocational surveys and tool feedback, supporting Target 4.4 (increasing relevant technical and vocational skills for employment).

\textbf{SDG 8 (Decent Work and Economic Growth):} Diversified crop production creates additional employment opportunities in sustainable agriculture, from planting through harvest and post-harvest processing. Employment surveys document new labor positions generated by optimized farming operations, supporting Target 8.5 (full employment and decent work with equal pay).

\textbf{SDG 9 (Industry, Innovation and Infrastructure):} The deployment of the digital mapping and quantum optimization platform provides stakeholders with accessible technology tools for agricultural planning, directly supporting SDG Target 9.a (facilitating sustainable infrastructure development in developing countries).

\textbf{SDG 10 (Reduced Inequalities):} Platform design prioritizes accessibility for smallholder farmers, with usage metrics tracking the percentage of small-scale producers accessing quantum planning tools. This addresses Target 10.2 (promoting social inclusion of all).

\textbf{SDG 17 (Partnerships for the Goals):} Implementation establishes multi-stakeholder partnerships between research institutions, government agencies, NGOs, and farmer cooperatives for technology deployment. Partnership agreements and MOUs document collaboration frameworks, supporting Target 17.16 (enhanced global partnerships).



\subsubsection{Mid-Term Impacts (2-5 years)}

Mid-term impacts emerge as sustained implementation generates measurable improvements in nutrition, environmental quality, and ecosystem health. Key indicators shift from implementation metrics to outcome measures at household and community scales.

\textbf{SDG 2 (Zero Hunger):} Sustained implementation of optimized crop plans leads to measurable improvements in micronutrient intake. The bulk supply of key micronutrients (iron, vitamin A, zinc, folate) present in harvested crops quantifies nutritional improvements at the agricultural production level. Biomarker analysis (blood/urine samples) can quantify changes in key nutrients among participating households, addressing SDG Target 2.2 (ending all forms of malnutrition). The Household Dietary Diversity Score (HDDS)—measuring the average number of food groups consumed in the previous three months—tracks improvements in dietary variety through household dietary recall surveys at baseline and follow-up.

\textbf{SDG 6 (Clean Water and Sanitation):} Beyond water-use efficiency, mid-term impacts include reduced agricultural water eutrophication. The proportion of water bodies with good ambient water quality (low levels of pollutants) is assessed through water reserve sample analysis, addressing Target 6.4 (increasing water-use efficiency and sustainable water management).

\textbf{SDG 10 (Reduced Inequalities):} Income inequality among farmers decreases as smallholders adopt optimized practices previously accessible only to larger operations. Gini coefficient calculations track changes in income distribution, supporting Target 10.1 (income growth of bottom 40\%).

\textbf{SDG 12 (Responsible Consumption and Production):} Farms adopting sustainable production methods increase measurably. Practice surveys and certification records document implementation of environmentally-sound techniques, addressing Target 12.2 (sustainable management of natural resources).

\textbf{SDG 13 (Climate Action):} The environmental optimization component, when consistently applied, contributes to measurable reductions in greenhouse gas emissions from agricultural activities. GHG emissions per capita provide the key metric, with changes measured after adopting quantum-guided planting through field emission data, energy use logs, and CO$_2$ model estimates combined with baseline surveys. This supports climate mitigation efforts while maintaining food security and addresses Target 13.a on implementing climate commitments.

\textbf{SDG 15 (Life on Land):} Biodiversity on participating farmland improves through sustainable land use practices. The Red List Index, which shows trends in overall extinction risk for species, provides a standardized measure assessed through ecological surveys. This addresses Target 15.5 (reducing degradation of natural habitats and halting biodiversity loss).



\subsubsection{Long-Term Impacts (5-10 years)}

Long-term impacts manifest as sustained improvements compound over multiple growing seasons, generating transformational changes in community health, economic prosperity, and environmental sustainability.

\textbf{SDG 2 (Zero Hunger):} Long-term adoption leads to improved hunger rates at community and regional scales. The percentage of households meeting minimum dietary needs—defined as achieving required daily nutrient intake—provides the primary indicator, measured through household dietary recall surveys and nutritional assessments combined with national health survey (NHS) data. Nutritional adequacy translates to functional outcomes: adults demonstrate improved work capacity through average productivity self-rating scores (1--5 scale), while children show better school performance through composite scores tracking attendance, concentration, and test results. Both outcomes are measured through standardized surveys at baseline and follow-up, addressing Target 2.2 (ending all forms of malnutrition) and Target 2.3 (doubling agricultural productivity and incomes).

\textbf{SDG 1 (No Poverty):} Poverty rates in farming communities decline measurably as sustained agricultural improvements compound over time. The proportion of the farming population living below the national poverty line is tracked through census surveys supplemented by short-term survey data, documenting the share of farming households moving above poverty thresholds and approaching Target 1.1 (eradicating extreme poverty).

\textbf{SDG 3 (Good Health and Well-being):} Reduced malnutrition and improved dietary diversity lower healthcare costs at the community level through decreased incidence of nutrition-related diseases. Average monthly household healthcare expenditure (in USD) declines measurably, tracked through household expenditure surveys at baseline and follow-up combined with receipts and expenditure logs. This addresses Target 3.9 (reducing deaths and illnesses from pollution and contamination) and creates a positive feedback loop with agricultural productivity.

\textbf{SDG 4 (Quality Education):} Improved child nutrition enhances cognitive function and educational outcomes. School performance composite scores (combining attendance, concentration, and test results) show measurable improvements linked to diet, supporting Target 4.4 (increasing relevant technical and vocational skills).

\textbf{SDG 8 (Decent Work and Economic Growth):} Agricultural sector contribution to regional GDP increases measurably as productivity gains scale across farming communities. The annual growth rate of real GDP per employed person due to land optimization is tracked through employment statistics and tax returns combined with baseline survey data. Economic impact assessments document the sector's enhanced role in sustainable growth, supporting Target 8.1 (sustaining per capita economic growth) and Target 8.2 (achieving higher economic productivity through diversification and innovation).

\textbf{SDG 13 (Climate Action):} Cumulative emissions reductions from optimized agricultural systems contribute to global warming mitigation efforts. GHG reduction per farming season—the decrease in total greenhouse gas emissions from farms using optimized crop plans—is documented through GHG monitoring tools, soil tests, and energy-use records compared against baseline surveys. This generates measurable impacts on regional climate resilience, supporting Target 13.2 (integrating climate measures into policies and planning).

\textbf{SDG 15 (Life on Land):} Soil quality improves on farms using optimized agriculture, preventing land degradation. The proportion of land that is degraded over total land area, assessed through land quality sampling and baseline analysis, measures progress toward Target 15.3 (combating desertification and restoring degraded land). Soil testing at baseline and regular intervals measures composite indicators (organic matter, nutrients, pH), demonstrating that soil health improvements boost long-term productivity.



\subsubsection{Interlinkages}

\begin{itemize}
\item \textbf{SDG1}\par
\textbf{Positive Interlinkages:} Poverty reduction enables better access to nutritious food and improved health (SDG 2, SDG 3), supports educational attainment (SDG 4), and strengthens economic participation and productivity (SDG 8).

\textbf{Negative Interlinkages \& Mitigation:} Climate policies embedded in the optimization may increase short-term costs for poor households (SDG 13). Mitigation includes phased implementation with transition support, subsidies for sustainable inputs, social protection systems, climate-resilient livelihoods diversification, and ensuring affordability remains a weighted objective in the optimization function across all time horizons.

\item \textbf{SDG2}\par
\textbf{Positive Interlinkages:} Better nutrition from diversified crops improves health outcomes and reduces disease burden (SDG 3). Increased agricultural productivity through optimization reduces poverty in rural areas (SDG 1) and creates decent work opportunities (SDG 8). The optimized planning also reduces food waste and resource use (SDG 12).

\textbf{Negative Interlinkages \& Mitigation:} Without proper environmental constraints in the optimization model, intensive agriculture may increase GHG emissions (SDG 13), strain water resources (SDG 6), and harm terrestrial ecosystems (SDG 15). To mitigate these trade-offs, the optimization explicitly includes environmental impact scores from life-cycle assessments and constrains solutions to minimize land use, water consumption, and carbon footprint while maintaining nutritional adequacy.

\item \textbf{SDG3}\par
\textbf{Positive Interlinkages:} Better health outcomes enable improved agricultural productivity (SDG 2), better educational outcomes (SDG 4), and support economic growth through a healthier workforce (SDG 8).

\textbf{Negative Interlinkages \& Mitigation:} Healthcare infrastructure expansion may increase energy use and emissions (SDG 13) and generate medical waste (SDG 12). Mitigation includes integrating preventive health approaches that reduce infrastructure needs, using renewable energy for health facilities, and implementing waste management systems.

\item \textbf{SDG6}\par
\textbf{Positive Interlinkages:} Improved water access and quality directly enhance health outcomes (SDG 3), and sustainable water management protects ecosystems (SDG 15).

\textbf{Negative Interlinkages \& Mitigation:} Agricultural water use may compete with drinking water needs (SDG 2) in water-scarce regions. Mitigation includes: integrated water resource management, rainwater harvesting systems, drip irrigation technologies, and drought-resistant crop varieties in the optimization portfolio.

\item \textbf{SDG8}\par
\textbf{Positive Interlinkages:} Economic growth reduces poverty (SDG 1) and innovation infrastructure drives job creation (SDG 9).

\textbf{Negative Interlinkages \& Mitigation:} Economic growth without environmental safeguards may increase emissions (SDG 13). The optimization explicitly constrains environmental impacts while maximizing economic value, balancing growth with sustainability. Long-term mitigation includes mainstreaming environmental safeguards in agricultural policy, carbon pricing or offset mechanisms, and demonstrating that sustainable intensification supports both growth and climate goals.

\item \textbf{SDG9}\par
\textbf{Positive Interlinkages:} Innovation infrastructure drives economic growth and job creation (SDG 8), enables smart planning for sustainable cities (SDG 11), and strengthens partnerships through data sharing (SDG 17).

\textbf{Negative Interlinkages \& Mitigation:} Technology production and operation increase energy demand (SDG 13), generate electronic waste (SDG 12), and may worsen inequalities if access is uneven (SDG 10). Mitigation strategies include: using renewable energy for computing infrastructure, designing for minimal computational overhead through efficient hybrid algorithms, providing open-source tools to ensure equitable access, and incorporating training programs to bridge the digital divide.

\item \textbf{SDG10}\par
\textbf{Positive Interlinkages:} Reduced inequality lifts vulnerable populations (SDG 1) and fair wages support economic growth (SDG 8).

\textbf{Negative Interlinkages \& Mitigation:} Poverty reduction efforts that rely heavily on technology adoption may inadvertently widen inequalities if access is uneven (SDG 9, SDG 10). Mitigation includes zero-cost access models, offline-capable tools, agricultural extension partnerships for technology dissemination, community facilitators bridging the digital divide, differentiated support for smaller farms, cooperative models pooling resources, and progressive pricing structures favoring disadvantaged farmers.

\item \textbf{SDG12}\par
\textbf{Positive Interlinkages:} Responsible consumption reduces emissions (SDG 13) and reduced waste protects ecosystems (SDG 15).

\textbf{Negative Interlinkages \& Mitigation:} Consumption limits may slow short-term economic growth (SDG 8). Mitigation through: market development for sustainable products commanding premium prices, certification programs adding value, and demonstrating long-term economic benefits of sustainable practices.

\item \textbf{SDG13}\par
\textbf{Positive Interlinkages:} Climate action drives renewable energy adoption (SDG 7) and protects terrestrial (SDG 15) and marine (SDG 14) ecosystems through reduced agricultural runoff and land conversion.

\textbf{Negative Interlinkages \& Mitigation:} Climate policies and optimization constraints may increase costs for poor households (SDG 1), temporarily limit agricultural expansion options (SDG 2), and slow economic growth during transition (SDG 8). Mitigation requires careful policy design including gradual implementation timelines, subsidies for sustainable inputs, technical assistance for farmers, just transition frameworks ensuring equitable distribution of costs and benefits, investment in climate adaptation alongside mitigation, and continuous optimization refinement balancing affordability with environmental objectives.

\item \textbf{SDG15}\par
\textbf{Positive Interlinkages:} Ecosystem protection supports climate action (SDG 13) and forests regulate water cycles (SDG 6).

\textbf{Negative Interlinkages \& Mitigation:} Land conservation may limit agricultural expansion (SDG 2) in land-scarce contexts. Mitigation includes intensification on existing farmland via optimization rather than extensification, agroforestry systems integrating conservation with production, payment for ecosystem services programs compensating farmers for conservation, landscape-level planning balancing production and conservation zones, restoration of degraded lands for agricultural use, and demonstrating that soil health improvements boost long-term productivity, aligning conservation with food security.

\item \textbf{SDG17}\par
\textbf{Positive Interlinkages:} Effective partnerships facilitate technology transfer and innovation adoption (SDG 9) and strengthen food systems through multi-stakeholder collaboration (SDG 2).

\textbf{Negative Interlinkages \& Mitigation:} Unequal partnerships may reinforce power imbalances (SDG 10). Mitigation includes: participatory governance structures, farmer representatives in decision-making bodies, transparent benefit-sharing agreements, and capacity building to equalize negotiating power.
\end{itemize}

\subsection{Global Transposability}

The quantum-enabled crop optimization approach is highly transposable to diverse geographical contexts, particularly in regions facing the triple burden of malnutrition, environmental degradation, and climate vulnerability.

\textbf{Target Geographies:}
\begin{itemize}
\item \textbf{Sub-Saharan Africa:} With 45\% of the population in rural areas and less than 10\% of health professionals serving these communities \cite{Fanzo2022}, optimized agricultural planning can address both nutritional deficiencies and economic development. Small-scale farms ($<$1 hectare) constitute approximately 45\% of farms but control only 10\% of agricultural land \cite{LOWDER201616}, making optimization critical for efficient resource utilization.

\item \textbf{South and Southeast Asia:} High population density, diverse crop systems, and significant climate risks make these regions ideal candidates. The methodology has been validated using Indonesian food data, demonstrating cultural adaptability through locally acceptable food baskets.

\item \textbf{Latin America:} Regions with high biodiversity but also significant deforestation pressures (80\% driven by food systems \cite{Fanzo2022}) would benefit from optimization that explicitly balances productivity with ecosystem conservation.

\item \textbf{Small Island Developing States (SIDS):} Limited arable land and climate vulnerability necessitate highly efficient crop planning. The optimization approach can maximize nutritional output per hectare while minimizing import dependence.
\end{itemize}

\textbf{Data Requirements \& Feasibility:} The proof of concept requires three data categories: (1) nutritional composition and environmental impact scores (available from GAIN databases and life-cycle assessments), (2) local food preferences and prices (collectible through household surveys and market data), and (3) farm characteristics and constraints (accessible through agricultural census data or participatory mapping). These data types are increasingly available through global initiatives like the Food Systems Dashboard, making deployment feasible in most contexts with appropriate partnerships.



\subsection{SDG Interlinkages and Responsible Innovation}

The quantum optimization approach creates a complex network of SDG interlinkages that must be carefully managed to maximize synergies and minimize trade-offs \cite{Fanzo2022,UNFSS_Planet_2023,Sylvester2024}.


\begin{itemize}
\item \textbf{Nutrition-Health-Productivity:} The primary pathway from SDG 2 $\rightarrow$ SDG 3 $\rightarrow$ SDG 8 creates reinforcing benefits. Optimizing for nutritional density directly improves health outcomes, which in turn enhances workforce productivity and agricultural capacity.

\item \textbf{Technology-Innovation-Partnership:} The quantum computing application (SDG 9) strengthens international partnerships (SDG 17) by demonstrating cutting-edge technology transfer to developing contexts while building local technical capacity. This chain enables knowledge sharing, reduces inequalities through accessible technology (SDG 10), and drives economic growth (SDG 8).

\item \textbf{Climate-Ecosystem Protection:} Environmental optimization constraints simultaneously address SDG 13 (climate action), SDG 15 (terrestrial ecosystems), SDG 6 (water management), and SDG 14 (marine ecosystems) through integrated land-use planning.



\item \textbf{Poverty-Food-Economics:} The interconnection between poverty reduction (SDG 1), food security (SDG 2), and economic growth (SDG 8) forms a mutually reinforcing triangle. Agricultural optimization addressing all three simultaneously achieves greater total impact than addressing any single goal in isolation.
\end{itemize}

\textbf{Critical Trade-offs Requiring Active Management:}
\begin{itemize}
\item \textbf{Productivity-Environment Tension:} The fundamental trade-off between maximizing food production (SDG 2) and minimizing environmental impact (SDGs 6, 13, 14, 15) requires explicit multi-objective optimization. Sufficient analysis reveals optimal balance points where nutritional adequacy, affordability, and environmental sustainability coexist.

\item \textbf{Technology Access and Digital Divide:} While quantum optimization provides powerful capabilities (SDG 9), it risks exacerbating inequalities (SDG 10) if access is limited to well-resourced actors. Mitigation strategies include: open-source software releases, capacity-building programs, user-friendly interfaces requiring minimal technical expertise, partnerships with agricultural extension services ensuring smallholder farmer access, and community-based facilitator models bridging the digital divide.

\item \textbf{Climate Policy Costs:} Environmental optimization may increase short-term costs for implementing sustainable practices, potentially harming poor households (SDG 1) and slowing immediate economic growth (SDG 8). This trade-off requires careful balancing between climate action (SDG 13) and poverty reduction. Complementary policy interventions include: transition subsidies protecting vulnerable households, technical assistance reducing implementation costs, market development for sustainable products commanding premium prices, payments for ecosystem services compensating farmers for environmental benefits, and staged implementation allowing gradual adaptation.


% \item \textbf{Conservation-Production Balance:} Protecting biodiversity and preventing land degradation (SDG 15) may constrain agricultural expansion (SDG 2), creating tension between environmental protection and food security. Resolution requires: landscape-level planning designating conservation and production zones, sustainable intensification on existing farmland rather than extensification into natural areas, agroforestry and agroecological approaches integrating conservation with production, and demonstrating that ecosystem health supports long-term agricultural productivity.



\textbf{Monitoring and Adaptive Management:} Responsible innovation requires ongoing assessment of both intended and unintended consequences. The impact framework includes: baseline and endline surveys for all indicator measurements, participatory monitoring engaging farming communities, regular SDG interlinkage assessments using network analysis methods \cite{Fanzo2022}, and adaptive optimization that adjusts objective weights based on observed outcomes and stakeholder feedback. This ensures the solution remains aligned with holistic sustainable development goals rather than optimizing narrow metrics at the expense of broader welfare.


\end{itemize}

\subsection{Risks, Assumptions, and Key Activities}

Achieving the anticipated impacts depends on managing key risks, validating critical assumptions, and implementing essential preparatory activities.

\subsubsection{Key Risks}

Several risks could affect impact achievement across different time horizons. Income inequality between farms may persist or widen if optimization benefits accrue primarily to well-resourced farmers rather than smallholders, affecting both short-term and mid-term outcomes. Lack of market access for new crops poses a significant barrier—even optimally selected crops cannot improve incomes if buyers and processors are unavailable, particularly impacting short-term and mid-term economic gains. Regulatory barriers to new seed varieties or agricultural inputs may slow implementation at the short-term stage. 

Climate-related risks span multiple timeframes. Crops may not be resilient enough for accelerating climate change impacts, threatening both short-term and mid-term productivity gains. More acutely, severe climate shocks (droughts, floods, extreme temperatures) can dramatically reduce yields across all timeframes—short-term, mid-term, and long-term—potentially undermining the entire impact logic chain. Finally, poor quality or missing baseline data compromises the ability to measure impacts accurately in the short term, weakening evidence generation and adaptive management.

\subsubsection{Critical Assumptions}

The impact logic chain rests on several critical assumptions that must be validated during implementation. First, local input supply chains must be able to scale to support new crop varieties—seed suppliers, fertilizer distributors, and agricultural extension services need sufficient capacity to meet changing demands. Second, buyers will purchase new nutrient-dense crops, creating market incentives for farmers to adopt optimized crop plans. Third, national regulations must permit trial of new varieties, enabling innovation while maintaining biosafety and quality standards. Fourth, farmers will accept and implement optimized crop plans if they prove profitable, assuming economic rationality drives adoption decisions. Fifth, sufficient soil and climate data are available for baseline surveys, enabling accurate optimization and impact measurement.

\subsubsection{Preparatory Activities}

Essential activities must occur before full-scale implementation. Conducting baseline soil, climate, and household surveys establishes the data foundation for optimization algorithms and impact evaluation, quantifying initial conditions across agronomic, environmental, and socioeconomic dimensions. Linking producer groups with buyers and processors creates the market channels necessary for farmers to sell optimized crop outputs, addressing the market access risk proactively. Engaging regulators and documenting compliance needs ensures regulatory pathways are clear before implementation, reducing delays and enabling adaptive policy development that supports innovation while protecting public interests.

\subsection{Stakeholder Engagement}

Successful implementation requires coordinated engagement with diverse stakeholders, each playing distinct roles in the impact pathway. The Ministry of Agriculture serves as both regulator and policy architect, with primary interests in ensuring food security and regulatory compliance. As a direct stakeholder, engagement occurs through policy dialogue mechanisms, establishing regulatory frameworks that enable innovation while protecting public interests.

Farmers constitute the primary producers and direct beneficiaries of the optimization system. Their central interest—increasing yields and income while managing risk—drives adoption decisions. Direct engagement through field training programs builds capacity for using optimization tools and implementing recommended crop plans. Complementing farmer engagement, buyers and offtakers (including agricultural cooperatives, food processors, and wholesalers) require reliable supplies of nutritious crops. While indirect stakeholders, buyers are critical to closing the value chain. Market linkage activities connect optimized production with demand, ensuring farmers can sell diversified outputs profitably.

Retailers sell nutrient-dense crops to end consumers, with interests in maintaining reliable supply of diverse, high-quality produce. As direct stakeholders, retailers are engaged through market linkage strategies that align optimized crop production with retail demand patterns and quality standards. The local community represents the ultimate beneficiaries—end users of improved food systems seeking better nutrition, stable food access, and lower costs. Direct community engagement through outreach programs ensures the optimization system responds to local food preferences, affordability constraints, and cultural contexts, making nutrition improvements socially sustainable.



