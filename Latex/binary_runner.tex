\documentclass{article}
\usepackage{amsmath}
\usepackage{amssymb}
\usepackage{geometry}
\usepackage{enumitem}
\usepackage{hyperref}

\geometry{margin=1in}

\title{Mathematical Documentation: \texttt{solver\_runner\_BINARY.py}}
\author{Food Optimization Problem - Binary and Continuous Formulations}
\date{\today}

\begin{document}

\maketitle

\tableofcontents
\newpage

\section{Overview}

This script implements a food optimization problem that maximizes agricultural value across multiple farms or land plots while respecting various constraints. It supports two distinct formulations:

\begin{enumerate}
    \item \textbf{Binary Formulation} (Even Grid): Used when land is divided into equal-sized plots
    \item \textbf{Continuous Formulation} (Uneven Distribution): Used when farms have varying sizes
\end{enumerate}

The script solves the optimization problem using three different methods:
\begin{itemize}
    \item PuLP with Gurobi solver (classical optimization)
    \item D-Wave Hybrid CQM Sampler (quantum-classical hybrid)
    \item D-Wave Hybrid BQM Sampler (quantum-enabled via CQM→BQM conversion)
\end{itemize}

\section{Decision Variables}

\subsection{Continuous Formulation (Uneven Distribution)}

\begin{itemize}
    \item $A_{f,c} \in \mathbb{R}^+$: Continuous area variable representing the land area (in hectares) allocated to crop $c$ on farm $f$
    \item $Y_{f,c} \in \{0,1\}$: Binary selection variable indicating whether crop $c$ is planted on farm $f$
\end{itemize}

\textbf{Domain Constraints:}
$$0 \leq A_{f,c} \leq L_f \quad \forall f \in F, c \in C$$
where $L_f$ is the total available land on farm $f$.

\subsection{Binary Formulation (Even Grid)}

\begin{itemize}
    \item $Y_{p,c} \in \{0,1\}$: Binary assignment variable where $Y_{p,c} = 1$ indicates that plot $p$ is assigned to crop $c$, and $Y_{p,c} = 0$ otherwise
\end{itemize}

\textbf{Note:} In the binary formulation, there are no continuous area variables. Each plot has a fixed area $a_p$, and the assignment is discrete.

\section{Parameters and Sets}

\subsection{Sets}
\begin{itemize}
    \item $F$: Set of farms (continuous formulation) or plots (binary formulation)
    \item $C$: Set of crops/foods
    \item $G$: Set of food groups
    \item $C_g \subseteq C$: Set of crops belonging to food group $g$
\end{itemize}

\subsection{Parameters}

\begin{itemize}
    \item $L_f$: Total available land area on farm/plot $f$ (hectares)
    \item $a_p$: Fixed area of plot $p$ in binary formulation (hectares)
    \item $A_{min,c}$: Minimum planting area for crop $c$ if selected (hectares)
    \item $w_{nv}$: Weight for nutritional value objective component
    \item $w_{nd}$: Weight for nutrient density objective component
    \item $w_{ei}$: Weight for environmental impact objective component
    \item $w_{af}$: Weight for affordability objective component
    \item $w_{su}$: Weight for sustainability objective component
    \item $v_{nv,c}$: Nutritional value score for crop $c$
    \item $v_{nd,c}$: Nutrient density score for crop $c$
    \item $v_{ei,c}$: Environmental impact score for crop $c$
    \item $v_{af,c}$: Affordability score for crop $c$
    \item $v_{su,c}$: Sustainability score for crop $c$
    \item $N_{min,g}$: Minimum number of different crops from food group $g$ per farm
    \item $N_{max,g}$: Maximum number of different crops from food group $g$ per farm
\end{itemize}

\section{Objective Functions}

\subsection{Continuous Formulation Objective}

The objective function maximizes the weighted sum of agricultural value metrics, normalized by total available land:

$$\text{maximize} \quad Z = \frac{1}{\sum_{f \in F} L_f} \sum_{f \in F} \sum_{c \in C} v_c \cdot A_{f,c}$$

where the composite value $v_c$ for crop $c$ is defined as:

$$v_c = w_{nv} \cdot v_{nv,c} + w_{nd} \cdot v_{nd,c} - w_{ei} \cdot v_{ei,c} + w_{af} \cdot v_{af,c} + w_{su} \cdot v_{su,c}$$

\textbf{Note:} The environmental impact term is subtracted because lower values are better.

\subsection{Binary Formulation Objective}

For the binary formulation, the objective accounts for the fixed area of each plot:

$$\text{maximize} \quad Z = \frac{1}{\sum_{p \in F} a_p} \sum_{p \in F} \sum_{c \in C} a_p \cdot v_c \cdot Y_{p,c}$$

where $v_c$ is defined identically as in the continuous formulation.

\textbf{Interpretation:} Each selected assignment contributes the plot's area multiplied by the crop's value density.

\section{Constraints}

\subsection{Continuous Formulation Constraints}

\subsubsection{Land Availability Constraints}

Each farm cannot allocate more land than available:

$$\sum_{c \in C} A_{f,c} \leq L_f \quad \forall f \in F$$

\textbf{Label:} \texttt{Land\_Availability\_\{farm\}}

\subsubsection{Minimum Planting Area Constraints}

If a crop is selected on a farm, it must occupy at least the minimum required area:

$$A_{f,c} \geq A_{min,c} \cdot Y_{f,c} \quad \forall f \in F, c \in C$$

\textbf{Label:} \texttt{Min\_Area\_If\_Selected\_\{farm\}\_\{crop\}}

\textbf{Logical Interpretation:}
\begin{itemize}
    \item If $Y_{f,c} = 1$: $A_{f,c} \geq A_{min,c}$ (enforces minimum area)
    \item If $Y_{f,c} = 0$: $A_{f,c} \geq 0$ (no planting, so area can be zero)
\end{itemize}

\subsubsection{Maximum Planting Area Constraints}

If a crop is not selected, its allocated area must be zero:

$$A_{f,c} \leq L_f \cdot Y_{f,c} \quad \forall f \in F, c \in C$$

\textbf{Label:} \texttt{Max\_Area\_If\_Selected\_\{farm\}\_\{crop\}}

\textbf{Logical Interpretation:}
\begin{itemize}
    \item If $Y_{f,c} = 1$: $A_{f,c} \leq L_f$ (area can be up to farm size)
    \item If $Y_{f,c} = 0$: $A_{f,c} \leq 0$ (forces area to zero)
\end{itemize}

\subsubsection{Food Group Minimum Constraints}

Each farm must cultivate at least a minimum number of different crops from specified food groups:

$$\sum_{c \in C_g} Y_{f,c} \geq N_{min,g} \quad \forall f \in F, g \in G \text{ where } N_{min,g} \text{ is defined}$$

\textbf{Label:} \texttt{Food\_Group\_Min\_\{group\}\_\{farm\}}

\subsubsection{Food Group Maximum Constraints}

Each farm must not exceed a maximum number of different crops from specified food groups:

$$\sum_{c \in C_g} Y_{f,c} \leq N_{max,g} \quad \forall f \in F, g \in G \text{ where } N_{max,g} \text{ is defined}$$

\textbf{Label:} \texttt{Food\_Group\_Max\_\{group\}\_\{farm\}}

\subsection{Binary Formulation Constraints}

\subsubsection{Plot Assignment Constraints}

Each plot can be assigned to at most one crop (or remain idle):

$$\sum_{c \in C} Y_{p,c} \leq 1 \quad \forall p \in F$$

\textbf{Label:} \texttt{Max\_Assignment\_\{plot\}}

\textbf{Interpretation:}
\begin{itemize}
    \item $\sum_{c \in C} Y_{p,c} = 0$: Plot remains idle
    \item $\sum_{c \in C} Y_{p,c} = 1$: Plot is assigned to exactly one crop
\end{itemize}

\subsubsection{Minimum Plots Per Crop Constraints}

For crops with minimum planting area requirements, the constraint is converted to a minimum number of plots:

$$\sum_{p \in F} Y_{p,c} \geq \left\lceil \frac{A_{min,c}}{a_p} \right\rceil \quad \forall c \in C \text{ where } A_{min,c} > 0$$

where $a_p$ is the area of each plot (assumed equal in even grid).

\textbf{Label:} \texttt{Min\_Plots\_\{crop\}}

\textbf{Interpretation:} If a crop $c$ requires minimum area $A_{min,c}$, it must be planted on at least $\lceil A_{min,c} / a_p \rceil$ plots.

\subsubsection{Maximum Plots Per Crop Constraints}

For crops with maximum planting area limits, the constraint is converted to a maximum number of plots:

$$\sum_{p \in F} Y_{p,c} \leq \left\lfloor \frac{A_{max,c}}{a_p} \right\rfloor \quad \forall c \in C \text{ where } A_{max,c} \text{ is defined}$$

\textbf{Label:} \texttt{Max\_Plots\_\{crop\}}

\textbf{Interpretation:} If a crop $c$ has maximum area $A_{max,c}$, it can be planted on at most $\lfloor A_{max,c} / a_p \rfloor$ plots.

\subsubsection{Food Group Constraints}

The same food group minimum and maximum constraints apply as in the continuous formulation:

$$\sum_{c \in C_g} Y_{p,c} \geq N_{min,g} \quad \forall p \in F, g \in G$$

$$\sum_{c \in C_g} Y_{p,c} \leq N_{max,g} \quad \forall p \in F, g \in G$$

\textbf{Labels:} \texttt{Food\_Group\_Min\_\{group\}\_\{plot\}}, \texttt{Food\_Group\_Max\_\{group\}\_\{plot\}}

\section{Function Documentation}

\subsection{\texttt{create\_cqm\_farm(farms, foods, food\_groups, config)}}

\textbf{Purpose:} Creates a Constrained Quadratic Model (CQM) for the continuous formulation with area and selection variables.

\textbf{Model Type:} Mixed-Integer Nonlinear Programming (MINLP)
\begin{itemize}
    \item Continuous variables: $A_{f,c} \in [0, L_f]$
    \item Binary variables: $Y_{f,c} \in \{0,1\}$
    \item Objective: Linear in both variable types
    \item Constraints: Linear (including bilinear coupling terms $A_{f,c} = a \cdot Y_{f,c}$)
\end{itemize}

\textbf{Mathematical Model:}
\begin{align*}
\text{maximize} \quad & Z = -\left(\frac{1}{\sum_{f} L_f} \sum_{f \in F} \sum_{c \in C} v_c \cdot A_{f,c}\right) \\
\text{subject to} \quad & \sum_{c \in C} A_{f,c} \leq L_f \quad \forall f \in F \\
& A_{f,c} \geq A_{min,c} \cdot Y_{f,c} \quad \forall f \in F, c \in C \\
& A_{f,c} \leq L_f \cdot Y_{f,c} \quad \forall f \in F, c \in C \\
& \sum_{c \in C_g} Y_{f,c} \geq N_{min,g} \quad \forall f \in F, g \in G \\
& \sum_{c \in C_g} Y_{f,c} \leq N_{max,g} \quad \forall f \in F, g \in G \\
& A_{f,c} \in [0, L_f], \quad Y_{f,c} \in \{0,1\}
\end{align*}

\textbf{Returns:} Tuple $(cqm, A, Y, \text{constraint\_metadata})$

\subsection{\texttt{create\_cqm\_plots(farms, foods, food\_groups, config)}}

\textbf{Purpose:} Creates a CQM for the binary formulation where each plot has a fixed area and binary assignment.

\textbf{Model Type:} Binary Integer Programming (BIP)
\begin{itemize}
    \item Binary variables only: $Y_{p,c} \in \{0,1\}$
    \item Objective: Linear in binary variables with area-weighted coefficients
    \item Constraints: Linear
\end{itemize}

\textbf{Mathematical Model:}
\begin{align*}
\text{maximize} \quad & Z = -\left(\frac{1}{\sum_{p} a_p} \sum_{p \in F} \sum_{c \in C} a_p \cdot v_c \cdot Y_{p,c}\right) \\
\text{subject to} \quad & \sum_{c \in C} Y_{p,c} \leq 1 \quad \forall p \in F \\
& \sum_{p \in F} Y_{p,c} \geq \left\lceil \frac{A_{min,c}}{a_p} \right\rceil \quad \forall c \text{ with } A_{min,c} > 0 \\
& \sum_{p \in F} Y_{p,c} \leq \left\lfloor \frac{A_{max,c}}{a_p} \right\rfloor \quad \forall c \text{ with } A_{max,c} \text{ defined} \\
& \sum_{c \in C_g} Y_{p,c} \geq N_{min,g} \quad \forall p \in F, g \in G \\
& \sum_{c \in C_g} Y_{p,c} \leq N_{max,g} \quad \forall p \in F, g \in G \\
& Y_{p,c} \in \{0,1\}
\end{align*}

\textbf{Note:} The objective is negated because D-Wave minimizes by default.

\textbf{Constraint Conversion:} Minimum and maximum area requirements are converted to discrete plot counts using ceiling and floor functions respectively.

\textbf{Returns:} Tuple $(cqm, Y, \text{constraint\_metadata})$

\subsection{\texttt{solve\_with\_pulp\_farm(farms, foods, food\_groups, config)}}

\textbf{Purpose:} Solves the continuous formulation using PuLP with Gurobi solver.

\textbf{Solver Configuration:}
\begin{itemize}
    \item Method=2: Barrier method (GPU-accelerated if available)
    \item Crossover=0: Disable crossover for GPU computation
    \item BarHomogeneous=1: Homogeneous barrier algorithm (GPU-friendly)
    \item Threads=0: Use all available CPU threads
    \item MIPFocus=1: Focus on finding good solutions quickly
    \item Presolve=2: Aggressive presolve
    \item TimeLimit=100: 100-second time limit
\end{itemize}

\textbf{Mathematical Formulation:}

Variables:
\begin{itemize}
    \item $A_{f,c} \geq 0$: Continuous area variables
    \item $Y_{f,c} \in \{0,1\}$: Binary selection variables
\end{itemize}

Objective:
$$\text{maximize} \quad \frac{1}{\sum_f L_f} \sum_{f \in F} \sum_{c \in C} v_c \cdot A_{f,c}$$

Constraints: Same as \texttt{create\_cqm\_farm}

\textbf{Returns:} Tuple $(\text{model}, \text{results})$ where results contain status, objective value, solve time, areas, and selections.

\subsection{\texttt{solve\_with\_pulp\_plots(farms, foods, food\_groups, config)}}

\textbf{Purpose:} Solves the binary formulation using PuLP with Gurobi solver.

\textbf{Solver Configuration:} Same as \texttt{solve\_with\_pulp\_farm} but with TimeLimit=300 seconds.

\textbf{Mathematical Formulation:}

Variables:
\begin{itemize}
    \item $Y_{p,c} \in \{0,1\}$: Binary assignment variables
\end{itemize}

Objective:
$$\text{maximize} \quad \frac{1}{\sum_p a_p} \sum_{p \in F} \sum_{c \in C} a_p \cdot v_c \cdot Y_{p,c}$$

Constraints:
\begin{align*}
& \sum_{c \in C} Y_{p,c} \leq 1 \quad \forall p \in F \\
& \sum_{c \in C_g} Y_{p,c} \geq N_{min,g} \quad \forall p \in F, g \in G \\
& \sum_{c \in C_g} Y_{p,c} \leq N_{max,g} \quad \forall p \in F, g \in G
\end{align*}

\textbf{Returns:} Tuple $(\text{model}, \text{results})$ where results contain status, objective value, solve time, and plantations.

\subsection{\texttt{solve\_with\_dwave\_cqm(cqm, token)}}

\textbf{Purpose:} Solves a CQM using D-Wave's Leap Hybrid CQM Sampler.

\textbf{Solver:} LeapHybridCQMSampler
\begin{itemize}
    \item Quantum-classical hybrid solver
    \item Handles both continuous and binary variables natively
    \item Processes constraints explicitly
    \item Limited QPU usage
\end{itemize}

\textbf{Process:}
\begin{enumerate}
    \item Submits CQM directly to D-Wave Leap cloud service
    \item Solver decomposes problem into quantum-amenable subproblems
    \item Returns sampleset with multiple solutions
\end{enumerate}

\textbf{Returns:} Tuple $(\text{sampleset}, \text{solve\_time})$

\subsection{\texttt{solve\_with\_dwave\_bqm(cqm, token)}}

\textbf{Purpose:} Converts CQM to BQM and solves using D-Wave's Hybrid BQM Sampler for enhanced QPU utilization.

\textbf{Solver:} LeapHybridBQMSampler
\begin{itemize}
    \item Quantum-classical hybrid solver optimized for binary problems
    \item Higher QPU utilization than CQM solver
    \item Better scaling for larger problems
    \item All variables must be binary (continuous variables are discretized)
\end{itemize}

\textbf{Conversion Process:}

The \texttt{cqm\_to\_bqm()} function discretizes continuous variables:
\begin{enumerate}
    \item Each continuous variable $x \in [L, U]$ is discretized into $n$ binary variables: $b_0, b_1, \ldots, b_{n-1}$
    \item The continuous variable is represented as: $x \approx L + (U - L) \sum_{i=0}^{n-1} 2^i b_i / (2^n - 1)$
    \item Constraints are converted to penalty terms in the objective
\end{enumerate}

\textbf{Mathematical Representation:}

Original CQM:
\begin{align*}
\text{minimize} \quad & f(x, y) \\
\text{subject to} \quad & g_i(x, y) \leq 0 \quad i = 1, \ldots, m \\
& x \in [L, U], \quad y \in \{0,1\}
\end{align*}

Converted BQM:
$$\text{minimize} \quad f(b, y) + \lambda \sum_{i=1}^m \max(0, g_i(b, y))^2$$

where $\lambda$ is a penalty coefficient ensuring constraint satisfaction.

\textbf{Timing Information:}
\begin{itemize}
    \item \texttt{bqm\_conversion\_time}: Time to discretize continuous variables
    \item \texttt{hybrid\_time}: Total hybrid solver time (includes QPU)
    \item \texttt{qpu\_access\_time}: Actual time spent on quantum processing unit
\end{itemize}

\textbf{Returns:} Tuple $(\text{sampleset}, \text{hybrid\_time}, \text{qpu\_time}, \text{bqm\_conversion\_time}, \text{invert})$

The \texttt{invert} function maps BQM solutions back to CQM variable space.

\subsection{\texttt{solve\_with\_gurobi\_qubo(bqm, ...)}}

\textbf{Purpose:} Solves a Binary Quadratic Model using Gurobi's native QUBO solver from \texttt{gurobi\_optimods}.

\textbf{Mathematical Model:}

A Binary Quadratic Model (BQM) has the form:
$$E(x) = \sum_i h_i x_i + \sum_{i<j} J_{ij} x_i x_j + c$$

where:
\begin{itemize}
    \item $x_i \in \{0, 1\}$: Binary decision variables
    \item $h_i$: Linear biases (coefficients)
    \item $J_{ij}$: Quadratic interaction strengths
    \item $c$: Constant offset
\end{itemize}

The QUBO (Quadratic Unconstrained Binary Optimization) formulation:
$$\text{minimize} \quad x^T Q x + c$$

where $Q$ is an upper-triangular matrix with:
\begin{itemize}
    \item $Q_{ii} = h_i$ (diagonal: linear terms)
    \item $Q_{ij} = J_{ij}$ for $i < j$ (off-diagonal: quadratic interactions)
\end{itemize}

\textbf{Conversion Process:}
\begin{enumerate}
    \item Extract QUBO dictionary from BQM: \texttt{Q, offset = bqm.to\_qubo()}
    \item Map variables to indices: $\text{var} \rightarrow \text{idx}$
    \item Construct matrix $Q \in \mathbb{R}^{n \times n}$
    \item Solve using Gurobi's QUBO optimizer
    \item Map solution indices back to variable names
\end{enumerate}

\textbf{Gurobi Parameters:}
\begin{itemize}
    \item Threads=0: Use all available threads
    \item TimeLimit: User-specified (default 100 seconds)
\end{itemize}

\textbf{Returns:} Dictionary containing:
\begin{itemize}
    \item \texttt{status}: Solution status
    \item \texttt{solution}: Dictionary mapping variable names to binary values
    \item \texttt{bqm\_energy}: BQM energy $E(x)$ (includes penalties)
    \item \texttt{objective\_value}: Reconstructed CQM objective (if parameters provided)
    \item \texttt{solve\_time}: Wall-clock solve time
    \item \texttt{validation}: Constraint validation results (if config provided)
\end{itemize}

\subsection{\texttt{main(scenario, land\_method, n\_units, total\_land)}}

\textbf{Purpose:} Main execution function orchestrating the complete optimization workflow.

\textbf{Workflow:}
\begin{enumerate}
    \item \textbf{Load Scenario:} Load food data, weights, and constraints
    \item \textbf{Generate Land:} Create land distribution based on method
    \begin{itemize}
        \item \texttt{even\_grid}: $n$ plots with $a_p = \frac{\text{total\_land}}{n}$ each
        \item \texttt{uneven\_distribution}: Farms with realistic size distribution
    \end{itemize}
    \item \textbf{Create Model:} Build CQM based on land method
    \item \textbf{Save CQM:} Serialize model to file
    \item \textbf{Solve with PuLP:} Classical optimization with Gurobi
    \item \textbf{Solve with D-Wave:} Quantum-classical hybrid (binary only)
    \item \textbf{Save Results:} Store all solutions and metadata
    \item \textbf{Create Manifest:} Generate run summary
\end{enumerate}

\textbf{Arguments:}
\begin{itemize}
    \item \texttt{scenario}: Problem instance ('simple', 'intermediate', 'full')
    \item \texttt{land\_method}: Generation method ('even\_grid', 'uneven\_distribution')
    \item \texttt{n\_units}: Number of farms/plots
    \item \texttt{total\_land}: Total land area (hectares)
\end{itemize}

\textbf{Returns:} Path to manifest file containing all run metadata.

\subsection{\texttt{calculate\_model\_complexity(formulation\_type, ...)}}

\textbf{Purpose:} Calculate comprehensive complexity metrics for a given formulation to enable comparison with benchmark papers in optimization literature.

\textbf{Arguments:}
\begin{itemize}
    \item \texttt{formulation\_type}: 'continuous' or 'binary'
    \item \texttt{n\_farms}: Number of farms/plots
    \item \texttt{n\_foods}: Number of crops/foods
    \item \texttt{n\_food\_groups}: Number of food groups (optional)
    \item \texttt{n\_crops\_with\_min\_area}: Number of crops with minimum area constraints
    \item \texttt{n\_crops\_with\_max\_area}: Number of crops with maximum area constraints
    \item \texttt{has\_food\_group\_constraints}: Whether food group constraints are active
\end{itemize}

\textbf{Returns:} Dictionary containing:
\begin{itemize}
    \item \texttt{n\_variables}: Total number of decision variables
    \item \texttt{n\_binary\_vars}: Number of binary variables
    \item \texttt{n\_continuous\_vars}: Number of continuous variables
    \item \texttt{n\_constraints}: Total number of constraints
    \item \texttt{n\_linear\_coefficients}: Number of non-zero linear coefficients
    \item \texttt{n\_quadratic\_coefficients}: Number of non-zero quadratic coefficients
    \item \texttt{problem\_class}: Classification (LP, MILP, MINLP, BIP, etc.)
\end{itemize}

\textbf{Mathematical Formulas:}

For \textbf{continuous} formulation:
\begin{align*}
n_{\text{vars}} &= 2|F||C| \\
n_{\text{const}} &= |F| + 2|F||C| + 2|F||G| \\
n_{\text{linear}} &\approx 6|F||C| + 2|F|\bar{n}_g|G| \\
n_{\text{quadratic}} &= 2|F||C|
\end{align*}

For \textbf{binary} formulation:
\begin{align*}
n_{\text{vars}} &= |P||C| \\
n_{\text{const}} &= |P| + n_{\min} + n_{\max} + 2|P||G| \\
n_{\text{linear}} &\approx 2|P||C| + |P|(n_{\min} + n_{\max}) + 2|P|\bar{n}_g|G| \\
n_{\text{quadratic}} &= 0
\end{align*}

\textbf{Usage:} This function is called automatically during model creation to generate benchmark-ready statistics. Results are included in the constraints JSON file and printed as a comparison table.

\subsection{\texttt{print\_model\_complexity\_comparison(continuous, binary)}}

\textbf{Purpose:} Generate a formatted comparison table showing complexity metrics side-by-side for both formulations.

\textbf{Arguments:}
\begin{itemize}
    \item \texttt{continuous\_complexity}: Dictionary from \texttt{calculate\_model\_complexity} for continuous formulation
    \item \texttt{binary\_complexity}: Dictionary from \texttt{calculate\_model\_complexity} for binary formulation
\end{itemize}

\textbf{Output:} Prints three sections:
\begin{enumerate}
    \item \textbf{Model Complexity Comparison:} Side-by-side metrics
    \item \textbf{Complexity Reduction Analysis:} Percentage reductions
    \item \textbf{Quadratic Elimination:} Whether bilinear terms are removed
\end{enumerate}

\textbf{Sample Output:}
\begin{verbatim}
================================================================================
MODEL COMPLEXITY COMPARISON
================================================================================

Metric                                   Continuous           Binary              
--------------------------------------------------------------------------------
Problem Class                            MINLP                BIP                 
Total Variables                          500                  250                 
  - Continuous Variables                 250                  0                   
  - Binary Variables                     250                  250                 
Total Constraints                        675                  175                 
Linear Coefficients                      1500                 650                 
Quadratic Coefficients (bilinear)        500                  0                   

================================================================================
COMPLEXITY REDUCTION ANALYSIS
================================================================================
Variable Reduction                           50.00%
Constraint Reduction                         74.07%
Linear Coefficient Reduction                 56.67%
Quadratic Terms Eliminated                   YES (100%)
\end{verbatim}

\section{Formulation Comparison}

\subsection{Key Differences}

\begin{center}
\begin{tabular}{|l|p{5cm}|p{5cm}|}
\hline
\textbf{Aspect} & \textbf{Continuous} & \textbf{Binary} \\
\hline
Variables & $A_{f,c} \in \mathbb{R}^+$, $Y_{f,c} \in \{0,1\}$ & $Y_{p,c} \in \{0,1\}$ only \\
\hline
Land & Varies by farm & Equal per plot \\
\hline
Objective & Area-weighted value & Plot assignment value \\
\hline
Constraints & Land limits, min/max area linking, food groups & Plot assignment, min/max plots per crop, food groups \\
\hline
Min/Max Area & Bilinear linking constraints & Discrete plot count constraints \\
\hline
Complexity & MINLP (mixed-integer) & BIP (pure binary) \\
\hline
D-Wave & CQM solver (limited QPU) & BQM solver (more QPU) \\
\hline
Scalability & Better for few farms & Better for many small plots \\
\hline
\end{tabular}
\end{center}

\subsection{When to Use Each}

\textbf{Continuous Formulation:}
\begin{itemize}
    \item Realistic farm scenarios with varying sizes
    \item When precise area allocation is important
    \item When farms can be partially utilized
    \item Smaller number of larger units
\end{itemize}

\textbf{Binary Formulation:}
\begin{itemize}
    \item Grid-based land division (equal plots)
    \item When discrete assignment is acceptable
    \item Maximum QPU utilization on D-Wave
    \item Larger number of smaller units
    \item Better scaling for quantum solvers
\end{itemize}

\section{Constraint Metadata Structure}

The \texttt{constraint\_metadata} dictionary stores detailed information about each constraint for validation:

\subsection{Continuous Formulation}

\begin{verbatim}
{
  'land_availability': {
    farm: {
      'type': 'land_availability',
      'farm': farm_name,
      'max_land': L_f
    }
  },
  'min_area_if_selected': {
    (farm, food): {
      'type': 'min_area_if_selected',
      'farm': farm_name,
      'food': crop_name,
      'min_area': A_min
    }
  },
  'max_area_if_selected': {
    (farm, food): {
      'type': 'max_area_if_selected',
      'farm': farm_name,
      'food': crop_name,
      'max_land': L_f
    }
  },
  'food_group_min': {
    (group, farm): {
      'type': 'food_group_min',
      'group': group_name,
      'farm': farm_name,
      'min_foods': N_min,
      'foods_in_group': [crop1, crop2, ...]
    }
  },
  'food_group_max': {
    (group, farm): {
      'type': 'food_group_max',
      'group': group_name,
      'farm': farm_name,
      'max_foods': N_max,
      'foods_in_group': [crop1, crop2, ...]
    }
  }
}
\end{verbatim}

\subsection{Binary Formulation}

\begin{verbatim}
{
  'plantation_limit': {
    plot: {
      'type': 'land_unit_assignment',
      'farm': plot_name,
      'area_ha': a_p
    }
  },
  'min_plots_per_crop': {
    crop: {
      'type': 'min_plots_per_crop',
      'food': crop_name,
      'min_area_ha': A_min,
      'plot_area_ha': a_p,
      'min_plots': ceil(A_min / a_p)
    }
  },
  'max_plots_per_crop': {
    crop: {
      'type': 'max_plots_per_crop',
      'food': crop_name,
      'max_area_ha': A_max,
      'plot_area_ha': a_p,
      'max_plots': floor(A_max / a_p)
    }
  },
  'food_group_min': { ... },  # Same as continuous
  'food_group_max': { ... }   # Same as continuous
}
\end{verbatim}

\section{Model Complexity Analysis}

\subsection{Complexity Metrics}

For benchmarking and comparison with optimization literature, the script provides comprehensive complexity metrics using the \texttt{calculate\_model\_complexity()} function.

\subsubsection{Continuous Formulation Complexity}

Given $|F|$ farms, $|C|$ crops, and $|G|$ food groups:

\begin{itemize}
    \item \textbf{Variables:} $2|F||C|$ total
    \begin{itemize}
        \item Continuous: $|F||C|$ area variables $A_{f,c}$
        \item Binary: $|F||C|$ selection variables $Y_{f,c}$
    \end{itemize}
    
    \item \textbf{Constraints:} $|F| + 2|F||C| + 2|F||G|$
    \begin{itemize}
        \item Land availability: $|F|$
        \item Minimum area linking: $|F||C|$
        \item Maximum area linking: $|F||C|$
        \item Food group (min \& max): $2|F||G|$
    \end{itemize}
    
    \item \textbf{Linear Coefficients:} $\approx 6|F||C| + 2|F|\bar{n}_g|G|$
    
    where $\bar{n}_g$ is the average number of crops per food group.
    
    \item \textbf{Quadratic Coefficients:} $2|F||C|$ (bilinear terms $A_{f,c} \cdot Y_{f,c}$)
    
    \item \textbf{Problem Class:} MINLP (Mixed-Integer Nonlinear Program)
\end{itemize}

\subsubsection{Binary Formulation Complexity}

Given $|P|$ plots, $|C|$ crops, $|G|$ food groups, $n_{min}$ crops with minimum area, and $n_{max}$ crops with maximum area:

\begin{itemize}
    \item \textbf{Variables:} $|P||C|$ total (all binary)
    \begin{itemize}
        \item Binary: $|P||C|$ assignment variables $Y_{p,c}$
        \item Continuous: 0
    \end{itemize}
    
    \item \textbf{Constraints:} $|P| + n_{min} + n_{max} + 2|P||G|$
    \begin{itemize}
        \item Plot assignment: $|P|$
        \item Minimum plots per crop: $n_{min}$
        \item Maximum plots per crop: $n_{max}$
        \item Food group (min \& max): $2|P||G|$
    \end{itemize}
    
    \item \textbf{Linear Coefficients:} $\approx 2|P||C| + |P|(n_{min} + n_{max}) + 2|P|\bar{n}_g|G|$
    
    \item \textbf{Quadratic Coefficients:} 0 (pure linear program in binary space)
    
    \item \textbf{Problem Class:} BIP (Binary Integer Program - pure 0-1 optimization)
\end{itemize}

\subsection{Complexity Reduction Analysis}

The binary formulation achieves significant complexity reduction compared to continuous:

\begin{enumerate}
    \item \textbf{Variable Reduction:} $50\%$ reduction
    $$\text{Variables}_{\text{binary}} = \frac{|P||C|}{2|F||C|} = 50\% \text{ of continuous (when } |P| = |F|)$$
    
    \item \textbf{Constraint Reduction:} Depends on problem parameters
    $$\text{Constraints}_{\text{binary}} = \frac{|P| + n_{min} + n_{max} + 2|P||G|}{|F| + 2|F||C| + 2|F||G|}$$
    
    For typical problems where $n_{min}, n_{max} \ll |F||C|$:
    $$\text{Constraint reduction} \approx \frac{|P|(1 + 2|G|)}{|F|(1 + 2|C| + 2|G|)} \approx \frac{1 + 2|G|}{1 + 2|C| + 2|G|}$$
    
    \item \textbf{Quadratic Terms Eliminated:} $100\%$
    
    All bilinear terms eliminated, converting MINLP to BIP.
    
    \item \textbf{Computational Advantage:}
    \begin{itemize}
        \item No variable discretization needed for quantum solvers
        \item Direct QUBO formulation possible
        \item Better QPU utilization on D-Wave systems
        \item Faster branch-and-bound for classical solvers
    \end{itemize}
\end{enumerate}

\subsection{Benchmark Comparison Table}

\begin{center}
\begin{tabular}{|l|c|c|c|}
\hline
\textbf{Metric} & \textbf{Continuous} & \textbf{Binary} & \textbf{Reduction} \\
\hline
Variables & $2|F||C|$ & $|P||C|$ & $50\%$ \\
Binary Variables & $|F||C|$ & $|P||C|$ & $0\%$ \\
Continuous Variables & $|F||C|$ & $0$ & $100\%$ \\
Constraints & $O(|F||C|)$ & $O(|P| + |C|)$ & $\approx 50-90\%$ \\
Linear Coefficients & $O(|F||C|)$ & $O(|P||C|)$ & $\approx 40-70\%$ \\
Quadratic Terms & $2|F||C|$ & $0$ & $100\%$ \\
\hline
\textbf{Problem Class} & MINLP & BIP & Non-convex $\to$ Convex relaxation \\
\hline
\end{tabular}
\end{center}

\subsection{Example: 25 Plots, 10 Crops, 3 Food Groups}

\textbf{Continuous Formulation:}
\begin{itemize}
    \item Variables: $2 \times 25 \times 10 = 500$ (250 continuous + 250 binary)
    \item Constraints: $25 + 2(25)(10) + 2(25)(3) = 675$
    \item Quadratic terms: $2 \times 25 \times 10 = 500$ bilinear terms
\end{itemize}

\textbf{Binary Formulation:}
\begin{itemize}
    \item Variables: $25 \times 10 = 250$ (all binary)
    \item Constraints: $25 + n_{min} + n_{max} + 2(25)(3) \approx 175$ (assuming $n_{min} + n_{max} \approx 0$)
    \item Quadratic terms: $0$
\end{itemize}

\textbf{Reductions:}
\begin{itemize}
    \item Variables: $50\%$ reduction
    \item Constraints: $74\%$ reduction
    \item Quadratic terms: $100\%$ elimination
\end{itemize}

\subsection{Implications for Quantum Computing}

The binary formulation's complexity advantages are particularly significant for quantum annealing:

\begin{enumerate}
    \item \textbf{Direct QUBO Mapping:} No discretization overhead
    \item \textbf{QPU Efficiency:} More qubits available for problem variables
    \item \textbf{Embedding Quality:} Simpler graph structure for quantum annealer
    \item \textbf{Solution Quality:} Fewer approximation errors from discretization
\end{enumerate}

\section{Output Files}

\subsection{CQM Models}

\textbf{File:} \texttt{CQM\_Models/cqm\_\{type\}\_\{scenario\}\_\{timestamp\}.cqm}

Binary serialization of the Constrained Quadratic Model using D-Wave's format.

\subsection{PuLP Results}

\textbf{File:} \texttt{PuLP\_Results/pulp\_\{type\}\_\{scenario\}\_\{timestamp\}.json}

\begin{verbatim}
{
  "status": "Optimal",
  "objective_value": 12.345,
  "solve_time": 1.234,
  "areas": {"farm1_crop1": 10.5, ...},      # Continuous only
  "selections": {"farm1_crop1": 1, ...},    # Continuous only
  "plantations": {"plot1_crop1": 1, ...}    # Binary only
}
\end{verbatim}

\subsection{D-Wave Results}

\textbf{File:} \texttt{DWave\_Results/dwave\_\{type\}\_\{scenario\}\_\{timestamp\}.json}

\begin{verbatim}
{
  "status": "Optimal",
  "objective_value": 12.345,
  "solve_time": 5.678,
  "qpu_access_time": 0.0012,
  "bqm_conversion_time": 0.234,
  "num_samples": 100,
  "formulation": "BQUBO (binary only)"
}
\end{verbatim}

\subsection{Constraints}

\textbf{File:} \texttt{Constraints/constraints\_\{scenario\}\_\{timestamp\}.json}

Contains complete problem specification including constraint metadata, food data, configuration parameters, and formulation type.

\subsection{Run Manifest}

\textbf{File:} \texttt{run\_manifest\_\{scenario\}\_\{timestamp\}.json}

Summary of the entire optimization run with paths to all output files and comparative results.

\section{Mathematical Properties}

\subsection{Objective Function Properties}

\begin{enumerate}
    \item \textbf{Linearity:} Both formulations have linear objectives in their respective variable types
    \item \textbf{Bounded:} $Z \in [-\infty, Z_{max}]$ where $Z_{max}$ depends on land availability and crop values
    \item \textbf{Normalization:} Division by total land ensures comparability across different problem sizes
    \item \textbf{Maximization:} Both formulations maximize value (negated for D-Wave minimization)
\end{enumerate}

\subsection{Constraint Properties}

\begin{enumerate}
    \item \textbf{Feasibility:} Both formulations always have at least one feasible solution (all idle)
    \item \textbf{Convexity:} 
    \begin{itemize}
        \item Continuous: Non-convex due to bilinear terms ($A \cdot Y$)
        \item Binary: Convex hull relaxation is convex
    \end{itemize}
    \item \textbf{Tightness:} Linking constraints ensure logical consistency between area and selection
\end{enumerate}

\subsection{Computational Complexity}

\begin{itemize}
    \item \textbf{Continuous:} NP-hard (mixed-integer nonlinear program)
    \item \textbf{Binary:} NP-hard (binary integer program with cardinality constraints)
    \item \textbf{Variables:}
    \begin{itemize}
        \item Continuous: $2 \times |F| \times |C|$ variables
        \item Binary: $|F| \times |C|$ variables
    \end{itemize}
    \item \textbf{Constraints:}
    \begin{itemize}
        \item Continuous: $|F| + 2|F||C| + 2|F||G|$ constraints
        \item Binary: $|F| + 2|F||G|$ constraints
    \end{itemize}
\end{itemize}

\section{Solver Comparison}

\subsection{PuLP + Gurobi}

\textbf{Algorithm:} Branch-and-bound with barrier method

\textbf{Strengths:}
\begin{itemize}
    \item Proven optimality guarantees
    \item Fast for small-to-medium problems
    \item GPU acceleration available
    \item Handles both formulations
\end{itemize}

\textbf{Limitations:}
\begin{itemize}
    \item Exponential worst-case complexity
    \item Struggles with large-scale problems
    \item Limited parallelization for discrete problems
\end{itemize}

\subsection{D-Wave Hybrid CQM}

\textbf{Algorithm:} Quantum-classical hybrid decomposition

\textbf{Strengths:}
\begin{itemize}
    \item Handles mixed-integer problems natively
    \item Good for problems with many constraints
    \item Cloud-based (no local resources needed)
\end{itemize}

\textbf{Limitations:}
\begin{itemize}
    \item Limited QPU utilization
    \item Heuristic (no optimality guarantee)
    \item Slower than classical for small problems
\end{itemize}

\subsection{D-Wave Hybrid BQM}

\textbf{Algorithm:} Quantum annealing with classical refinement

\textbf{Strengths:}
\begin{itemize}
    \item Maximum QPU utilization
    \item Excellent scaling for large binary problems
    \item Faster convergence for QUBO-amenable problems
\end{itemize}

\textbf{Limitations:}
\begin{itemize}
    \item Binary variables only (requires discretization)
    \item Constraints become penalties (soft constraints)
    \item Solution quality depends on penalty tuning
\end{itemize}

\section{Conclusion}

This script provides a comprehensive framework for solving food optimization problems with two distinct mathematical formulations:

\begin{enumerate}
    \item \textbf{Continuous Formulation:} Realistic farm-based optimization with area allocation
    \item \textbf{Binary Formulation:} Grid-based discrete assignment optimization
\end{enumerate}

Both formulations can be solved using classical (Gurobi) and quantum-enabled (D-Wave) methods, allowing for performance comparison and validation. The binary formulation is particularly well-suited for quantum computing approaches due to its QUBO structure, achieving higher QPU utilization and better scaling properties.

\end{document}
