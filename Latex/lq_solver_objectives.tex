\documentclass[12pt]{article}
\usepackage{amsmath}
\usepackage{amssymb}
\usepackage{geometry}
\usepackage{xcolor}
\usepackage{tcolorbox}

\geometry{margin=1in}

\title{Linear-Quadratic Objective Formulations\\Solver-Specific Representations}
\author{OQI-UC002-DWave}
\date{\today}

\begin{document}

\maketitle

\section{Problem Overview}

The Linear-Quadratic (LQ) optimization problem combines:
\begin{itemize}
    \item \textbf{Linear term}: Area-weighted food attributes
    \item \textbf{Quadratic term}: Synergy bonus for planting crops from the same food group on the same farm
\end{itemize}

\subsection{Decision Variables}

\begin{itemize}
    \item $A_{f,c} \in \mathbb{R}_+$: Continuous area (hectares) allocated to crop $c$ on farm $f$
    \item $Y_{f,c} \in \{0,1\}$: Binary indicator whether crop $c$ is selected on farm $f$
    \item $f \in \mathcal{F}$: Set of farms ($|\mathcal{F}| = N_f$)
    \item $c \in \mathcal{C}$: Set of crops/foods ($|\mathcal{C}| = N_c$)
\end{itemize}

\subsection{Parameters}

\begin{itemize}
    \item $w_k$: Weight for attribute $k$ (nutritional\_value, nutrient\_density, environmental\_impact, affordability, sustainability)
    \item $a_{k,c}$: Value of attribute $k$ for crop $c$
    \item $w_s$: Synergy bonus weight (typically $w_s = 0.1$)
    \item $s_{c_1,c_2}$: Synergy value between crops $c_1$ and $c_2$ (non-zero only if in same food group)
    \item $B_c = \sum_k w_k \cdot a_{k,c}$: Composite benefit of crop $c$
\end{itemize}

\section{General Mathematical Formulation}

The original LQ objective function with area normalization is:

\begin{tcolorbox}[colback=blue!5!white, colframe=blue!75!black, title=General LQ Objective (Area-Normalized)]
\begin{equation}
\max \quad \underbrace{\sum_{f \in \mathcal{F}} \sum_{c \in \mathcal{C}} B_c \cdot A_{f,c}}_{\text{Linear term}} + \underbrace{\frac{w_s}{A_{\text{total}}} \sum_{f \in \mathcal{F}} \sum_{\substack{c_1, c_2 \in \mathcal{C} \\ c_1 < c_2}} s_{c_1,c_2} \cdot A_{f,c_1} \cdot A_{f,c_2}}_{\text{Quadratic synergy term (area-normalized)}}
\label{eq:general_lq}
\end{equation}
\end{tcolorbox}

where:
\begin{itemize}
    \item $A_{\text{total}} = \sum_{f \in \mathcal{F}} L_f$ is the total available land area
    \item The quadratic term is proportional to the product of allocated areas, normalized by total area
    \item This ensures the objective scales properly regardless of farm sizes
\end{itemize}

where:
\begin{equation}
B_c = \sum_{k} w_k \cdot a_{k,c}
\end{equation}

Expanded form:
\begin{align}
B_c &= w_{\text{nutr}} \cdot a_{\text{nutr},c} + w_{\text{dens}} \cdot a_{\text{dens},c} - w_{\text{env}} \cdot a_{\text{env},c} \nonumber \\
    &\quad + w_{\text{afford}} \cdot a_{\text{afford},c} + w_{\text{sust}} \cdot a_{\text{sust},c}
\end{align}

\section{Solver-Specific Formulations}

\subsection{DWave CQM (Constrained Quadratic Model)}

\textbf{Implementation:} \texttt{create\_cqm()} in \texttt{solver\_runner\_LQ.py}

DWave's CQM natively supports quadratic objectives, but \textbf{does not allow products of continuous variables} ($A \cdot A$ not permitted). Therefore, we approximate the area product with the farm's total available area $L_f$:

\begin{tcolorbox}[colback=red!5!white, colframe=red!75!black, title=DWave CQM Objective (Approximation)]
\begin{equation}
\max \quad \sum_{f \in \mathcal{F}} \sum_{c \in \mathcal{C}} B_c \cdot A_{f,c} + \frac{w_s}{A_{\text{total}}} \sum_{f \in \mathcal{F}} L_f \sum_{\substack{c_1, c_2 \in \mathcal{C} \\ c_1 < c_2}} s_{c_1,c_2} \cdot Y_{f,c_1} \cdot Y_{f,c_2}
\label{eq:dwave_cqm}
\end{equation}
\end{tcolorbox}

\textbf{Approximation:} Since $A_{f,c_1} \cdot A_{f,c_2}$ is not allowed, we approximate it with $L_f \cdot Y_{f,c_1} \cdot Y_{f,c_2}$ where $L_f$ is the farm's capacity.

\textbf{Code snippet:}
\begin{verbatim}
total_area = sum(land_availability.values())

# Linear term
for farm in farms:
    for food in foods:
        objective += B_c * A[(farm, food)]

# Quadratic term (approximated with farm area)
optimizer = SynergyOptimizer(synergy_matrix, foods)
for farm in farms:
    farm_area = land_availability[farm]
    for crop1, crop2, boost_value in optimizer.iter_pairs_with_names():
        objective += (synergy_bonus_weight * boost_value * farm_area * 
                     Y[(farm, crop1)] * Y[(farm, crop2)] / total_area)

cqm.set_objective(-objective)
\end{verbatim}

\textbf{Variable count:} $2 N_f N_c$ variables ($N_f N_c$ continuous $A$ variables + $N_f N_c$ binary $Y$ variables)

\textbf{Constraint count:} Base constraints only (no linearization needed)


\subsection{PuLP (Mixed-Integer Quadratic Programming)}

\textbf{Implementation:} \texttt{solve\_with\_pulp()} in \texttt{solver\_runner\_LQ.py}

PuLP uses \textbf{Gurobi} as the backend solver. Gurobi supports quadratic objectives (MIQP), so we can use area products $A_{f,c_1} \cdot A_{f,c_2}$ directly without linearization:

\begin{tcolorbox}[colback=green!5!white, colframe=green!75!black, title=PuLP Quadratic Objective (Area Products)]
\begin{equation}
\max \quad \sum_{f \in \mathcal{F}} \sum_{c \in \mathcal{C}} B_c \cdot A_{f,c} + \frac{w_s}{A_{\text{total}}} \sum_{f \in \mathcal{F}} \sum_{\substack{c_1, c_2 \in \mathcal{C} \\ c_1 < c_2}} s_{c_1,c_2} \cdot A_{f,c_1} \cdot A_{f,c_2}
\label{eq:pulp}
\end{equation}
\end{tcolorbox}

\textbf{Note:} Unlike the previous McCormick linearization approach, we now use quadratic area products directly since Gurobi handles MIQP problems efficiently.

\textbf{Code snippet:}
\begin{verbatim}
total_area = sum(land_availability.values())

# Objective - Linear term
objective_terms = [B_c * A_pulp[(f, c)] for f in farms for c in foods]

# Objective - Quadratic synergy (area products)
synergy_terms = []
for f, crop1, crop2, boost_value in synergy_pairs:
    synergy_terms.append(synergy_bonus_weight * boost_value * 
                        A_pulp[(f, crop1)] * A_pulp[(f, crop2)] / total_area)

goal = pl.lpSum(objective_terms) + pl.lpSum(synergy_terms)
model += goal, "Objective"
\end{verbatim}

\textbf{Variable count:} $2 N_f N_c$ variables ($N_f N_c$ continuous $A$ variables + $N_f N_c$ binary $Y$ variables)

\textbf{Constraint count:} Base constraints only (no linearization constraints needed)


\subsection{Pyomo (Mixed-Integer Quadratic Programming)}

\textbf{Implementation:} \texttt{solve\_with\_pyomo()} in \texttt{solver\_runner\_LQ.py}

Pyomo uses \textbf{Gurobi} as the backend solver (same as PuLP). Gurobi handles MIQP problems directly when the objective is quadratic, so Pyomo uses area products $A_{f,c_1} \cdot A_{f,c_2}$ without linearization:

\begin{tcolorbox}[colback=purple!5!white, colframe=purple!75!black, title=Pyomo Native Quadratic Objective (Area Products)]
\begin{equation}
\max \quad \sum_{f \in \mathcal{F}} \sum_{c \in \mathcal{C}} B_c \cdot A_{f,c} + \frac{w_s}{A_{\text{total}}} \sum_{f \in \mathcal{F}} \sum_{\substack{c_1, c_2 \in \mathcal{C} \\ c_1 < c_2}} s_{c_1,c_2} \cdot A_{f,c_1} \cdot A_{f,c_2}
\label{eq:pyomo}
\end{equation}
\end{tcolorbox}

This is identical to the PuLP formulation. Pyomo passes the quadratic objective directly to Gurobi's MIQP solver.

\textbf{Code snippet:}
\begin{verbatim}
total_area = sum(land_availability.values())

def objective_rule(m):
    obj = 0
    
    # Linear term
    for f in m.farms:
        for c in m.foods:
            obj += B_c * m.A[f, c]
    
    # Quadratic synergy (area products)
    optimizer = SynergyOptimizer(synergy_matrix, foods)
    for crop1, crop2, boost_value in optimizer.iter_pairs_with_names():
        for f in m.farms:
            obj += (synergy_bonus_weight * boost_value * 
                   m.A[f, crop1] * m.A[f, crop2] / total_area)
    
    return obj

model.obj = pyo.Objective(rule=objective_rule, sense=pyo.maximize)
\end{verbatim}

\textbf{Variable count:} $2 N_f N_c$ variables ($N_f N_c$ continuous $A$ variables + $N_f N_c$ binary $Y$ variables)

\textbf{Constraint count:} Base constraints only (no linearization needed)
\max \quad \sum_{f \in \mathcal{F}} \sum_{c \in \mathcal{C}} B_c \cdot A_{f,c} + w_s \sum_{f \in \mathcal{F}} \sum_{\substack{c_1, c_2 \in \mathcal{C} \\ c_1 < c_2}} s_{c_1,c_2} \cdot Y_{f,c_1} \cdot Y_{f,c_2}
\label{eq:pyomo}
\end{equation}
\end{tcolorbox}

This is identical to the DWave formulation. Pyomo passes the quadratic objective directly to Gurobi's MIQP solver.

\textbf{Code snippet:}
\begin{verbatim}
def objective_rule(m):
    obj = 0
    
    # Linear term
    for f in m.farms:
        for c in m.foods:
            obj += B_c * m.A[f, c]
    
    # Quadratic synergy bonus (native quadratic)
    optimizer = SynergyOptimizer(synergy_matrix, foods)
    for crop1, crop2, boost_value in optimizer.iter_pairs_with_names():
        for f in m.farms:
            obj += synergy_bonus_weight * boost_value * m.Y[f, crop1] * m.Y[f, crop2]
    
    return obj

model.obj = pyo.Objective(rule=objective_rule, sense=pyo.maximize)
\end{verbatim}

\textbf{Variable count:} $2 N_f N_c$ variables ($N_f N_c$ continuous $A$ variables + $N_f N_c$ binary $Y$ variables)

\textbf{Constraint count:} Base constraints only (no linearization needed)


\section{Comparison Summary}

\begin{table}[h]
\centering
\begin{tabular}{|l|c|c|c|}
\hline
\textbf{Solver} & \textbf{Quadratic Form} & \textbf{Variables} & \textbf{Constraints} \\
\hline
DWave CQM & $L_f \cdot Y \cdot Y$ (approx) & $2N_f N_c$ & Base \\
PuLP (Gurobi MIQP) & $A \cdot A$ (exact) & $2N_f N_c$ & Base \\
Pyomo (Gurobi MIQP) & $A \cdot A$ (exact) & $2N_f N_c$ & Base \\
\hline
\end{tabular}
\caption{Solver formulation comparison. All use area-normalized synergy: $\frac{1}{A_{\text{total}}} \sum \text{(area products)}$.}
\end{table}

\subsection{Key Differences}

\begin{enumerate}
    \item \textbf{DWave CQM}: Cannot use $A \cdot A$ products (CQM limitation), so approximates with $L_f \cdot Y \cdot Y$ where $L_f$ is farm capacity. Uses quantum-inspired annealing.
    
    \item \textbf{PuLP}: Uses Gurobi's MIQP solver with exact area products $A_{f,c_1} \cdot A_{f,c_2}$. No linearization needed.
    
    \item \textbf{Pyomo}: Uses Gurobi's MIQP solver with exact area products $A_{f,c_1} \cdot A_{f,c_2}$. Same formulation as PuLP.
\end{enumerate}

\subsection{Solution Equivalence}

PuLP and Pyomo should produce \textbf{identical solutions} since they use the same formulation and solver (Gurobi MIQP).

DWave's solution may differ slightly due to the approximation $A_{f,c_1} \cdot A_{f,c_2} \approx L_f \cdot Y_{f,c_1} \cdot Y_{f,c_2}$, but should be close for problems where selected crops utilize most of the farm's capacity.

\section{Numerical Example}

Consider a small instance with:
\begin{itemize}
    \item $N_f = 2$ farms: $\mathcal{F} = \{\text{Farm1}, \text{Farm2}\}$ with $L_{\text{F1}} = 10$ ha, $L_{\text{F2}} = 15$ ha
    \item $A_{\text{total}} = 25$ ha
    \item $N_c = 3$ crops: $\mathcal{C} = \{\text{Wheat}, \text{Rice}, \text{Corn}\}$
    \item All crops in same food group (Grains) with $s_{c_1,c_2} = 0.1$ for $c_1 \neq c_2$
    \item $B_{\text{Wheat}} = 0.8$, $B_{\text{Rice}} = 0.7$, $B_{\text{Corn}} = 0.6$
    \item $w_s = 0.1$
\end{itemize}

\subsection{DWave CQM Formulation (Approximation)}

\begin{align}
\max \quad &0.8(A_{\text{F1,Wheat}} + A_{\text{F2,Wheat}}) + 0.7(A_{\text{F1,Rice}} + A_{\text{F2,Rice}}) \nonumber \\
&+ 0.6(A_{\text{F1,Corn}} + A_{\text{F2,Corn}}) \nonumber \\
&+ \frac{0.01}{25} \Big[ 10 \left( Y_{\text{F1,Wheat}} \cdot Y_{\text{F1,Rice}} + Y_{\text{F1,Wheat}} \cdot Y_{\text{F1,Corn}} + Y_{\text{F1,Rice}} \cdot Y_{\text{F1,Corn}} \right) \nonumber \\
&\quad\quad\quad + 15 \left( Y_{\text{F2,Wheat}} \cdot Y_{\text{F2,Rice}} + Y_{\text{F2,Wheat}} \cdot Y_{\text{F2,Corn}} + Y_{\text{F2,Rice}} \cdot Y_{\text{F2,Corn}} \right) \Big]
\end{align}

Variables: $2 \times 2 \times 3 = 12$ variables (6 $A$ + 6 $Y$)

\subsection{PuLP/Pyomo Formulation (Exact Area Products)}

\begin{align}
\max \quad &0.8(A_{\text{F1,Wheat}} + A_{\text{F2,Wheat}}) + 0.7(A_{\text{F1,Rice}} + A_{\text{F2,Rice}}) \nonumber \\
&+ 0.6(A_{\text{F1,Corn}} + A_{\text{F2,Corn}}) \nonumber \\
&+ \frac{0.01}{25} \Big[ A_{\text{F1,Wheat}} \cdot A_{\text{F1,Rice}} + A_{\text{F1,Wheat}} \cdot A_{\text{F1,Corn}} + A_{\text{F1,Rice}} \cdot A_{\text{F1,Corn}} \nonumber \\
&\quad\quad\quad + A_{\text{F2,Wheat}} \cdot A_{\text{F2,Rice}} + A_{\text{F2,Wheat}} \cdot A_{\text{F2,Corn}} + A_{\text{F2,Rice}} \cdot A_{\text{F2,Corn}} \Big]
\end{align}

Variables: $2 \times 2 \times 3 = 12$ variables (6 $A$ + 6 $Y$)

Constraints: Base only (no linearization)


\section{Conclusion}

The Linear-Quadratic formulation with area normalization demonstrates how different optimization solvers handle quadratic area products:

\begin{itemize}
    \item \textbf{Quantum-inspired solvers (DWave)}: Cannot use $A \cdot A$ products, so approximate with $L_f \cdot Y \cdot Y$ (farm capacity × binary selection)
    \item \textbf{Classical MIQP solvers (PuLP/Pyomo with Gurobi)}: Native support for $A \cdot A$ products via quadratic cone cuts and branch-and-bound
\end{itemize}

All formulations use area normalization: dividing the quadratic synergy term by $A_{\text{total}}$ ensures the objective scales properly regardless of total farm sizes.

\subsection{Key Insights}

\begin{enumerate}
    \item \textbf{Area normalization}: Critical for comparing solutions across different problem sizes
    \item \textbf{DWave approximation}: $A_{f,c_1} \cdot A_{f,c_2} \approx L_f \cdot Y_{f,c_1} \cdot Y_{f,c_2}$ works well when farms are fully utilized
    \item \textbf{No linearization needed}: Modern MIQP solvers (Gurobi) handle quadratic area products directly
    \item \textbf{Synergy scaling}: The $1/A_{\text{total}}$ factor prevents larger farms from dominating the synergy term
\end{enumerate}

The SynergyOptimizer preprocessing step reduces complexity by precomputing the $P$ synergy pairs (where $P \ll N_c^2$ since only same-group crops have synergy), providing significant speedup during model construction for all three solvers.

\end{document}
