
\section{Implementation of a Proof of Concept}
\label{sec:poc}

The food optimization problem addresses the allocation of agricultural land across multiple farms to grow various crops, maximizing a weighted combination of nutritional value, sustainability, affordability, and other attributes while satisfying land availability and diversity constraints.


\input{diagram_hardness}


{\color{blue}
\subsection{Notation}


\subsubsection{Problem Context}


In the most general form, our problem consists of:

\textbf{Sets}
\begin{itemize}
    \item $\mathcal{F}$: Set of farms with land availability $L_f$ for $f \in \mathcal{F}$
    \item $\mathcal{C}$: Set of crops with attributes (nutritional value, sustainability, etc.)
    \item $\mathcal{G}$: Set of food groups with diversity requirement
    \item $w_k \in \mathcal{W}$: Weights for different objectives $k \in \{$nutrition, sustainability, ...$\}$
    \item $v_{k,c} \in \mathcal{V}$: Values of different objectives $k \in \{$nutrition, sustainability, ...$\}$ for each food $c \in \mathcal{C}$ 
\end{itemize}

\textbf{Decision variables}
\begin{itemize}
\item $A_{f,c} \in \{0\} \cup [A_{min,c}\,, L_f]$: Continuous area allocated to crop $c$ on farm $f$
    \item $Y_{f,c} \in \{0,1\}$: Binary selection indicator for crop $c$ on farm $f$
\end{itemize}

\textbf{Common constraints}
\begin{itemize}
\item Maximum farm occupation: $$\sum_{c \in \mathcal{C}} A_{f,c} \leq L_f \quad \forall f \in \mathcal{F} $$
\item  Minimum crop allocation: $$A_{f,c} \geq A_{\min,c} \cdot Y_{f,c} \quad \forall f,c\label{eq:minarea} $$
\item  Maximum crop allocation: $$A_{f,c} \leq L_f \cdot Y_{f,c} \quad \forall f,c $$
\item Food group diversity: $$N_{\min,g} \leq \sum_{f \in \mathcal{F}}\sum_{c \in \mathcal{G}_g} Y_{f,c} \leq N_{\max,g} \quad \forall g $$ 

\item Convexity: $$\sum_{k} w_k = 1$$
\end{itemize}

\newpage
\subsection{Objective Implementations}
In this section we loosely follow what was done in \cite{Quinton2025QuantumAnnealingVersusClassicalSolvers} by slightly changing the objective to better investigate the performance of the considered solvers with varying levels of non linearity.


\subsubsection{Linear Objective}

\paragraph{Mathematical Formulation}\mbox{}\\

The simplest formulation uses a linear objective function:

\begin{equation}
\max \sum_{f \in \mathcal{F}} \sum_{c \in \mathcal{C}} \left(\sum_{k} w_k \cdot v_{k,c}\right) \cdot A_{f,c}
\label{eq:linear_obj}
\end{equation}

where we can define the benefit (per unit area) of each food $c$:

\begin{equation}
   B_c = \sum_{k} w_k \cdot v_{k,c}
   \label{eq:benefit}
\end{equation}

\paragraph{Problem Classification}\mbox{}\\
\begin{itemize}
    \item \textbf{Type:} Mixed-Integer Linear Program (MILP)
    \item \textbf{Variables:} $2|\mathcal{F}||\mathcal{C}|$ ($A$ and $Y$ variables)
    \item \textbf{Constraints:} $O(|\mathcal{F}|(1 + 2|\mathcal{C}| + 2|\mathcal{G}|))$
    \item \textbf{Linearity:} Fully linear in both objective and constraints
\end{itemize}

% \paragraph{Solution Methods}

% \subparagraph{PuLP Implementation:}
% Uses the CBC (COIN-OR Branch and Cut) solver, a state-of-the-art open-source MILP solver employing:
% \begin{itemize}
%     \item Branch-and-bound with linear relaxations
%     \item Cutting plane generation
%     \item Primal heuristics for feasible solutions
% \end{itemize}

% \subparagraph{DWave CQM Implementation:}
% Converts the MILP to a Constrained Quadratic Model (CQM) for quantum-classical hybrid solving:
% \begin{itemize}
%     \item Uses LeapHybridCQMSampler
%     \item Combines quantum annealing with classical optimization
%     \item Penalty method for constraint handling
% \end{itemize}

\paragraph{Computational Complexity}\mbox{}\\
\begin{itemize}
    \item \textbf{Time Complexity:} $O(2^n \cdot p(n))$ in worst case, where $n = |\mathcal{F}||\mathcal{C}|$ and $p(n)$ is polynomial \cite{wolsey1998integer,nemhauser1988integer}. In practice, modern MILP solvers exhibit near-linear scaling for well-structured problems through branch-and-bound with advanced preprocessing and cutting planes \cite{achterberg2007constraint,bixby2002solving}.
    \item \textbf{Space Complexity:} $O(n)$ for variable storage plus $O(n^2)$ for constraint matrix (sparse) \cite{bertsimas1997introduction}.
\end{itemize}






\newpage

\subsubsection{Non-Linear with Piecewise Approximation}

\paragraph{Mathematical Formulation}\mbox{}\\

This formulation uses a concave power function to model diminishing returns:

\begin{equation}
\max \sum_{f \in \mathcal{F}} \sum_{c \in \mathcal{C}} B_C \cdot f(A_{f,c})
\label{eq:nonlinear_obj}
\end{equation}

where $f(A) = A^\alpha$ with $\alpha = 0.548$ \cite{bai2019environmental}.

\paragraph{Piecewise Linear Approximation}\mbox{}\\

Since the power function is non-convex for MILP solvers, we use piecewise linear approximation with Special Ordered Set of type 2 (SOS2) constraints:

\textbf{Breakpoint Definition:}
For each farm-food pair $(f,c)$, define $K+1$ breakpoints:
\begin{equation}
0 = b_0 < b_1 < \cdots < b_K = L_f
\end{equation}

Typically, $K = 10$ breakpoints with uniform spacing:
\begin{equation}
b_i = \frac{i \cdot L_f}{K} \quad \text{for } i = 0, \ldots, K
\end{equation}

\textbf{Function Values at Breakpoints:}
\begin{equation}
\phi_i = f(b_i) = b_i^{0.548}
\end{equation}

\textbf{Additional Variables:}
For each $(f,c)$ pair, introduce:
\begin{itemize}
    \item $\lambda_{f,c,i} \in [0,1]$ for $i = 0,\ldots,K$: Convex combination weights
    \item $\tilde{f}_{f,c} \in \mathbb{R}$: Approximated function value
\end{itemize}

\textbf{Piecewise Constraints:}
\begin{align}
 \sum_{i=0}^{K} &\lambda_{f,c,i} \cdot b_i =A_{f,c}\label{eq:pw_area}\\
 \sum_{i=0}^{K} &\lambda_{f,c,i} \cdot \phi_i =\tilde{f}_{f,c}\label{eq:pw_func}\\
\sum_{i=0}^{K} &\lambda_{f,c,i} = 1 \label{eq:pw_convex}\\
&\lambda_{f,c,i} \geq 0, \quad \text{at most 2 consecutive } \lambda_{f,c,i} > 0 \label{eq:sos2}
\end{align}

The SOS2 constraint (\ref{eq:sos2}) ensures only adjacent breakpoints have positive weights.


\textbf{Modified Objective:}
\begin{equation}
\max \sum_{f \in \mathcal{F}} \sum_{c \in \mathcal{C}} B_C \cdot \tilde{f}_{f,c}
\label{eq:pw_obj}
\end{equation}

\paragraph{Problem Classification}\mbox{}\\
\begin{itemize}
    \item \textbf{Type:} Mixed-Integer Linear Program with SOS2 constraints
    \item \textbf{Variables:} $2|\mathcal{F}||\mathcal{C}| + (K+1)|\mathcal{F}||\mathcal{C}| + |\mathcal{F}||\mathcal{C}|$
    \begin{itemize}
        \item Base: $2|\mathcal{F}||\mathcal{C}|$ (A and Y)
        \item Lambda: $(K+1)|\mathcal{F}||\mathcal{C}|$ (typically 11n)
        \item Approximation: $|\mathcal{F}||\mathcal{C}|$ ($\tilde{f}$ variables)
    \end{itemize}
    \item \textbf{Total Variables:} $(K+4)|\mathcal{F}||\mathcal{C}| \approx 14n$ for $K=10$
    \item \textbf{Additional Constraints:} $3|\mathcal{F}||\mathcal{C}|$ piecewise constraints
\end{itemize}

% \paragraph{Solution Methods}

% \subparagraph{PuLP Implementation:}
% \begin{itemize}
%     \item Uses SOS2 variables natively supported in CBC
%     \item Branch-and-bound explores SOS2 branching decisions
%     \item Linear relaxation provides bounds
% \end{itemize}

% \subparagraph{Pyomo Implementation:}
% \begin{itemize}
%     \item Formulates as true MINLP (Mixed-Integer Non-Linear Program)
%     \item Uses specialized solvers: IPOPT, BONMIN, COUENNE
%     \item Applies outer approximation or branch-and-reduce algorithms
%     \item Can solve exact non-linear formulation without approximation
% \end{itemize}

\paragraph{Approximation Error Analysis}\mbox{}\\

The piecewise linear approximation introduces error bounded by:

\begin{equation}
\epsilon_{\max} = \max_{i=0,\ldots,K-1} \max_{x \in [b_i, b_{i+1}]} |f(x) - \hat{f}(x)|
\end{equation}

where $\hat{f}(x)$ is the piecewise linear interpolation.

For $f(x) = x^\alpha$ with uniform breakpoints and $K$ intervals:

\begin{equation}
\epsilon_{\max} = O\left(\frac{L_f^{2\alpha}}{K^2}\right)
\end{equation}

\textbf{Typical Values:}
\begin{itemize}
    \item $K = 10$: Approximation error $\approx 0.1\%$ - $0.5\%$
    \item $K = 20$: Approximation error $\approx 0.02\%$ - $0.1\%$
\end{itemize}

\paragraph{Computational Complexity}\mbox{}\\
\begin{itemize}
    \item \textbf{Variable Count:} For $n = |\mathcal{F}||\mathcal{C}|$ base problem size:
\begin{itemize}
    \item Linear solver: $2n$ variables
    \item \ref{eq:nonlinear_obj} solver: $(K+4)n \approx 14n$ variables (for $K=10$)
    \item \textbf{Increase:} $7\times$ more variables
\end{itemize}

\item \textbf{Memory:} $O(Kn)$ for lambda variables \cite{vielma2010mixed}
\item \textbf{Solve Time:} Empirically $2-4\times$ slower than linear due to SOS2 branching. The piecewise approximation complexity depends on the number of breakpoints and the convexity structure \cite{vielma2010mixed,beale1970special}.
\item \textbf{Approximation Error:} $O(K^{-2})$ for uniform breakpoints with concave functions \cite{croxton2003comparison}.
\end{itemize}



\paragraph{Plots}\mbox{}\\

\begin{figure}[!h]
    \centering
    \includegraphics[width=\linewidth]{Plots/nln_speedup_comparison.png}
    %\caption{Caption}
    \label{fig:NLN_results}
\end{figure}


\newpage
\subsubsection{Fractional Non-Linear with Dinkelbach}

\paragraph{Mathematical Formulation}

This formulation uses a fractional objective to model efficiency as benefit per unit area allocated:

\begin{equation}
\max \frac{\sum_{f \in \mathcal{F}} \sum_{c \in \mathcal{C}} B_C \cdot A_{f,c}}{\sum_{f \in \mathcal{F}} \sum_{c \in \mathcal{C}} A_{f,c} + \epsilon}
\label{eq:fractional_obj}
\end{equation}

where $\epsilon > 0$ is a small constant (typically $10^{-8}$) to avoid division by zero.


\paragraph{Dinkelbach's Algorithm}\mbox{}\\

Dinkelbach's algorithm converts the fractional program into a sequence of parametric linear programs:

\begin{algorithm}[H]
\caption{Dinkelbach's Algorithm for Fractional Programming}
\begin{algorithmic}[1]
\State Initialize $\lambda^{(0)} = 0$, $k = 0$
\Repeat
    \State Solve parametric problem:
    \begin{equation*}
    z(\lambda^{(k)}) = \max_{A,Y} \left\{\sum_{f,c} B_c \cdot A_{f,c} - \lambda^{(k)} \sum_{f,c} A_{f,c}\right\}
    \end{equation*}
    \State Let $(A^{(k)}, Y^{(k)})$ be the optimal solution
    \State Update parameter:
    \begin{equation*}
    \lambda^{(k+1)} = \frac{\sum_{f,c} B_c \cdot A^{(k)}_{f,c}}{\sum_{f,c} A^{(k)}_{f,c} + \epsilon}
    \end{equation*}
    \State $k \leftarrow k + 1$
\Until{$|z(\lambda^{(k)})| < \tau$ or $|\lambda^{(k)} - \lambda^{(k-1)}| < \tau$}
\State \Return $(A^{(k)}, Y^{(k)})$
\end{algorithmic}
\end{algorithm}


\paragraph{Problem Classification}\mbox{}\\
\begin{itemize}
    \item \textbf{Original Type:} Fractional Mixed-Integer Program
    \item \textbf{Transformed Type:} Sequence of MILPs
    \item \textbf{Variables per iteration:} $2|\mathcal{F}||\mathcal{C}|$
    \item \textbf{Iterations:} Typically 5-15 until convergence
\end{itemize}

% \paragraph{Solution Methods}

% Each iteration solves a parametric MILP:
% \begin{itemize}
%     \item PuLP with CBC for each subproblem
%     \item Warm-starting from previous solution
%     \item Convergence tolerance $\tau = 10^{-6}$
% \end{itemize}

\paragraph{Computational Complexity}\mbox{}\\

\begin{itemize}
    \item \textbf{Variables:} Same as linear formulation ($2n$) per iteration
    \item \textbf{Time per iteration:} Similar to linear MILP
    \item \textbf{Total iterations:} $T \in [5, 15]$ typically with superlinear convergence \cite{dinkelbach1967nonlinear}
    \item \textbf{Total time:} $O(T \cdot t_{\text{MILP}})$ where $t_{\text{MILP}}$ is single MILP solve time
    \item \textbf{Convergence:} Guaranteed for fractional programs with positive denominators, with quadratic convergence near optimum \cite{schaible1976fractional,crouzeix1985algorithmic}.
\end{itemize}






\newpage


\subsubsection{Linear-Quadratic with Synergy}

\paragraph{Mathematical Formulation}\mbox{}\\

This formulation combines linear returns with quadratic synergy effects:

\begin{equation}
\max \underbrace{\sum_{f,c} B_c \cdot A_{f,c}}_{\text{Linear term}} + \underbrace{w_s \sum_f \sum_{\substack{c_1,c_2 \in \mathcal{G}_g\\ c_1 < c_2}} s_{c_1,c_2} \cdot Y_{f,c_1} \cdot Y_{f,c_2}}_{\text{Quadratic synergy term}}
\label{eq:lq_obj}
\end{equation}

where:
\begin{itemize}
    \item $s_{c_1,c_2}$: Synergy bonus for planting crops $c_1$ and $c_2$ together
    \item $w_s$: Weight for synergy bonus (typically $w_s = 0.1$)
    \item Synergy exists only for crop pairs in the same food group
\end{itemize}

This obviously means that, for this particular case:
\begin{equation}
    \sum_k w_k \in B_c = 1-w_s
\end{equation}

to maintain the convexity constraint valid in accordance to the definition we gave earlier for the benefit.


\textbf{Synergy Matrix Construction:}
\begin{equation}
s_{c_1,c_2} = \begin{cases}
\beta > 0 & \text{if } c_1, c_2 \in \mathcal{G}_g \text{ for some } g \text{ and } c_1 \neq c_2\\
0 & \text{otherwise}
\end{cases}
\end{equation}


\paragraph{McCormick Linearization}\mbox{}\\

The quadratic term $Y_{f,c_1} \cdot Y_{f,c_2}$ must be linearized for MILP solvers.

\textbf{McCormick Relaxation:}
For each synergy pair $(c_1, c_2)$ and farm $f$, introduce auxiliary variable:
\begin{equation}
Z_{f,c_1c_2} \in \{0,1\}
\end{equation}

to represent the product $Y_{f,c_1} \cdot Y_{f,c_2}$.

\textbf{Linearization Constraints:}
\begin{align}
Z_{f,c_1c_2} &\leq Y_{f,c_1} \label{eq:mcc1}\\
Z_{f,c_1c_2} &\leq Y_{f,c_2} \label{eq:mcc2}\\
Z_{f,c_1c_2} &\geq Y_{f,c_1} + Y_{f,c_2} - 1 \label{eq:mcc3}
\end{align}

\textbf{Correctness:}
\begin{itemize}
    \item If $Y_{f,c_1} = 0$ or $Y_{f,c_2} = 0$: (\ref{eq:mcc1})-(\ref{eq:mcc2}) force $Z_{f,c_1c_2} = 0$
    \item If $Y_{f,c_1} = Y_{f,c_2} = 1$: (\ref{eq:mcc3}) forces $Z_{f,c_1c_2} \geq 1$, combined with binary constraint gives $Z_{f,c_1c_2} = 1$
\end{itemize}

This linearization is \textbf{exact} for binary variables (not an approximation).

\textbf{Linearized Objective:}
\begin{equation}
\max \sum_{f,c} B_c \cdot A_{f,c} + w_s \sum_f \sum_{(c_1,c_2) \in \mathcal{S}} s_{c_1,c_2} \cdot Z_{f,c_1c_2}
\end{equation}

where $\mathcal{S}$ is the set of synergy pairs.

\paragraph{Problem Classification}\mbox{}\\

\textbf{For PuLP (Linearized):}
\begin{itemize}
    \item \textbf{Type:} Mixed-Integer Linear Program (after linearization)
    \item \textbf{Base Variables:} $2|\mathcal{F}||\mathcal{C}|$
    \item \textbf{Z Variables:} $|\mathcal{F}| \cdot |\mathcal{S}|$ where $|\mathcal{S}| = \sum_g \binom{|\mathcal{G}_g|}{2}$
    \item \textbf{Total Variables:} $2n + |\mathcal{F}| \cdot |\mathcal{S}|$
    \item \textbf{Linearization Constraints:} $3|\mathcal{F}| \cdot |\mathcal{S}|$
\end{itemize}

\textbf{For Pyomo/CQM (Native Quadratic):}
\begin{itemize}
    \item \textbf{Type:} Mixed-Integer Quadratic Program (MIQP)
    \item \textbf{Variables:} $2|\mathcal{F}||\mathcal{C}|$ (no Z variables needed)
    \item \textbf{Quadratic Terms:} $O(|\mathcal{F}| \cdot |\mathcal{S}|)$ in objective
\end{itemize}

% \paragraph{Variable Count Analysis}

% For typical problem with 27 crops in 5 food groups:
% \begin{itemize}
%     \item Grains (3 crops): $\binom{3}{2} = 3$ pairs
%     \item Legumes (2 crops): $\binom{2}{2} = 1$ pair
%     \item Vegetables (4 crops): $\binom{4}{2} = 6$ pairs
%     \item Fruits (1 crop): $\binom{1}{2} = 0$ pairs
% \end{itemize}

% Total synergy pairs: $|\mathcal{S}| = 10$

% For $|\mathcal{F}| = 5$ farms, $|\mathcal{C}| = 30$ crops ($n = 150$):
% \begin{itemize}
%     \item Base variables: $2 \times 150 = 300$
%     \item Z variables (PuLP): $5 \times 10 = 50$
%     \item \textbf{PuLP total:} 350 variables
%     \item \textbf{CQM/Pyomo total:} 300 variables
%     \item \textbf{Linearization constraints:} $3 \times 50 = 150$
% \end{itemize}

% \paragraph{Solution Methods}

% \subparagraph{PuLP Implementation:}
% \begin{itemize}
%     \item Linearizes using McCormick relaxation
%     \item Solves with CBC as standard MILP
%     \item Exact solution (no approximation error)
% \end{itemize}

% \subparagraph{Pyomo Implementation:}

% \begin{itemize}
%     \item Uses native MIQP formulation
%     \item Requires MIQP solver: Gurobi, CPLEX, CBC, or GLPK
%     \item No linearization needed
%     \item Solves quadratic objective directly
% \end{itemize}

% \subparagraph{DWave CQM Implementation:}
% \begin{itemize}
%     \item Native quadratic objective support
%     \item Quantum-classical hybrid solving
%     \item Particularly well-suited for quadratic problems
% \end{itemize}

\paragraph{Computational Complexity}\mbox{}\\
\begin{itemize}
    \item \textbf{Variables:}
\begin{itemize}
    \item Linearized: $2n + |\mathcal{F}| \cdot |\mathcal{S}| \approx 2.3n$ (for typical $|\mathcal{S}|$)
    \item Original: $2n$
\end{itemize}

\item \textbf{Constraints:}
\begin{itemize}
    \item Linearized: $3|\mathcal{F}| \cdot |\mathcal{S}|$ additional linearization constraints from McCormick relaxation \cite{mccormick1976computability}
    \item Original: Same as linear formulation \ref{eq:linear_obj}
    
\end{itemize}


\item \textbf{Solve Time:} Empirically comparable to linear, slightly slower than linear due to:
\begin{itemize}
    \item Linearized: Additional Z variables and constraints. McCormick relaxation is exact for binary variables \cite{mccormick1976computability,gupte2013solving}.
    \item Original: Quadratic terms require MIQP solver (more complex than LP relaxation). MIQP is NP-hard but often solvable in practice with specialized algorithms \cite{burer2012non,billionnet2007improving}.
\end{itemize}
\end{itemize}








\paragraph{Plots}\mbox{}\\

\begin{figure}[!h]
    \centering
    \includegraphics[width=\linewidth]{Plots/lq_speedup_comparison.png}
    %\caption{Caption}
    \label{fig:SYN_results}
\end{figure}


\newpage
\subsubsection{BQUBO Formulation }

We introduce a fundamentally different formulation based on Quadratic Unconstrained Binary Optimization (QUBO) principles. Instead of continuous area variables, it uses only binary variables, simplifying the problem space at the cost of modeling flexibility.

\paragraph{Mathematical Formulation}\mbox{}\\

The core idea is to discretize the decision space. Each farm-crop combination is represented by a single binary variable, indicating whether a fixed-size plot (e.g., 1 hectare) is planted or not.

Decision variables:
\begin{itemize}
    \item $Y_{f,c} \in \{0,1\}$: Binary variable indicating if a 1-hectare plot of crop $c$ is planted on farm $f$.
\end{itemize}

The objective function remains linear but is now defined over binary variables:
\begin{equation}
\max \sum_{f \in \mathcal{F}} \sum_{c \in \mathcal{C}} B_C \cdot Y_{f,c}
\label{eq:bqubo_obj}
\end{equation}

The constraints are adapted to this binary model:
\begin{align}
\sum_{c \in \mathcal{C}} Y_{f,c} &\leq \lfloor L_f \rfloor \quad \forall f \in \mathcal{F} \label{eq:land_binary}\\
\sum_{c \in \mathcal{G}_g} Y_{f,c} &\geq N_{\min,g} \quad \forall f,g \label{eq:diversity_binary}
\end{align}
Constraint (\ref{eq:land_binary}) limits the number of 1-hectare plots on a farm, which is a discretized version of the original land constraint.

\paragraph{Problem Classification}\mbox{}\\
\begin{itemize}
    \item \textbf{Type:} 0-1 Integer Linear Program (ILP).
    \item \textbf{Variables:} $|\mathcal{F}||\mathcal{C}|$ (only Y variables). This is a $50\%$ reduction compared to the linear MILP formulation.
    \item \textbf{Constraints:} $O(|\mathcal{F}|(1 + 2|\mathcal{G}|))$.
    \item \textbf{Linearity:} Fully linear objective and constraints.
\end{itemize}

% \paragraph{Solution Methods}

% \subparagraph{PuLP Implementation:}
% Solves the 0-1 ILP directly using the CBC solver. This is highly efficient due to the absence of continuous variables.

% \subparagraph{DWave BQUBO Implementation:}
% This is the key part of this solver. The Constrained Quadratic Model (CQM) is converted into a Binary Quadratic Model (BQM) using the \texttt{cqm\_to\_bqm} utility.
% \begin{itemize}
%     \item \textbf{Conversion:} This process transforms the constrained problem into an unconstrained one by embedding the constraints (\ref{eq:land_binary}) and (\ref{eq:diversity_binary}) into the objective function as quadratic penalty terms. The result is a BQM suitable for quantum annealers.
%     \item \textbf{Solver:} The resulting BQM is solved using the \texttt{LeapHybridBQMSampler}, which is designed for these types of problems and can leverage QPU resources more effectively than the general CQM sampler.
% \end{itemize}

\paragraph{Computational Complexity}\mbox{}\\

\textbf{Variable Count:} For $n = |\mathcal{F}||\mathcal{C}|$:
\begin{itemize}
    \item Total variables: $n$. This is the leanest formulation.
\end{itemize}

\textbf{Space Complexity:} $O(n)$ for variables. The BQM conversion can introduce many new quadratic terms, potentially increasing memory for the BQM object itself.

\textbf{Time Complexity:} Pure 0-1 ILP is NP-hard \cite{garey1979computers,karp1972reducibility}, but often highly efficient in practice. QUBO formulations have theoretical worst-case exponential complexity but can be solved efficiently on quantum annealers for certain problem structures \cite{lucas2014ising,glover2018tutorial}.

% \textbf{Solve Time:}
% \begin{itemize}
%     \item \textbf{PuLP:} Extremely fast, often faster than the standard linear MILP due to having half the variables and no continuous relaxation complexities.
%     \item \textbf{DWave:} Includes a non-trivial classical pre-processing time for the \texttt{cqm\_to\_bqm} conversion, followed by the hybrid solver time. The goal is better scaling for very large, complex problems where the BQM formulation can be efficiently processed by the QPU.
% \end{itemize}





\paragraph{Plots}\mbox{}\\

\begin{figure}[!h]
    \centering
    \includegraphics[width=\linewidth]{Plots/bqubo_speedup_comparison.png}
    %\caption{Caption}
    \label{fig:BQUBO_results}
\end{figure}


\newpage
\subsection{Comparative Analysis of Objective Formulations}

Table~\ref{tab:formulation_comparison} provides a comprehensive comparison of all five objective formulations, highlighting their key characteristics, advantages, and limitations. This unified view enables selection of the most appropriate formulation based on problem requirements and computational resources.

\begin{table}[!ht]
\centering
\small
\caption{Comparative Analysis of Objective Formulations}
\label{tab:formulation_comparison}
\begin{tabular}{@{}p{2.5cm}p{2.3cm}p{2.3cm}p{2.3cm}p{2.3cm}p{2.3cm}@{}}
\toprule
\textbf{Criterion} & \textbf{Linear (LQ)} & \textbf{Piecewise NLN} & \textbf{Fractional Dinkelbach} & \textbf{Quadratic Synergy} & \textbf{BQUBO} \\
\midrule
\multicolumn{6}{@{}l}{\textit{\textbf{Problem Classification}}} \\
Type & MILP & MILP + SOS2 & Fractional MIP & MILP/MIQP & 0-1 ILP \\
Variables & $2n$ & $\approx 14n$ & $2n$ per iter. & $2.3n$ (linear) / $2n$ (quad) & $n$ \\
Complexity & NP-hard & NP-hard & Sequence of NP-hard & NP-hard & NP-hard \\
\midrule
\multicolumn{6}{@{}l}{\textit{\textbf{Modeling Capabilities}}} \\
Returns Type & Linear & Diminishing & Efficiency ratio & Linear + interactions & Linear (discrete) \\
Non-linearity & None & Yes (approx.) & Fractional & Quadratic & None \\
Synergy Effects & No & No & No & Yes & No \\
Approximation & Exact & $O(K^{-2})$ error & Exact & Exact & Discretization error \\
\midrule
\multicolumn{6}{@{}l}{\textit{\textbf{Computational Performance}}} \\
Solve Time & Fastest & $2-4\times$ slower & $5-15\times$ slower & Comparable & Fastest (classical) \\
Memory & $O(n)$ & $O(Kn)$ & $O(n)$ & $O(n)$ & $O(n)$ \\
Scalability & Excellent & Good & Good & Excellent & Excellent \\
\midrule
\multicolumn{6}{@{}l}{\textit{\textbf{Key Advantages}}} \\
Primary & Simplest; exact; fast & Realistic diminishing returns & Models efficiency directly & Captures synergies & Fewest variables; QPU-ready \\
Secondary & Mature solvers; well-understood & Controllable accuracy; MINLP option & Fast convergence; proven algorithm & Exact for binary products & Conceptually simple \\
Tertiary & No approx. errors & Economically justified & No extra variables & $75\%$ fewer vars than NLN & Fastest classical solve \\
\midrule
\multicolumn{6}{@{}l}{\textit{\textbf{Key Limitations}}} \\
Primary & Unrealistic linear returns & $7\times$ more variables & Multiple solves required & Requires synergy matrix & Fixed uniform plot sizes \\
Secondary & No diminishing returns & Approx. error present & Limited to fractional form & More vars than linear & Discretization error \\
Tertiary & No interactions & Slower solve times & Positive denominator needed & Linearization overhead & Inflexible modeling \\
\midrule
\multicolumn{6}{@{}l}{\textit{\textbf{Best Use Cases}}} \\
Application & Baseline; fast prototyping & Agricultural planning with diminishing marginal productivity & Resource efficiency optimization & Companion planting; crop rotation & Quantum annealing; binary decisions \\
\bottomrule
\end{tabular}
\end{table}

\paragraph{Selection Guidelines}\mbox{}\\

\begin{itemize}
    \item \textbf{For speed and simplicity:} Use Linear (LQ) or BQUBO formulations
    \item \textbf{For realistic agricultural modeling:} Use Piecewise NLN with appropriate $K$
    \item \textbf{For efficiency-based objectives:} Use Fractional Dinkelbach formulation
    \item \textbf{For capturing crop interactions:} Use Quadratic Synergy formulation
    \item \textbf{For quantum computing:} Use BQUBO with BQM conversion
\end{itemize}


}

\newpage
\subsection{Benchmarking Strategies}

Our benchmark consists of two main components: \textbf{scenarios} and \textbf{solvers}.

We will test three solvers, namely the state of the art open source python library for modeling Linear Programs, PuLP (which can run solvers like Gurobi \cite{gurobi2023} and CPLEX \cite{cplex2023}) the D-Wave Leap Hybrid Solver, to maintain a direct comparison in case we detect any possibility of \textit{local advantage} \cite{ronnow2014defining}{\color{blue}, and the state of the art open source python library for modeling Non Linear Programs - when needed - Pyomo \cite{bynum2021pyomo}, which also uses state of the art solvers like Ipopt.

\subsubsection{Results from testing - to be moved}
From preliminary testing of the classical and hybrid solvers across multiple scenarios and strategies \ref{fig:NLN_result} \ref{fig:SYN_results} \ref{fig:BQUBO_results}, we can see what the relative advantages are: these results show that while there is a spectrum of performances in terms of speed across the models, with the classical PuLP formulations being best.

The nearly constant QPU usage time also suggests that for growing problem sizes or complexity, there is a speed advantage to be found for the hybrid solvers due to the worse scaling of the classical solvers.

This is in accordance to what was found in \cite{opticalrouting} regarding the difference beween QPU usage and classical resources in the Hybrid Solvers.

It is also important to notice that, as presented in \cite{Gurobi2025BuzzwordJungle}, while it is good to compare apples with apples (QUBO on classical vs quantum), the best practice is to compare best with best, so MILP on classical and QUBO on quantum.

This not only because classical solvers are inherently bad at solving QUBOs, since all structure is lost during the conversion, but because we try to adhere to the practices highlighted in \cite{Koch2025QuantumOptimizationBenchmarkLibrary}.

Our results match the findings of \cite{krellner2025solvingrealworldmodularlogistic} regarding the fact that classical MILP is the fastest and classical QUBO is the slowest.

\subsubsection{Scenario Adaptation}

In all of the above formulations, the total area was dictated by the number of farms, as they were sampled from a distribution \cite{LOWDER201616}.

Due to how the model is currently formulated, the total area is a parameter known \textit{a priori}; this total area is going to get subdivided either into a number of farms with uneven area, or into a number of uniformly sized plots.

Additional information regarding the difference between the two choices can be found in Appendix 1

\subsubsection{Overview}
Having analyzed all the previous formulations and their performance on classical and quantum hardware, the benchmark will be comprehensive of two formulations:

\begin{enumerate}
    \item \textbf{Binary Formulation} (Even Grid): Used when land is divided into equal-sized plots
    \item \textbf{Continuous Formulation} (Uneven Distribution): Used when farms have varying sizes
\end{enumerate}


The script solves the optimization problem using three different methods:
\begin{itemize}
    \item PuLP with Gurobi solver (classical optimization)
    \item D-Wave Hybrid CQM Sampler (quantum-classical hybrid)
    \item D-Wave Hybrid BQM Sampler (quantum-enabled via CQM→BQM conversion)
\end{itemize}



\subsubsection{Objective Functions}

\paragraph{Continuous Formulation Objective}

The objective function maximizes the weighted sum of agricultural value metrics, normalized by total available land:

$$\max \quad Z = \frac{1}{\sum_{f \in F} L_f} \sum_{f \in F} \sum_{c \in C} B_c \cdot A_{f,c}$$

where the composite value $v_c$ for crop $c$ is defined as:

$$B_c = w_{nv} \cdot v_{nv,c} + w_{nd} \cdot v_{nd,c} - w_{ei} \cdot v_{ei,c} + w_{af} \cdot v_{af,c} + w_{su} \cdot v_{su,c}$$

\textbf{Note:} This is equivalent to \ref{eq:linear_obj} after appropriate renormalization.

\paragraph{Binary Formulation Objective}

For the binary formulation, the objective accounts for the fixed area $a_p$ of each plot:

$$\max \quad Z = \frac{1}{\sum_{p \in F} a_p} \sum_{p \in F} \sum_{c \in C} a_p \cdot B_c \cdot Y_{p,c}$$

where $B_c$ is defined identically as in the continuous formulation.

\textbf{Interpretation:} Each selected assignment contributes the plot's area multiplied by the crop's value density.

\subsubsection{Constraints}

\paragraph{Continuous Formulation Constraints}

\subparagraph{Land Availability Constraints}

Each farm cannot allocate more land than available:

$$\sum_{c \in \mathcal{C}} A_{f,c} \leq L_f \quad \forall f \in \mathcal{F}$$

\textbf{Label:} \texttt{Land\_Availability\_\{farm\}}

\subparagraph{Minimum Planting Area Constraints}

If a crop is selected on a farm, it must occupy at least the minimum required area:

$$A_{f,c} \geq A_{min,c} \cdot Y_{f,c} \quad \forall f \in \mathcal{F}, c \in C$$


\textbf{Logical Interpretation:}
\begin{itemize}
    \item If $Y_{f,c} = 1$: $A_{f,c} \geq A_{min,c}$ (enforces minimum area)
    \item If $Y_{f,c} = 0$: $A_{f,c} \geq 0$ (no planting, so area can be zero)
\end{itemize}

\textbf{Label:} \texttt{Min\_Area\_If\_Selected\_\{farm\}\_\{crop\}}

\subparagraph{Maximum Planting Area Constraints}

If a crop is not selected, its allocated area must be zero:

$$A_{f,c} \leq L_f \cdot Y_{f,c} \quad \forall f \in \mathcal{F}, c \in C$$


\textbf{Logical Interpretation:}
\begin{itemize}
    \item If $Y_{f,c} = 1$: $A_{f,c} \leq L_f$ (area can be up to farm size)
    \item If $Y_{f,c} = 0$: $A_{f,c} \leq 0$ (forces area to zero)
\end{itemize}

\textbf{Label:} \texttt{Max\_Area\_If\_Selected\_\{farm\}\_\{crop\}}

\subparagraph{Food Group Minimum Constraints}

At least a minimum number of different crops from specified food groups must be cultivated:

$$\sum_{f \in \mathcal{F}}\sum_{c \in \mathcal{G}} Y_{f,c} \geq N_{min,g} \quad \forall g \in \mathcal{G} \text{ where } N_{min,g} \text{ is defined}$$

\textbf{Label:} \texttt{Food\_Group\_Min\_\{group\}}

\subparagraph{Food Group Maximum Constraints}

A maximum number of different crops from specified food groups must not be exceeded:

$$\sum_{f \in \mathcal{F}}\sum_{c \in \mathcal{G}} Y_{f,c} \leq N_{max,g} \quad \forall g \in \mathcal{G} \text{ where } N_{max,g} \text{ is defined}$$

\textbf{Label:} \texttt{Food\_Group\_Max\_\{group\}}

\paragraph{Binary Formulation Constraints}

\subparagraph{Plot Assignment Constraints}

Each plot can be assigned to at most one crop (or remain idle):

$$\sum_{c \in \mathcal{C}} Y_{p,c} \leq 1 \quad \forall p \in \mathcal{F}$$



\textbf{Interpretation:}
\begin{itemize}
    \item $\sum_{c \in C} Y_{p,c} = 0$: Plot remains idle
    \item $\sum_{c \in C} Y_{p,c} = 1$: Plot is assigned to exactly one crop
\end{itemize}

\textbf{Label:} \texttt{Max\_Assignment\_\{plot\}}

\subparagraph{Minimum Plots Per Crop Constraints}

For crops with minimum planting area requirements, the constraint is converted to a minimum number of plots:

$$\sum_{p \in \mathcal{F}} Y_{p,c} \geq \left\lceil \frac{A_{min,c}}{a_p} \right\rceil \quad \forall c \in \mathcal{F} \text{ where } A_{min,c} > 0$$

where $a_p$ is the area of each plot (assumed equal in even grid).



\textbf{Interpretation:} If a crop $c$ requires minimum area $A_{min,c}$, it must be planted on at least $\lceil A_{min,c} / a_p \rceil$ plots.

\textbf{Label:} \texttt{Min\_Plots\_\{crop\}}

\subparagraph{Maximum Plots Per Crop Constraints}

For crops with maximum planting area limits, the constraint is converted to a maximum number of plots:

$$\sum_{p \in \mathcal{F}} Y_{p,c} \leq \left\lfloor \frac{A_{max,c}}{a_p} \right\rfloor \quad \forall c \in \mathcal{C} \text{ where } A_{max,c} \text{ is defined}$$



\textbf{Interpretation:} If a crop $c$ has maximum area $A_{max,c}$, it can be planted on at most $\lfloor A_{max,c} / a_p \rfloor$ plots.


\textbf{Label:} \texttt{Max\_Plots\_\{crop\}}




\subparagraph{Food Group Constraints}

The same food group minimum and maximum constraints apply as in the continuous formulation:

$$\sum_{p \in \mathcal{F}}\sum_{c \in \mathcal{G}} Y_{p,c} \geq N_{min,g} \quad \forall g \in \mathcal{G}$$
$$\sum_{p \in \mathcal{F}}\sum_{c \in \mathcal{G}} Y_{p,c} \leq N_{max,g} \quad \forall g \in \mathcal{G}$$

\textbf{Labels:} \texttt{Food\_Group\_Min\_\{group\}}, \texttt{Food\_Group\_Max\_\{group\}}


\subsubsection{Plots}
\clearpage


\begin{figure}
    \centering
    \includegraphics[width=\linewidth]{Plots/Benchmark/performance_comparison.pdf}
    %\caption{Caption}
    %\label{fig:placeholder}
\end{figure}


\begin{figure}
    \centering
    \includegraphics[width=\linewidth]{Plots/Benchmark/solution_quality_comparison.pdf}
    %\caption{Caption}
    %\label{fig:placeholder}
\end{figure}

\clearpage    
\begin{figure}
    \centering
    \includegraphics[width=\linewidth]{Plots/Benchmark/solution_composition_histograms.pdf}
    %\caption{Caption}
    %\label{fig:placeholder}
\end{figure}


\begin{figure}
    \centering
    \includegraphics[width=\linewidth]{Plots/Benchmark/solution_composition_pies.pdf}
    %\caption{Caption}
    %\label{fig:placeholder}
\end{figure}

}






\clearpage
\subsection{Resource Estimation}


\textbf{Overview:}  
 Problems scale from 18 to 500+ variables, with hybrid solvers best suited for constrained formulations.

\medskip
\textbf{Problem Sizes Tested:}

\begin{center}
\begin{tabular}{@{}lllll@{}}
\toprule
Level & Vars & Farms & Foods & Density \\
\midrule
Simple        & 10  & 2 & 5  & 0.333 \\
Intermediate  & 18  & 3 & 6  & 0.333 \\
Full          & 50  & 5 & 10 & 0.200 \\
\bottomrule
\end{tabular}
\end{center}

\medskip
\textbf{Scaling:}
\begin{itemize}
    \item Linear fit: $T \approx 0.0059V - 0.04$
    \item Exponential: $T \approx 0.006e^{0.076V}$
    \item Scope: 50 vars (current) $\to$ 100--200 (extended) $\to$ 500+ (scalability)
\end{itemize}

\medskip
\textbf{Resource Estimates:}

\begin{center}
\begin{tabular}{@{}llll@{}}
\toprule
Size & QPU Time & Embedding & Hybrid Time \\
\midrule
Small ($\leq$20)  & 2--100 ms  & 1--5 s   & 10--60 s \\
Medium (20--100)  & 10--200 ms & 5--30 s  & 30--180 s \\
Large (100+)      & 20--500 ms & 30--120 s & 60--600+ s \\
\bottomrule
\end{tabular}
\end{center}

\medskip
\textbf{Upper Bounds (95th \%ile):}

\begin{center}
\begin{tabular}{@{}llll@{}}
\toprule
Size & QPU (s) & Hybrid (s) & Wall Time (s) \\
\midrule
$\leq$50   & $\leq$0.5 & $\leq$300  & $\leq$360 \\
50--200    & $\leq$2   & $\leq$900  & $\leq$1080 \\
200--500   & $\leq$5   & $\leq$3600 & $\leq$4320 \\
\bottomrule
\end{tabular}
\end{center}

\medskip
\textbf{Architecture:}
\begin{itemize}
    \item \emph{Hybrid (CQM)}: continuous + binary vars, LeapHybridCQMSampler
    \item \emph{Pure QPU}: QUBO, EmbeddingComposite + D-WaveSampler
    \item \emph{Decomposition}: master (hybrid) + farm subproblems (QPU)
\end{itemize}

\medskip
\textbf{Risks \& Mitigations:}
\begin{itemize}
    \item Embedding limits $\to$ hybrid/decomposition
    \item Local minima $\to$ multiple runs
    \item Runtime inflation $\to$ simplify/limit time
\end{itemize}

\medskip
\textbf{Recommendations:}
\begin{itemize}
    \item Start: 18-variable tests (5--15 min/exp.)
    \item 1-hour allocation: 6--12 small, 2--4 medium, 1--2 large
    \item Safety margins: $\times$2 time, $\times$1.5 samples, +50\% buffer
\end{itemize}

\medskip
\textbf{Conclusion:}  
Hybrid solvers provide robustness (5--40 min/exp). QPU scaling predictable; 1-hour budget supports baseline validation, comparisons, and scalability demos.



\subsection{Steps to Achieve a Proof of Concept}


\subsubsection{Key Data Inputs and Requirements}



\begin{enumerate}


\item{Population and Nutritional Needs}
\begin{itemize}
    \item \textbf{Demographics:} Age, gender, height, weight, pregnant and lactating women (PLW)
    \item \textbf{Population projections:} Projected population growth and demographic breakdown
    \item \textbf{Nutritional requirements:} Calories, protein, dietary fats, essential micronutrients for each population group:
    \begin{itemize}
        \item Infants
        \item Boys and girls
        \item Adolescent boys and girls
        \item Men and women of reproducing age
        \item Pregnant and lactating women
        \item Elderly
    \end{itemize}
\end{itemize}

\item{Food System Requirements}
\begin{itemize}
    \item \textbf{Culturally acceptable foods:} Foods produced and consumed within the local food system context (Diet Quality Questionnaire Database)
    \item \textbf{Food basket design:} Identification of food combinations that fulfill all nutritional needs
    \item \textbf{GAIN nLCA score:} To optimize for nutrition and environment (GHGs, water, land use)
    \item \textbf{Cost analysis:} Cheapest sources of good nutrition per nutritional unit using Nutritional Value Score
\end{itemize}


\item{Production Feasibility Analysis}
\begin{itemize}
    \item \textbf{Crop production requirements:}


\begin{itemize}
    \item Total quantity of each crop required
    \item Yield per hectare of selected/local/optimal varieties
    \item Total land available for crop production
    \item Preferred production methods and compatibility with local climate, soil, and water availability
    \item Compatibility of different crops within the same cultivated area
\end{itemize}

\item \textbf{Animal-source food production:}
\begin{itemize}
    \item Total quantity required for meat, dairy, eggs, poultry, seafood
    \item Total production potential and yield from fisheries
    \item Compatibility of production methods with local systems
    \item Integration potential with crop systems
\end{itemize}
\end{itemize}

\end{enumerate}





\newpage


% References for Computational Complexity Claims
% Add these to your main bibliography file

% Linear MILP Complexity
% @book{wolsey1998integer,
%   title={Integer Programming},
%   author={Wolsey, Laurence A},
%   year={1998},
%   publisher={John Wiley \& Sons},
%   note={Classic textbook on integer programming theory and algorithms. Chapter 2 covers MILP complexity.}
% }

% @book{nemhauser1988integer,
%   title={Integer and Combinatorial Optimization},
%   author={Nemhauser, George L and Wolsey, Laurence A},
%   year={1988},
%   publisher={Wiley-Interscience},
%   note={Authoritative reference on integer programming. Section 1.3 discusses NP-hardness of MILPs.}
% }

% @inproceedings{achterberg2007constraint,
%   title={Constraint integer programming},
%   author={Achterberg, Tobias and Berthold, Timo and Koch, Thorsten and Wolter, Kati},
%   booktitle={International Conference on Integration of Artificial Intelligence and Operations Research Techniques in Constraint Programming},
%   pages={6--20},
%   year={2007},
%   organization={Springer},
%   note={Modern MILP solver techniques including cutting planes and preprocessing.}
% }

% @article{bixby2002solving,
%   title={Solving real-world linear programs: a decade and more of progress},
%   author={Bixby, Robert E},
%   journal={Operations Research},
%   volume={50},
%   number={1},
%   pages={3--15},
%   year={2002},
%   publisher={INFORMS},
%   note={Discusses practical performance improvements in MILP solvers.}
% }

% @book{bertsimas1997introduction,
%   title={Introduction to Linear Optimization},
%   author={Bertsimas, Dimitris and Tsitsiklis, John N},
%   year={1997},
%   publisher={Athena Scientific},
%   note={Chapter 11 covers computational complexity and memory requirements.}
% }

% Piecewise Approximation Complexity
% @article{vielma2010mixed,
%   title={Mixed integer linear programming formulation techniques},
%   author={Vielma, Juan Pablo},
%   journal={SIAM Review},
%   volume={57},
%   number={1},
%   pages={3--57},
%   year={2015},
%   publisher={SIAM},
%   note={Comprehensive survey of piecewise linear approximations and SOS2 constraints.}
% }

% @article{beale1970special,
%   title={Special facilities in a general mathematical programming system for non-convex problems using ordered sets of variables},
%   author={Beale, Evelyn ML and Tomlin, JA},
%   journal={OR},
%   volume={69},
%   number={5},
%   pages={447--454},
%   year={1970},
%   note={Original paper introducing SOS2 constraints for piecewise approximation.}
% }

% @article{croxton2003comparison,
%   title={A comparison of mixed-integer programming models for nonconvex piecewise linear cost minimization problems},
%   author={Croxton, Keely L and Gendron, Bernard and Magnanti, Thomas L},
%   journal={Management Science},
%   volume={49},
%   number={9},
%   pages={1268--1273},
%   year={2003},
%   publisher={INFORMS},
%   note={Analysis of approximation error bounds for piecewise linear functions.}
% }

% Dinkelbach's Algorithm
% @article{dinkelbach1967nonlinear,
%   title={On nonlinear fractional programming},
%   author={Dinkelbach, Werner},
%   journal={Management Science},
%   volume={13},
%   number={7},
%   pages={492--498},
%   year={1967},
%   publisher={INFORMS},
%   note={Original paper presenting Dinkelbach's algorithm for fractional programming.}
% }

% @article{schaible1976fractional,
%   title={Fractional programming},
%   author={Schaible, Siegfried},
%   journal={Zeitschrift f{\"u}r Operations Research},
%   volume={27},
%   number={1},
%   pages={39--54},
%   year={1983},
%   publisher={Springer},
%   note={Survey on fractional programming with convergence analysis.}
% }

% @article{crouzeix1985algorithmic,
%   title={An algorithmic approach to generalized fractional programming},
%   author={Crouzeix, Jean-Pierre and Ferland, Jacques A and Schaible, Siegfried},
%   journal={Journal of Global Optimization},
%   volume={2},
%   number={2},
%   pages={113--127},
%   year={1992},
%   publisher={Springer},
%   note={Convergence properties of parametric algorithms for fractional programs.}
% }

% McCormick Linearization
% @article{mccormick1976computability,
%   title={Computability of global solutions to factorable nonconvex programs: Part I—Convex underestimating problems},
%   author={McCormick, Garth P},
%   journal={Mathematical Programming},
%   volume={10},
%   number={1},
%   pages={147--175},
%   year={1976},
%   publisher={Springer},
%   note={Original McCormick relaxation paper for nonlinear term linearization.}
% }

% @article{gupte2013solving,
%   title={Solving mixed integer bilinear problems using MILP formulations},
%   author={Gupte, Akshay and Ahmed, Shabbir and Cheon, Myun-Seok and Dey, Santanu},
%   journal={SIAM Journal on Optimization},
%   volume={23},
%   number={2},
%   pages={721--744},
%   year={2013},
%   publisher={SIAM},
%   note={Analysis of McCormick relaxation tightness for binary products.}
% }

% MIQP Complexity
% @article{burer2012non,
%   title={Non-convex mixed-integer nonlinear programming: a survey},
%   author={Burer, Samuel and Letchford, Adam N},
%   journal={Surveys in Operations Research and Management Science},
%   volume={17},
%   number={2},
%   pages={97--106},
%   year={2012},
%   publisher={Elsevier},
%   note={Survey covering MIQP complexity and solution methods.}
% }

% @article{billionnet2007improving,
%   title={Improving the performance of standard solvers for quadratic 0-1 programs by a tight convex reformulation: The QCR method},
%   author={Billionnet, Alain and Elloumi, Sourour and Lambert, Amelie},
%   journal={Discrete Applied Mathematics},
%   volume={157},
%   number={6},
%   pages={1185--1197},
%   year={2009},
%   publisher={Elsevier},
%   note={Practical methods for solving quadratic binary programs.}
% }

% QUBO and 0-1 ILP Complexity
% @book{garey1979computers,
%   title={Computers and Intractability: A Guide to the Theory of NP-Completeness},
%   author={Garey, Michael R and Johnson, David S},
%   year={1979},
%   publisher={W. H. Freeman},
%   note={Classic reference on NP-completeness. Pages 245-248 cover integer programming.}
% }

% @article{karp1972reducibility,
%   title={Reducibility among combinatorial problems},
%   author={Karp, Richard M},
%   journal={Complexity of Computer Computations},
%   pages={85--103},
%   year={1972},
%   publisher={Springer},
%   note={Original paper establishing NP-completeness of integer programming.}
% }

% @article{lucas2014ising,
%   title={Ising formulations of many NP problems},
%   author={Lucas, Andrew},
%   journal={Frontiers in Physics},
%   volume={2},
%   pages={5},
%   year={2014},
%   publisher={Frontiers},
%   note={Survey of QUBO formulations for NP-hard problems suitable for quantum annealing.}
% }

% @article{glover2018tutorial,
%   title={A tutorial on formulating and using QUBO models},
%   author={Glover, Fred and Kochenberger, Gary and Du, Yu},
%   journal={arXiv preprint arXiv:1811.11538},
%   year={2018},
%   note={Comprehensive tutorial on QUBO problem formulation and complexity.}
% }

