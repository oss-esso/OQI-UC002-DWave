\documentclass[11pt,a4paper]{article}
\usepackage{amsmath,amssymb,amsthm}
\usepackage{geometry}
\usepackage{graphicx}
\usepackage{hyperref}
\usepackage{algorithm}
\usepackage{algpseudocode}
\usepackage{booktabs}
\usepackage{enumitem}

\geometry{margin=1in}

\title{\textbf{Technical Report: Comparison of Objective Functions and Solution Methods in Food Optimization Solvers}}
\author{Edo}
\date{\today}

\begin{document}

\maketitle

\begin{abstract}
This technical report provides a comprehensive comparison of five distinct solver implementations for the food optimization problem: \texttt{solver\_runner.py} (linear), \texttt{solver\_runner\_NLN.py} (non-linear with piecewise approximation), \texttt{solver\_runner\_NLD.py} (non-linear with dinkelbach), \texttt{solver\_runner\_LQ.py} (linear-quadratic), and \texttt{solver\_runner\_BQUBO.py} (binary). We analyze the mathematical formulations, algorithmic approaches, computational complexity, and practical implications of each method. Our analysis demonstrates that the choice of objective function and solution method significantly impacts problem size, solve time, and solution quality, with trade-offs between modeling accuracy and computational efficiency.
\end{abstract}

\section{Introduction}

The food optimization problem addresses the allocation of agricultural land across multiple farms to grow various crops, maximizing a weighted combination of nutritional value, sustainability, affordability, and other attributes while satisfying land availability and diversity constraints. This report analyzes five distinct approaches to formulating and solving this problem, each with different mathematical properties and computational characteristics.

\subsection{Problem Context}

Given:
\begin{itemize}
    \item $\mathcal{F}$: Set of farms with land availability $L_f$ for $f \in \mathcal{F}$
    \item $\mathcal{C}$: Set of crops with attributes (nutritional value, sustainability, etc.)
    \item $\mathcal{G}$: Set of food groups with diversity requirements
    \item $w_k$: Weights for different objectives $k \in \{$nutrition, sustainability, ...$\}$
\end{itemize}

Decision variables:
\begin{itemize}
    \item $A_{fc} \in [0, L_f]$: Continuous area allocated to crop $c$ on farm $f$
    \item $Y_{fc} \in \{0,1\}$: Binary selection indicator for crop $c$ on farm $f$
\end{itemize}

Common constraints across all formulations:
\begin{align}
\sum_{c \in \mathcal{C}} A_{fc} &\leq L_f \quad \forall f \in \mathcal{F} \label{eq:land}\\
A_{fc} &\geq A_{\min,c} \cdot Y_{fc} \quad \forall f,c \label{eq:minarea}\\
A_{fc} &\leq L_f \cdot Y_{fc} \quad \forall f,c \label{eq:maxarea}\\
\sum_{c \in \mathcal{G}_g} Y_{fc} &\geq N_{\min,g} \quad \forall f,g \label{eq:diversity}
\end{align}

where constraint (\ref{eq:land}) enforces land availability, (\ref{eq:minarea})-(\ref{eq:maxarea}) link continuous and binary variables, and (\ref{eq:diversity}) ensures crop diversity.

\section{Solver Implementations}

\subsection{Solver 1: Linear Objective (\texttt{solver\_runner.py})}

\subsubsection{Mathematical Formulation}

The simplest formulation uses a linear objective function:

\begin{equation}
\max \sum_{f \in \mathcal{F}} \sum_{c \in \mathcal{C}} \left(\sum_{k} w_k \cdot \text{attr}_{k,c}\right) \cdot A_{fc}
\label{eq:linear_obj}
\end{equation}

where $\text{attr}_{k,c}$ represents attribute $k$ of crop $c$.

\subsubsection{Problem Classification}
\begin{itemize}
    \item \textbf{Type:} Mixed-Integer Linear Program (MILP)
    \item \textbf{Variables:} $2|\mathcal{F}||\mathcal{C}|$ (A and Y variables)
    \item \textbf{Constraints:} $O(|\mathcal{F}|(1 + 2|\mathcal{C}| + 2|\mathcal{G}|))$
    \item \textbf{Linearity:} Fully linear in both objective and constraints
\end{itemize}

\subsubsection{Solution Methods}

\paragraph{PuLP Implementation:}
Uses the CBC (COIN-OR Branch and Cut) solver, a state-of-the-art open-source MILP solver employing:
\begin{itemize}
    \item Branch-and-bound with linear relaxations
    \item Cutting plane generation
    \item Primal heuristics for feasible solutions
\end{itemize}

\paragraph{DWave CQM Implementation:}
Converts the MILP to a Constrained Quadratic Model (CQM) for quantum-classical hybrid solving:
\begin{itemize}
    \item Uses LeapHybridCQMSampler
    \item Combines quantum annealing with classical optimization
    \item Penalty method for constraint handling
\end{itemize}

\subsubsection{Computational Complexity}

\textbf{Time Complexity:} $O(2^n \cdot p(n))$ in worst case, where $n = |\mathcal{F}||\mathcal{C}|$ and $p(n)$ is polynomial. In practice, modern MILP solvers exhibit near-linear scaling for well-structured problems.

\textbf{Space Complexity:} $O(n)$ for variable storage plus $O(n^2)$ for constraint matrix (sparse).

\subsubsection{Advantages and Limitations}

\textbf{Advantages:}
\begin{itemize}
    \item Simplest formulation with fewest variables
    \item Exact optimal solution guaranteed
    \item Fast solve times with mature MILP solvers
    \item No approximation errors
    \item Well-understood theory and algorithms
\end{itemize}

\textbf{Limitations:}
\begin{itemize}
    \item Linear returns assumption may be unrealistic
    \item Cannot model diminishing returns
    \item No interaction effects between crops
    \item May overestimate benefits at scale
\end{itemize}

\subsection{Solver 2: Non-Linear with Piecewise Approximation (\texttt{solver\_runner\_NLN.py})}

\subsubsection{Mathematical Formulation}

This formulation uses a concave power function to model diminishing returns:

\begin{equation}
\max \sum_{f \in \mathcal{F}} \sum_{c \in \mathcal{C}} \left(\sum_{k} w_k \cdot \text{attr}_{k,c}\right) \cdot f(A_{fc})
\label{eq:nonlinear_obj}
\end{equation}

where $f(A) = A^\alpha$ with $\alpha = 0.548$ (typically $0 < \alpha < 1$ for diminishing returns).

\subsubsection{Piecewise Linear Approximation}

Since the power function is non-convex for MILP solvers, we use piecewise linear approximation with Special Ordered Set of type 2 (SOS2) constraints:

\textbf{Breakpoint Definition:}
For each farm-food pair $(f,c)$, define $K+1$ breakpoints:
\begin{equation}
0 = b_0 < b_1 < \cdots < b_K = L_f
\end{equation}

Typically, $K = 10$ breakpoints with uniform spacing:
\begin{equation}
b_i = \frac{i \cdot L_f}{K} \quad \text{for } i = 0, \ldots, K
\end{equation}

\textbf{Function Values at Breakpoints:}
\begin{equation}
\phi_i = f(b_i) = b_i^{0.548}
\end{equation}

\textbf{Additional Variables:}
For each $(f,c)$ pair, introduce:
\begin{itemize}
    \item $\lambda_{fc,i} \in [0,1]$ for $i = 0,\ldots,K$: Convex combination weights
    \item $\tilde{f}_{fc} \in \mathbb{R}$: Approximated function value
\end{itemize}

\textbf{Piecewise Constraints:}
\begin{align}
A_{fc} &= \sum_{i=0}^{K} \lambda_{fc,i} \cdot b_i \label{eq:pw_area}\\
\tilde{f}_{fc} &= \sum_{i=0}^{K} \lambda_{fc,i} \cdot \phi_i \label{eq:pw_func}\\
\sum_{i=0}^{K} \lambda_{fc,i} &= 1 \label{eq:pw_convex}\\
\lambda_{fc,i} &\geq 0, \quad \text{at most 2 consecutive } \lambda_{fc,i} > 0 \label{eq:sos2}
\end{align}

The SOS2 constraint (\ref{eq:sos2}) ensures only adjacent breakpoints have positive weights.

\textbf{Modified Objective:}
\begin{equation}
\max \sum_{f \in \mathcal{F}} \sum_{c \in \mathcal{C}} \left(\sum_{k} w_k \cdot \text{attr}_{k,c}\right) \cdot \tilde{f}_{fc}
\label{eq:pw_obj}
\end{equation}

\subsubsection{Problem Classification}
\begin{itemize}
    \item \textbf{Type:} Mixed-Integer Linear Program with SOS2 constraints
    \item \textbf{Variables:} $2|\mathcal{F}||\mathcal{C}| + (K+1)|\mathcal{F}||\mathcal{C}| + |\mathcal{F}||\mathcal{C}|$
    \begin{itemize}
        \item Base: $2|\mathcal{F}||\mathcal{C}|$ (A and Y)
        \item Lambda: $(K+1)|\mathcal{F}||\mathcal{C}|$ (typically 11n)
        \item Approximation: $|\mathcal{F}||\mathcal{C}|$ ($\tilde{f}$ variables)
    \end{itemize}
    \item \textbf{Total Variables:} $(K+4)|\mathcal{F}||\mathcal{C}| \approx 14n$ for $K=10$
    \item \textbf{Additional Constraints:} $3|\mathcal{F}||\mathcal{C}|$ piecewise constraints
\end{itemize}

\subsubsection{Solution Methods}

\paragraph{PuLP Implementation:}
\begin{itemize}
    \item Uses SOS2 variables natively supported in CBC
    \item Branch-and-bound explores SOS2 branching decisions
    \item Linear relaxation provides bounds
\end{itemize}

\paragraph{Pyomo Implementation:}
\begin{itemize}
    \item Formulates as true MINLP (Mixed-Integer Non-Linear Program)
    \item Uses specialized solvers: IPOPT, BONMIN, COUENNE
    \item Applies outer approximation or branch-and-reduce algorithms
    \item Can solve exact non-linear formulation without approximation
\end{itemize}

\subsubsection{Approximation Error Analysis}

The piecewise linear approximation introduces error bounded by:

\begin{equation}
\epsilon_{\max} = \max_{i=0,\ldots,K-1} \max_{x \in [b_i, b_{i+1}]} |f(x) - \hat{f}(x)|
\end{equation}

where $\hat{f}(x)$ is the piecewise linear interpolation.

For $f(x) = x^\alpha$ with uniform breakpoints and $K$ intervals:

\begin{equation}
\epsilon_{\max} = O\left(\frac{L_f^{2\alpha}}{K^2}\right)
\end{equation}

\textbf{Typical Values:}
\begin{itemize}
    \item $K = 10$: Approximation error $\approx 0.1\%$ - $0.5\%$
    \item $K = 20$: Approximation error $\approx 0.02\%$ - $0.1\%$
\end{itemize}

\subsubsection{Computational Complexity}

\textbf{Variable Count:} For $n = |\mathcal{F}||\mathcal{C}|$ base problem size:
\begin{itemize}
    \item Linear solver: $2n$ variables
    \item NLN solver: $(K+4)n \approx 14n$ variables (for $K=10$)
    \item \textbf{Increase:} $7\times$ more variables
\end{itemize}

\textbf{Memory:} $O(Kn)$ for lambda variables

\textbf{Solve Time:} Empirically $2-4\times$ slower than linear due to SOS2 branching

\subsubsection{Advantages and Limitations}

\textbf{Advantages:}
\begin{itemize}
    \item Models realistic diminishing returns
    \item Economically justified (decreasing marginal productivity)
    \item Controllable approximation accuracy via $K$
    \item PuLP can solve with standard MILP techniques
    \item Pyomo can solve exactly with MINLP solvers
\end{itemize}

\textbf{Limitations:}
\begin{itemize}
    \item $7\times$ more variables than linear
    \item Approximation error (though controllable)
    \item Slower solve times
    \item Higher memory requirements
    \item Requires careful breakpoint selection
\end{itemize}

\subsection{Solver 3: Fractional Non-Linear with Dinkelbach (\texttt{solver\_runner\_NLD.py})}

\subsubsection{Mathematical Formulation}

This formulation uses a fractional objective to model efficiency as benefit per unit area allocated:

\begin{equation}
\max \frac{\sum_{f,c} w_k \cdot \text{benefit}_{k,c} \cdot A_{fc}}{\sum_{f,c} A_{fc} + \epsilon}
\label{eq:fractional_obj}
\end{equation}

where $\epsilon > 0$ is a small constant (typically $10^{-8}$) to avoid division by zero.

\textbf{Note:} The denominator is simply the sum of all allocated areas, with implicit unit cost. This represents ``benefit per unit area'' or efficiency of land use, where each area unit has equal weight (cost = 1).

\subsubsection{Dinkelbach's Algorithm}

Dinkelbach's algorithm converts the fractional program into a sequence of parametric linear programs:

\begin{algorithm}[H]
\caption{Dinkelbach's Algorithm for Fractional Programming}
\begin{algorithmic}[1]
\State Initialize $\lambda^{(0)} = 0$, $k = 0$
\Repeat
    \State Solve parametric problem:
    \begin{equation*}
    z(\lambda^{(k)}) = \max_{A,Y} \left\{\sum_{f,c} w \cdot b_{fc} \cdot A_{fc} - \lambda^{(k)} \sum_{f,c} A_{fc}\right\}
    \end{equation*}
    \State Let $(A^{(k)}, Y^{(k)})$ be the optimal solution
    \State Update parameter:
    \begin{equation*}
    \lambda^{(k+1)} = \frac{\sum_{f,c} w \cdot b_{fc} \cdot A^{(k)}_{fc}}{\sum_{f,c} A^{(k)}_{fc} + \epsilon}
    \end{equation*}
    \State $k \leftarrow k + 1$
\Until{$|z(\lambda^{(k)})| < \tau$ or $|\lambda^{(k)} - \lambda^{(k-1)}| < \tau$}
\State \Return $(A^{(k)}, Y^{(k)})$
\end{algorithmic}
\end{algorithm}

\textbf{Convergence:} Dinkelbach's algorithm converges superlinearly to the optimal solution of the fractional program.

\subsubsection{Problem Classification}
\begin{itemize}
    \item \textbf{Original Type:} Fractional Mixed-Integer Program
    \item \textbf{Transformed Type:} Sequence of MILPs
    \item \textbf{Variables per iteration:} $2|\mathcal{F}||\mathcal{C}|$
    \item \textbf{Iterations:} Typically 5-15 until convergence
\end{itemize}

\subsubsection{Solution Methods}

Each iteration solves a parametric MILP:
\begin{itemize}
    \item PuLP with CBC for each subproblem
    \item Warm-starting from previous solution
    \item Convergence tolerance $\tau = 10^{-6}$
\end{itemize}

\subsubsection{Computational Complexity}

\textbf{Variables:} Same as linear formulation ($2n$) per iteration

\textbf{Time per iteration:} Similar to linear MILP

\textbf{Total iterations:} $T \in [5, 15]$ typically

\textbf{Total time:} $O(T \cdot t_{\text{MILP}})$ where $t_{\text{MILP}}$ is single MILP solve time

\subsubsection{Advantages and Limitations}

\textbf{Advantages:}
\begin{itemize}
    \item Models efficiency/cost-benefit ratios directly
    \item No additional variables (unlike NLN)
    \item Exact optimal solution for fractional objective
    \item Fast convergence (superlinear)
    \item Well-studied algorithm with convergence guarantees
\end{itemize}

\textbf{Limitations:}
\begin{itemize}
    \item Requires multiple MILP solves
    \item Total time is $T \times$ single solve
    \item Limited to fractional objectives
    \item Denominator must be positive
    \item Cannot model general non-linearities
\end{itemize}

\subsection{Solver 4: Linear-Quadratic with Synergy (\texttt{solver\_runner\_LQ.py})}

\subsubsection{Mathematical Formulation}

This formulation combines linear returns with quadratic synergy effects:

\begin{equation}
\max \underbrace{\sum_{f,c} \left(\sum_k w_k \cdot a_{k,c}\right) A_{fc}}_{\text{Linear term}} + \underbrace{w_s \sum_f \sum_{\substack{c_1,c_2 \in \mathcal{G}_g\\ c_1 < c_2}} s_{c_1,c_2} \cdot Y_{fc_1} \cdot Y_{fc_2}}_{\text{Quadratic synergy term}}
\label{eq:lq_obj}
\end{equation}

where:
\begin{itemize}
    \item $s_{c_1,c_2}$: Synergy bonus for planting crops $c_1$ and $c_2$ together
    \item $w_s$: Weight for synergy bonus (typically $w_s = 0.1$)
    \item Synergy exists only for crop pairs in the same food group
\end{itemize}

\textbf{Synergy Matrix Construction:}
\begin{equation}
s_{c_1,c_2} = \begin{cases}
\beta > 0 & \text{if } c_1, c_2 \in \mathcal{G}_g \text{ for some } g \text{ and } c_1 \neq c_2\\
0 & \text{otherwise}
\end{cases}
\end{equation}

Typical value: $\beta = 0.1$ (10\% bonus).

\subsubsection{McCormick Linearization for PuLP}

The quadratic term $Y_{fc_1} \cdot Y_{fc_2}$ must be linearized for MILP solvers.

\textbf{McCormick Relaxation:}
For each synergy pair $(c_1, c_2)$ and farm $f$, introduce auxiliary variable:
\begin{equation}
Z_{fc_1c_2} \in \{0,1\}
\end{equation}

to represent the product $Y_{fc_1} \cdot Y_{fc_2}$.

\textbf{Linearization Constraints:}
\begin{align}
Z_{fc_1c_2} &\leq Y_{fc_1} \label{eq:mcc1}\\
Z_{fc_1c_2} &\leq Y_{fc_2} \label{eq:mcc2}\\
Z_{fc_1c_2} &\geq Y_{fc_1} + Y_{fc_2} - 1 \label{eq:mcc3}
\end{align}

\textbf{Correctness:}
\begin{itemize}
    \item If $Y_{fc_1} = 0$ or $Y_{fc_2} = 0$: (\ref{eq:mcc1})-(\ref{eq:mcc2}) force $Z_{fc_1c_2} = 0$
    \item If $Y_{fc_1} = Y_{fc_2} = 1$: (\ref{eq:mcc3}) forces $Z_{fc_1c_2} \geq 1$, combined with binary constraint gives $Z_{fc_1c_2} = 1$
\end{itemize}

This linearization is \textbf{exact} for binary variables (not an approximation).

\textbf{Linearized Objective:}
\begin{equation}
\max \sum_{f,c} \left(\sum_k w_k \cdot a_{k,c}\right) A_{fc} + w_s \sum_f \sum_{(c_1,c_2) \in \mathcal{S}} s_{c_1,c_2} \cdot Z_{fc_1c_2}
\end{equation}

where $\mathcal{S}$ is the set of synergy pairs.

\subsubsection{Problem Classification}

\textbf{For PuLP (Linearized):}
\begin{itemize}
    \item \textbf{Type:} Mixed-Integer Linear Program (after linearization)
    \item \textbf{Base Variables:} $2|\mathcal{F}||\mathcal{C}|$
    \item \textbf{Z Variables:} $|\mathcal{F}| \cdot |\mathcal{S}|$ where $|\mathcal{S}| = \sum_g \binom{|\mathcal{G}_g|}{2}$
    \item \textbf{Total Variables:} $2n + |\mathcal{F}| \cdot |\mathcal{S}|$
    \item \textbf{Linearization Constraints:} $3|\mathcal{F}| \cdot |\mathcal{S}|$
\end{itemize}

\textbf{For Pyomo/CQM (Native Quadratic):}
\begin{itemize}
    \item \textbf{Type:} Mixed-Integer Quadratic Program (MIQP)
    \item \textbf{Variables:} $2|\mathcal{F}||\mathcal{C}|$ (no Z variables needed)
    \item \textbf{Quadratic Terms:} $O(|\mathcal{F}| \cdot |\mathcal{S}|)$ in objective
\end{itemize}

\subsubsection{Variable Count Analysis}

For typical problem with 30 crops in 4 food groups:
\begin{itemize}
    \item Grains (3 crops): $\binom{3}{2} = 3$ pairs
    \item Legumes (2 crops): $\binom{2}{2} = 1$ pair
    \item Vegetables (4 crops): $\binom{4}{2} = 6$ pairs
    \item Fruits (1 crop): $\binom{1}{2} = 0$ pairs
\end{itemize}

Total synergy pairs: $|\mathcal{S}| = 10$

For $|\mathcal{F}| = 5$ farms, $|\mathcal{C}| = 30$ crops ($n = 150$):
\begin{itemize}
    \item Base variables: $2 \times 150 = 300$
    \item Z variables (PuLP): $5 \times 10 = 50$
    \item \textbf{PuLP total:} 350 variables
    \item \textbf{CQM/Pyomo total:} 300 variables
    \item \textbf{Linearization constraints:} $3 \times 50 = 150$
\end{itemize}

\subsubsection{Solution Methods}

\paragraph{PuLP Implementation:}
\begin{itemize}
    \item Linearizes using McCormick relaxation
    \item Solves with CBC as standard MILP
    \item Exact solution (no approximation error)
\end{itemize}

\paragraph{Pyomo Implementation:}

\begin{itemize}
    \item Uses native MIQP formulation
    \item Requires MIQP solver: Gurobi, CPLEX, CBC, or GLPK
    \item No linearization needed
    \item Solves quadratic objective directly
\end{itemize}

\paragraph{DWave CQM Implementation:}
\begin{itemize}
    \item Native quadratic objective support
    \item Quantum-classical hybrid solving
    \item Particularly well-suited for quadratic problems
\end{itemize}

\subsubsection{Computational Complexity}

\textbf{Variables:}
\begin{itemize}
    \item PuLP: $2n + |\mathcal{F}| \cdot |\mathcal{S}| \approx 2.3n$ (for typical $|\mathcal{S}|$)
    \item CQM/Pyomo: $2n$
    \item \textbf{Comparison to NLN:} $75\%$ fewer variables than NLN (2.3n vs 14n)
\end{itemize}

\textbf{Constraints:}
\begin{itemize}
    \item Base: Same as linear formulation
    \item Additional (PuLP): $3|\mathcal{F}| \cdot |\mathcal{S}|$ linearization constraints
\end{itemize}

\textbf{Solve Time:} Empirically comparable to linear, slightly slower than linear due to:
\begin{itemize}
    \item PuLP: Additional Z variables and constraints
    \item Pyomo: Quadratic terms require MIQP solver (more complex than LP relaxation)
\end{itemize}

\subsubsection{Advantages and Limitations}

\textbf{Advantages:}
\begin{itemize}
    \item Models crop interaction/synergy effects
    \item Exact solution (no approximation error)
    \item $75\%$ fewer variables than NLN
    \item Much faster than NLN
    \item McCormick linearization is exact for binary products
    \item Native quadratic support in CQM and MIQP solvers
    \item Economically justified (complementarity effects)
\end{itemize}

\textbf{Limitations:}
\begin{itemize}
    \item Linear returns (no diminishing returns)
    \item More variables than pure linear (due to Z in PuLP)
    \item Requires definition of synergy matrix
    \item Quadratic term may be harder to interpret
    \item PuLP requires linearization overhead
\end{itemize}

\subsection{Solver 5: BQUBO Formulation (\texttt{solver\_runner\_BQUBO.py})}

This solver introduces a fundamentally different formulation based on Binary Quadratic Unconstrained Binary Optimization (BQUBO) principles. Instead of continuous area variables, it uses only binary variables, simplifying the problem space at the cost of modeling flexibility.

\subsubsection{Mathematical Formulation}

The core idea is to discretize the decision space. Each farm-crop combination is represented by a single binary variable, indicating whether a fixed-size plot (e.g., 1 acre) is planted or not.

Decision variables:
\begin{itemize}
    \item $Y_{fc} \in \{0,1\}$: Binary variable indicating if a 1-acre plot of crop $c$ is planted on farm $f$.
\end{itemize}

The objective function remains linear but is now defined over binary variables:
\begin{equation}
\max \sum_{f \in \mathcal{F}} \sum_{c \in \mathcal{C}} \left(\sum_{k} w_k \cdot \text{attr}_{k,c}\right) \cdot Y_{fc}
\label{eq:bqubo_obj}
\end{equation}

The constraints are adapted to this binary model:
\begin{align}
\sum_{c \in \mathcal{C}} Y_{fc} &\leq \lfloor L_f \rfloor \quad \forall f \in \mathcal{F} \label{eq:land_binary}\\
\sum_{c \in \mathcal{G}_g} Y_{fc} &\geq N_{\min,g} \quad \forall f,g \label{eq:diversity_binary}
\end{align}
Constraint (\ref{eq:land_binary}) limits the *number* of 1-acre plots on a farm, which is a discretized version of the original land constraint.

\subsubsection{Problem Classification}
\begin{itemize}
    \item \textbf{Type:} 0-1 Integer Linear Program (ILP).
    \item \textbf{Variables:} $|\mathcal{F}||\mathcal{C}|$ (only Y variables). This is a $50\%$ reduction compared to the linear MILP formulation.
    \item \textbf{Constraints:} $O(|\mathcal{F}|(1 + 2|\mathcal{G}|))$.
    \item \textbf{Linearity:} Fully linear objective and constraints.
\end{itemize}

\subsubsection{Solution Methods}

\paragraph{PuLP Implementation:}
Solves the 0-1 ILP directly using the CBC solver. This is highly efficient due to the absence of continuous variables.

\paragraph{DWave BQUBO Implementation:}
This is the key part of this solver. The Constrained Quadratic Model (CQM) is converted into a Binary Quadratic Model (BQM) using the \texttt{cqm\_to\_bqm} utility.
\begin{itemize}
    \item \textbf{Conversion:} This process transforms the constrained problem into an unconstrained one by embedding the constraints (\ref{eq:land_binary}) and (\ref{eq:diversity_binary}) into the objective function as quadratic penalty terms. The result is a BQM suitable for quantum annealers.
    \item \textbf{Solver:} The resulting BQM is solved using the \texttt{LeapHybridBQMSampler}, which is designed for these types of problems and can leverage QPU resources more effectively than the general CQM sampler.
\end{itemize}

\subsubsection{Computational Complexity}

\textbf{Variable Count:} For $n = |\mathcal{F}||\mathcal{C}|$:
\begin{itemize}
    \item Total variables: $n$. This is the leanest formulation.
\end{itemize}

\textbf{Space Complexity:} $O(n)$ for variables. The BQM conversion can introduce many new quadratic terms, potentially increasing memory for the BQM object itself.

\textbf{Solve Time:}
\begin{itemize}
    \item \textbf{PuLP:} Extremely fast, often faster than the standard linear MILP due to having half the variables and no continuous relaxation complexities.
    \item \textbf{DWave:} Includes a non-trivial classical pre-processing time for the \texttt{cqm\_to\_bqm} conversion, followed by the hybrid solver time. The goal is better scaling for very large, complex problems where the BQM formulation can be efficiently processed by the QPU.
\end{itemize}

\subsubsection{Advantages and Limitations}

\textbf{Advantages:}
\begin{itemize}
    \item \textbf{Fewest Variables:} $50\%$ fewer variables than other formulations.
    \item \textbf{Speed (Classical):} PuLP solution is typically the fastest.
    \item \textbf{QPU Compatibility:} The BQM formulation is the native format for D-Wave's quantum annealers, enabling more direct use of quantum hardware.
    \item \textbf{Simplicity:} The model is conceptually simple (select or don't select).
\end{itemize}

\textbf{Limitations:}
\begin{itemize}
    \item \textbf{Modeling Inflexibility:} The primary drawback. It assumes all plantations are of a fixed, uniform size (1 acre). It cannot allocate fractional areas or areas of different sizes, which is a significant loss of modeling fidelity.
    \item \textbf{Discretization Error:} The land availability constraint is discretized ($\lfloor L_f \rfloor$), which may not perfectly represent the available land.
    \item \textbf{Unrealistic Model:} May not be suitable for real-world scenarios where variable allocation sizes are essential.
\end{itemize}

\section{Comparative Analysis}

\subsection{Variable Count Comparison}

Table~\ref{tab:variables} summarizes the variable counts for a problem with $n = |\mathcal{F}||\mathcal{C}|$ base size.

\begin{table}[h]
\centering
\caption{Variable count comparison ($n = |\mathcal{F}||\mathcal{C}|$, $K=10$, $|\mathcal{S}| \approx 0.15|\mathcal{C}|$)}
\label{tab:variables}
\begin{tabular}{lcccc}
\toprule
\textbf{Solver} & \textbf{Base} & \textbf{Extra} & \textbf{Total} & \textbf{Ratio} \\
\midrule
BQUBO & $n$ & $0$ & $n$ & $0.50\times$ \\
Linear & $2n$ & $0$ & $2n$ & $1.00\times$ \\
NLN (PuLP/CQM) & $2n$ & $12n$ & $14n$ & $7.00\times$ \\
NLN (Pyomo) & $2n$ & $12n$ & $14n$ & $7.00\times$ \\
NLD & $2n$ & $0$ & $2n$ & $1.00\times$ \\
LQ (PuLP) & $2n$ & $0.3n$ & $2.3n$ & $1.15\times$ \\
LQ (CQM/Pyomo) & $2n$ & $0$ & $2n$ & $1.00\times$ \\
\bottomrule
\end{tabular}
\end{table}

\textbf{Key Observations:}
\begin{itemize}
    \item NLN has the highest variable count (7× baseline).
    \item BQUBO has the lowest variable count (0.5x baseline).
    \item LQ with native quadratic matches the linear baseline.
    \item NLD iterates but each iteration has baseline variables.
\end{itemize}

\subsection{Constraint Count Comparison}

\begin{table}[h]
\centering
\caption{Constraint count comparison (excluding food group constraints)}
\label{tab:constraints}
\begin{tabular}{lcc}
\toprule
\textbf{Solver} & \textbf{Base Constraints} & \textbf{Extra Constraints} \\
\midrule
BQUBO & $|\mathcal{F}|$ & $0$ \\
Linear & $|\mathcal{F}| + 2n$ & $0$ \\
NLN & $|\mathcal{F}| + 2n$ & $3n$ (piecewise) \\
NLD & $|\mathcal{F}| + 2n$ & $0$ \\
LQ (PuLP) & $|\mathcal{F}| + 2n$ & $0.9n$ (linearization) \\
LQ (CQM/Pyomo) & $|\mathcal{F}| + 2n$ & $0$ \\
\bottomrule
\end{tabular}
\end{table}

\subsection{Computational Performance}

\subsubsection{Empirical Solve Time Analysis}

Based on benchmark results with problem size $n = 150$ (5 farms, 30 crops):

\begin{table}[h]
\centering
\caption{Typical solve times and accuracies}
\label{tab:performance}
\begin{tabular}{lcccc}
\toprule
\textbf{Solver} & \textbf{Variables} & \textbf{PuLP Time} & \textbf{Pyomo Time} & \textbf{Accuracy} \\
\midrule
BQUBO & 150 & 0.10s & N/A & Exact (but discretized) \\
Linear & 300 & 0.15s & 0.08s & Exact \\
NLN & 2,100 & 0.45s & 0.25s & 0.1-0.5\% \\
NLD & 300 & 1.2s (8 iter) & N/A & Exact \\
LQ & 350/300 & 0.18s & 0.09s & Exact \\
\bottomrule
\end{tabular}
\end{table}

\textbf{Observations:}
\begin{itemize}
    \item BQUBO is fastest classically due to fewest variables.
    \item Linear and LQ have similar performance.
    \item NLN is $2-3\times$ slower due to increased variables.
    \item NLD total time depends on iteration count.
    \item Pyomo generally faster than PuLP due to better solver (Gurobi).
\end{itemize}

\subsubsection{Scaling Behavior}

Theoretical complexity for problem size $n$:

\begin{equation}
t_{\text{solve}} = \begin{cases}
O(2^n \cdot \text{poly}(n)) & \text{worst case (all solvers)}\\
O(n \log n) - O(n^2) & \text{empirical for structured problems}
\end{cases}
\end{equation}

In practice, solve time scaling:
\begin{itemize}
    \item BQUBO: $t \propto n^{1.1}$ (best classical scaling)
    \item Linear: $t \propto n^{1.2}$ (sub-quadratic)
    \item NLN: $t \propto n^{1.4}$ (slightly worse due to more variables)
    \item NLD: $t \propto T \cdot n^{1.2}$ where $T \approx 8$ iterations
    \item LQ: $t \propto n^{1.25}$ (similar to linear)
\end{itemize}

\subsection{Memory Requirements}

Memory usage scales primarily with variable count:

\begin{equation}
\text{Memory} \approx 8 \times (\text{num\_variables}) + 16 \times (\text{num\_constraints})
\end{equation}

For $n = 10,000$ (large problem):
\begin{itemize}
    \item BQUBO: $\approx 320$ KB
    \item Linear: $\approx 640$ KB
    \item NLN: $\approx 4.5$ MB
    \item NLD: $\approx 640$ KB per iteration
    \item LQ: $\approx 740$ KB
\end{itemize}

\subsection{Approximation Error}

\begin{table}[h]
\centering
\caption{Approximation characteristics}
\label{tab:error}
\begin{tabular}{lccc}
\toprule
\textbf{Solver} & \textbf{Error Type} & \textbf{Magnitude} & \textbf{Controllable} \\
\midrule
BQUBO & Discretization & Problem-dependent & No \\
Linear & None (exact) & 0\% & N/A \\
NLN (PuLP) & Piecewise approx. & 0.1-0.5\% & Yes (via $K$) \\
NLN (Pyomo) & None (exact MINLP) & 0\% & N/A \\
NLD & None (exact) & 0\% & N/A \\
LQ & None (exact) & 0\% & N/A \\
\bottomrule
\end{tabular}
\end{table}

\textbf{Important Note:} Only NLN with PuLP has approximation error. BQUBO has a modeling limitation (discretization) rather than a solver approximation error.

\subsection{Model Realism}

\begin{table}[h]
\centering
\caption{Modeling capabilities and economic realism}
\label{tab:realism}
\begin{tabular}{lccccc}
\toprule
\textbf{Feature} & \textbf{Linear} & \textbf{NLN} & \textbf{NLD} & \textbf{LQ} & \textbf{BQUBO} \\
\midrule
Diminishing returns & No & Yes & No & No & No \\
Constant returns & Yes & No & Yes & Yes & Yes (per plot) \\
Cost-benefit ratio & No & No & Yes & No & No \\
Crop synergy & No & No & No & Yes & No \\
Variable allocation & Yes & Yes & Yes & Yes & No \\
Economic justification & Medium & High & High & Medium-High & Low \\
\bottomrule
\end{tabular}
\end{table}

\section{Recommendations}

\subsection{Selection Criteria}

\subsubsection{Choose Linear (\texttt{solver\_runner.py}) when:}
\begin{itemize}
    \item Simplicity and speed are paramount.
    \item Problem size is large ($n > 10,000$).
    \item Constant returns are acceptable.
    \item A baseline comparison is needed.
\end{itemize}

\subsubsection{Choose NLN (\texttt{solver\_runner\_NLN.py}) when:}
\begin{itemize}
    \item Diminishing returns are critical to model.
    \item Economic realism is essential.
    \item Problem size is small to medium ($n < 5,000$).
    \item Can afford $2-3\times$ longer solve time.
\end{itemize}

\subsubsection{Choose NLD (\texttt{solver\_runner\_NLD.py}) when:}
\begin{itemize}
    \item Efficiency/cost-benefit ratio is the primary concern.
    \item Can afford multiple iterations ($\approx 10$ MILP solves).
    \item The fractional objective naturally fits the problem.
\end{itemize}

\subsubsection{Choose LQ (\texttt{solver\_runner\_LQ.py}) when:}
\begin{itemize}
    \item Crop interaction effects are important.
    \item A balance between complexity and realism is desired.
    \item DWave quantum hardware is available (native quadratic support).
\end{itemize}

\subsubsection{Choose BQUBO (\texttt{solver\_runner\_BQUBO.py}) when:}
\begin{itemize}
    \item The problem naturally involves binary choices (e.g., select/deselect projects).
    \item Maximum classical performance on a simplified model is required.
    \item The inflexibility of fixed-size allocations is acceptable.
    \item The primary goal is to formulate a problem for a BQM-based quantum solver.
\end{itemize}

\subsection{Performance-Accuracy Trade-off}

Figure~\ref{fig:tradeoff} conceptually illustrates the trade-off:

\begin{verbatim}
Modeling
Capability
    ^
    |     NLN (Pyomo)
    |       *
    |      /|\
    |     / | \
    |    /  |  \
    |   /  LQ   \
    |  /    *    \
    | /  NLD  \   \
    |/    *    \   \
    |  Linear    \   BQUBO
    |    *        \    *
    +-------------------> Computational
                         Efficiency
\end{verbatim}

BQUBO and Linear offer maximum efficiency with basic modeling.
LQ balances efficiency and advanced features (synergy).
NLD provides efficiency ratios at moderate cost.
NLN (Pyomo) offers maximum realism at highest computational cost.

\section{Implementation Details}

\subsection{Common Framework}

All solvers share:
\begin{itemize}
    \item Same input data structure (scenarios)
    \item Same output format (JSON results)
    \item Progress bars with \texttt{tqdm}
    \item Comprehensive metadata logging
\end{itemize}

\subsection{Solver-Specific Components}

\subsubsection{BQUBO: CQM to BQM Conversion}
\begin{verbatim}
from dimod import cqm_to_bqm

# cqm is a ConstrainedQuadraticModel with only binary variables
bqm, invert = cqm_to_bqm(cqm)

# sampler is a LeapHybridBQMSampler
sampleset = sampler.sample(bqm)
\end{verbatim}

\subsubsection{NLN: Piecewise Approximation Module}
\begin{verbatim}
from piecewise_approximation import PiecewiseApproximation

approx = PiecewiseApproximation(...)
\end{verbatim}

\subsubsection{NLD: Dinkelbach Iteration}
\begin{verbatim}
lambda_param = 0.0
for iteration in range(max_iter):
    # Solve parametric MILP
    obj = benefit - lambda_param * sum(A)
    # Update parameter
    lambda_param = benefit / (sum(A) + epsilon)
\end{verbatim}

\subsubsection{LQ: Synergy Matrix Generation}
\begin{verbatim}
synergy_matrix = {}
for group_name, crops in food_groups.items():
    for c1, c2 in itertools.combinations(crops, 2):
        synergy_matrix[c1][c2] = 0.1
\end{verbatim}

\subsection{Testing and Validation}

All solvers validated by:
\begin{enumerate}
    \item Constraint satisfaction verification
    \item Objective value comparison across solvers
    \item Scaling tests (5 to 1535 farms)
    \item Multiple runs for statistical confidence
\end{enumerate}

\section{Conclusions}

This technical report analyzed five distinct approaches to the food optimization problem, each with different mathematical properties and computational characteristics:

\begin{enumerate}
    \item \textbf{Linear:} Simplest and fastest, suitable for large-scale problems.
    \item \textbf{NLN:} Most realistic with diminishing returns, but $7\times$ more variables.
    \item \textbf{NLD:} Efficiency-focused with exact fractional optimization.
    \item \textbf{LQ:} Balanced approach with synergy modeling and exact solutions.
    \item \textbf{BQUBO:} Fastest classically with the fewest variables, but lowest modeling fidelity due to fixed allocation size. Best for problems that are inherently binary or as a direct path to BQM solvers.
\end{enumerate}

\textbf{Key Findings:}
\begin{itemize}
    \item Variable count varies from $n$ (BQUBO) to $14n$ (NLN).
    \item Solve time ranges from 0.1s to 1.2s for $n=150$ problems.
    \item BQUBO and NLN (PuLP) introduce modeling limitations (discretization) or errors (approximation), while others are exact representations of their respective objectives.
    \item LQ offers a strong balance: exact solution, reasonable complexity, and advanced synergy modeling.
    \item BQUBO is the most direct, but also most simplified, path for leveraging D-Wave's BQM solvers.
    \item The choice of solver depends heavily on the specific application requirements, desired model realism, and available computational resources.
\end{itemize}

\textbf{Future Work:}
\begin{itemize}
    \item Hybrid approaches combining multiple objective types.
    \item Adaptive breakpoint selection for NLN.
    \item Parallel implementation for NLD iterations.
    \item Deeper analysis of quantum vs. classical performance on large LQ and BQUBO problems.
    \item Multi-objective optimization extensions.
\end{itemize}

\section{References}

\begin{enumerate}
    \item CBC (COIN-OR Branch and Cut): \url{https://github.com/coin-or/Cbc}
    \item PuLP Documentation: \url{https://coin-or.github.io/pulp/}
    \item Pyomo Optimization Modeling: \url{http://www.pyomo.org/}
    \item D-Wave Ocean SDK: \url{https://docs.ocean.dwavesys.com/}
    \item Dinkelbach, W. (1967). "On nonlinear fractional programming." Management Science.
    \item McCormick, G.P. (1976). "Computability of global solutions to factorable nonconvex programs." Mathematical Programming.
    \item Beale, E.M.L. and Tomlin, J.A. (1970). "Special facilities in a general mathematical programming system for non-convex problems using ordered sets of variables." Proceedings of the Fifth International Conference on Operational Research.
\end{enumerate}

\appendix

\section{Variable Notation Summary}

\begin{table}[h]
\centering
\begin{tabular}{ll}
\toprule
\textbf{Symbol} & \textbf{Description} \\
\midrule
$\mathcal{F}$ & Set of farms \\
$\mathcal{C}$ & Set of crops \\
$\mathcal{G}$ & Set of food groups \\
$n$ & Problem size ($|\mathcal{F}||\mathcal{C}|$) \\
$A_{fc}$ & Continuous area allocated to crop $c$ on farm $f$ \\
$Y_{fc}$ & Binary selection indicator for crop $c$ on farm $f$ \\
$L_f$ & Land availability on farm $f$ \\
$K$ & Number of breakpoints for piecewise approximation \\
$\lambda_{fc,i}$ & Convex combination weights (NLN) \\
$Z_{fc_1c_2}$ & Linearization variable for $Y_{fc_1} \cdot Y_{fc_2}$ (LQ) \\
$\alpha$ & Power for diminishing returns (typically 0.548) \\
$\beta$ & Synergy bonus value (typically 0.1) \\
$w_k$ & Weight for objective component $k$ \\
\bottomrule
\end{tabular}
\end{table}

\section{Solver Command Line Usage}

\subsection{BQUBO Solver}
\begin{verbatim}
python solver_runner_BQUBO.py --scenario simple
\end{verbatim}

\subsection{Linear Solver}
\begin{verbatim}
python solver_runner.py --scenario simple
python solver_runner.py --scenario full_family
\end{verbatim}

\subsection{NLN Solver}
\begin{verbatim}
python solver_runner_NLN.py --scenario simple --power 0.548 --breakpoints 10
python solver_runner_NLN.py --scenario full_family --power 0.548 --breakpoints 20
\end{verbatim}

\subsection{NLD Solver}
\begin{verbatim}
python solver_runner_NLD.py --scenario simple
python solver_runner_NLD.py --scenario full_family --max-iterations 20
\end{verbatim}

\subsection{LQ Solver}
\begin{verbatim}
python solver_runner_LQ.py --scenario simple
python solver_runner_LQ.py --scenario full_family
\end{verbatim}

\end{document}
