\subsubsection{Overview}
The benchmark will be comprehensive of two formulations:

\begin{enumerate}
    \item \textbf{Binary Formulation} (Even Grid): Used when land is divided into equal-sized plots
    \item \textbf{Continuous Formulation} (Uneven Distribution): Used when farms have varying sizes
\end{enumerate}


The script solves the optimization problem using three different methods:
\begin{itemize}
    \item PuLP with Gurobi solver (classical optimization)
    \item D-Wave Hybrid CQM Sampler (quantum-classical hybrid)
    \item D-Wave Hybrid BQM Sampler (quantum-enabled via CQM→BQM conversion)
\end{itemize}



\subsubsection{Objective Functions}

\paragraph{Continuous Formulation Objective}

The objective function maximizes the weighted sum of agricultural value metrics, normalized by total available land:

$$\max \quad Z = \frac{1}{\sum_{f \in F} L_f} \sum_{f \in F} \sum_{c \in C} B_c \cdot A_{f,c}$$

where the composite value $v_c$ for crop $c$ is defined as:

$$B_c = w_{nv} \cdot v_{nv,c} + w_{nd} \cdot v_{nd,c} - w_{ei} \cdot v_{ei,c} + w_{af} \cdot v_{af,c} + w_{su} \cdot v_{su,c}$$

\textbf{Note:} This is equivalent to \ref{eq:linear_obj} after appropriate renormalization.

\paragraph{Binary Formulation Objective}

For the binary formulation, the objective accounts for the fixed area $a_p$ of each plot:

$$\max \quad Z = \frac{1}{\sum_{p \in F} a_p} \sum_{p \in F} \sum_{c \in C} a_p \cdot B_c \cdot Y_{p,c}$$

where $B_c$ is defined identically as in the continuous formulation.

\textbf{Interpretation:} Each selected assignment contributes the plot's area multiplied by the crop's value density.

\subsubsection{Constraints}

\paragraph{Continuous Formulation Constraints}

\subparagraph{Land Availability Constraints}

Each farm cannot allocate more land than available:

$$\sum_{c \in \mathcal{C}} A_{f,c} \leq L_f \quad \forall f \in \mathcal{F}$$

\textbf{Label:} \texttt{Land\_Availability\_\{farm\}}

\subparagraph{Minimum Planting Area Constraints}

If a crop is selected on a farm, it must occupy at least the minimum required area:

$$A_{f,c} \geq A_{min,c} \cdot Y_{f,c} \quad \forall f \in \mathcal{F}, c \in C$$


\textbf{Logical Interpretation:}
\begin{itemize}
    \item If $Y_{f,c} = 1$: $A_{f,c} \geq A_{min,c}$ (enforces minimum area)
    \item If $Y_{f,c} = 0$: $A_{f,c} \geq 0$ (no planting, so area can be zero)
\end{itemize}

\textbf{Label:} \texttt{Min\_Area\_If\_Selected\_\{farm\}\_\{crop\}}

\subparagraph{Maximum Planting Area Constraints}

If a crop is not selected, its allocated area must be zero:

$$A_{f,c} \leq L_f \cdot Y_{f,c} \quad \forall f \in \mathcal{F}, c \in C$$


\textbf{Logical Interpretation:}
\begin{itemize}
    \item If $Y_{f,c} = 1$: $A_{f,c} \leq L_f$ (area can be up to farm size)
    \item If $Y_{f,c} = 0$: $A_{f,c} \leq 0$ (forces area to zero)
\end{itemize}

\textbf{Label:} \texttt{Max\_Area\_If\_Selected\_\{farm\}\_\{crop\}}

\subparagraph{Food Group Minimum Constraints}

At least a minimum number of different crops from specified food groups must be cultivated:

$$\sum_{f \in \mathcal{F}}\sum_{c \in \mathcal{G}} Y_{f,c} \geq N_{min,g} \quad \forall g \in \mathcal{G} \text{ where } N_{min,g} \text{ is defined}$$

\textbf{Label:} \texttt{Food\_Group\_Min\_\{group\}}

\subparagraph{Food Group Maximum Constraints}

A maximum number of different crops from specified food groups must not be exceeded:

$$\sum_{f \in \mathcal{F}}\sum_{c \in \mathcal{G}} Y_{f,c} \leq N_{max,g} \quad \forall g \in \mathcal{G} \text{ where } N_{max,g} \text{ is defined}$$

\textbf{Label:} \texttt{Food\_Group\_Max\_\{group\}}

\paragraph{Binary Formulation Constraints}

\subparagraph{Plot Assignment Constraints}

Each plot can be assigned to at most one crop (or remain idle):

$$\sum_{c \in \mathcal{C}} Y_{p,c} \leq 1 \quad \forall p \in \mathcal{F}$$



\textbf{Interpretation:}
\begin{itemize}
    \item $\sum_{c \in C} Y_{p,c} = 0$: Plot remains idle
    \item $\sum_{c \in C} Y_{p,c} = 1$: Plot is assigned to exactly one crop
\end{itemize}

\textbf{Label:} \texttt{Max\_Assignment\_\{plot\}}

\subparagraph{Minimum Plots Per Crop Constraints}

For crops with minimum planting area requirements, the constraint is converted to a minimum number of plots:

$$\sum_{p \in \mathcal{F}} Y_{p,c} \geq \left\lceil \frac{A_{min,c}}{a_p} \right\rceil \quad \forall c \in \mathcal{F} \text{ where } A_{min,c} > 0$$

where $a_p$ is the area of each plot (assumed equal in even grid).



\textbf{Interpretation:} If a crop $c$ requires minimum area $A_{min,c}$, it must be planted on at least $\lceil A_{min,c} / a_p \rceil$ plots.

\textbf{Label:} \texttt{Min\_Plots\_\{crop\}}

\subparagraph{Maximum Plots Per Crop Constraints}

For crops with maximum planting area limits, the constraint is converted to a maximum number of plots:

$$\sum_{p \in \mathcal{F}} Y_{p,c} \leq \left\lfloor \frac{A_{max,c}}{a_p} \right\rfloor \quad \forall c \in \mathcal{C} \text{ where } A_{max,c} \text{ is defined}$$



\textbf{Interpretation:} If a crop $c$ has maximum area $A_{max,c}$, it can be planted on at most $\lfloor A_{max,c} / a_p \rfloor$ plots.


\textbf{Label:} \texttt{Max\_Plots\_\{crop\}}




\subparagraph{Food Group Constraints}

The same food group minimum and maximum constraints apply as in the continuous formulation:

$$\sum_{p \in \mathcal{F}}\sum_{c \in \mathcal{G}} Y_{p,c} \geq N_{min,g} \quad \forall g \in \mathcal{G}$$
$$\sum_{p \in \mathcal{F}}\sum_{c \in \mathcal{G}} Y_{p,c} \leq N_{max,g} \quad \forall g \in \mathcal{G}$$

\textbf{Labels:} \texttt{Food\_Group\_Min\_\{group\}}, \texttt{Food\_Group\_Max\_\{group\}}