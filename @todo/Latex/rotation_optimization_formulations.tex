\documentclass[12pt,a4paper]{article}
\usepackage[utf8]{inputenc}
\usepackage{amsmath,amssymb,amsthm}
\usepackage{algorithm,algpseudocode}
\usepackage{graphicx}
\usepackage{booktabs}
\usepackage{hyperref}
\usepackage{geometry}
\usepackage{enumitem}
\usepackage{tikz}
\usetikzlibrary{shapes,arrows,positioning}

\geometry{margin=1in}

\title{\textbf{Comprehensive Mathematical Formulation}\\
\Large Crop Rotation Optimization with Temporal and Spatial Synergies:\\
Classical MIP and Hierarchical Quantum-Classical Decomposition}

\author{OQI-UC002-DWave Project}
\date{December 15, 2025}

\begin{document}

\maketitle

\begin{abstract}
This technical report presents a comprehensive mathematical formulation of the crop rotation optimization problem with temporal (rotation synergies) and spatial (neighbor interactions) components. We present three solution approaches: (1) classical Mixed-Integer Programming (MIP) solved with Gurobi, (2) hierarchical quantum-classical decomposition with D-Wave quantum annealing, and (3) direct BQM formulation for comparison. We provide detailed formulations, parameter specifications, computational complexity analysis, and empirical validation results demonstrating equivalence between formulations and quantifying quantum speedup potential.
\end{abstract}

\tableofcontents
\newpage

%==============================================================================
\section{Introduction}
%==============================================================================

Crop rotation planning is a fundamental agricultural optimization problem that seeks to maximize nutritional value, environmental sustainability, and economic viability while respecting agronomic constraints. The problem becomes computationally challenging when incorporating:
\begin{itemize}
    \item \textbf{Temporal dependencies}: Rotation synergies between crops planted in consecutive periods
    \item \textbf{Spatial interactions}: Neighboring farm effects (pollination, pest control, resource sharing)
    \item \textbf{Diversity requirements}: Encouraging varied crop portfolios across farms
    \item \textbf{Scale}: Hundreds of farms with dozens of crop options over multiple years
\end{itemize}

This report formalizes the complete problem as a Quadratic Unconstrained Binary Optimization (QUBO) problem suitable for quantum annealing, presents a classical MIP formulation for ground truth comparison, and develops a hierarchical decomposition strategy enabling solution of large-scale instances via quantum-classical hybrid computing.

\subsection{Problem Context}

Traditional crop rotation planning typically considers:
\begin{itemize}
    \item \textbf{Nitrogen fixing}: Legumes replenish soil nitrogen for subsequent grain crops
    \item \textbf{Pest suppression}: Alternating crop families breaks pest/disease cycles  
    \item \textbf{Soil health}: Root diversity maintains soil structure and microbiome
    \item \textbf{Market diversity}: Risk hedging through varied crop portfolios
\end{itemize}

Our formulation extends classical rotation planning by incorporating:
\begin{itemize}
    \item Explicit spatial coupling between neighboring farms
    \item Multi-attribute benefit functions (nutrition, environment, economics)
    \item Soft one-hot constraints enabling penalty-based feasibility
    \item Hierarchical aggregation (27 crops $\rightarrow$ 6 families) for scalability
\end{itemize}

%==============================================================================
\section{Problem Formulation: Core Components}
%==============================================================================

\subsection{Sets and Indices}

\begin{align}
    \mathcal{F} &= \{f_1, f_2, \ldots, f_F\} && \text{Set of farms} \\
    \mathcal{C} &= \{c_1, c_2, \ldots, c_C\} && \text{Set of crops/foods} \\
    \mathcal{T} &= \{1, 2, 3\} && \text{Set of rotation periods (years)} \\
    \mathcal{G} &= \{\text{Grains, Legumes, Vegetables, \ldots}\} && \text{Set of crop families}
\end{align}

For our benchmark scenarios:
\begin{itemize}
    \item Family-level problems: $|\mathcal{C}| = 6$ (one representative per family)
    \item Full crop-level problems: $|\mathcal{C}| = 27$ (all available crops)
    \item Number of farms: $|\mathcal{F}| \in \{5, 10, 15, 20, 25, 40, 50, 75, 100, 150, 200, 250, 350, 500, 1000\}$
    \item Rotation periods: $|\mathcal{T}| = 3$ (3-year rotation)
\end{itemize}

\subsection{Parameters}

\subsubsection{Farm Characteristics}
\begin{align}
    A_f &\in \mathbb{R}^+ && \text{Land availability for farm } f \in \mathcal{F} \text{ (hectares)}
\end{align}

\textbf{Empirical values} (from rotation scenarios):
\begin{itemize}
    \item $A_f \sim 4.75$ ha per farm (mean)
    \item Total area scales linearly: $\sum_{f \in \mathcal{F}} A_f \approx 4.75 \cdot |\mathcal{F}|$ ha
\end{itemize}

\subsubsection{Crop Benefits}
\begin{align}
    B_c &\in [0,1] && \text{Composite benefit score for crop } c \in \mathcal{C}
\end{align}

Computed as weighted combination of attributes:
\begin{equation}
B_c = \sum_{a \in \mathcal{A}} w_a \cdot \text{attr}_{c,a}
\end{equation}
where $\mathcal{A} = \{\text{nutrition}, \text{nutrient\_density}, \text{environment}, \text{affordability}, \text{sustainability}\}$

\textbf{Weights} (standard configuration):
\begin{equation}
\mathbf{w} = \begin{bmatrix}
0.25 \\ 0.20 \\ 0.25 \\ 0.15 \\ 0.15
\end{bmatrix} \quad \text{for} \quad \begin{bmatrix}
\text{nutritional\_value} \\
\text{nutrient\_density} \\
\text{environmental\_impact} \\
\text{affordability} \\
\text{sustainability}
\end{bmatrix}
\end{equation}

\subsubsection{Rotation Synergy Matrix}
\begin{align}
    R_{c_1,c_2} &\in [-1,1] && \text{Synergy from planting crop } c_1 \text{ before } c_2
\end{align}

\textbf{Generation method}:
\begin{itemize}
    \item Set random seed for reproducibility
    \item Initialize $R \in \mathbb{R}^{C \times C}$ with zeros
    \item For each crop pair $(c_1, c_2)$:
    \begin{equation}
    R_{c_1,c_2} = \begin{cases}
    \sim \mathcal{U}(0.5, 1.0) & \text{if favorable (with prob. } 1-\rho\text{)} \\
    \sim \mathcal{U}(-0.8, -0.3) & \text{if unfavorable (with prob. } \rho\text{)}
    \end{cases}
    \end{equation}
    where $\rho = 0.7$ is the frustration ratio
\end{itemize}

\textbf{Agronomic interpretation}:
\begin{itemize}
    \item $R_{\text{Legumes},\text{Grains}} > 0$: Legumes fix nitrogen, benefiting grains
    \item $R_{\text{Grains},\text{Grains}} < 0$: Same-family planting discouraged (pest cycles)
    \item $|R_{c_1,c_2}| \propto$ strength of interaction
\end{itemize}

\subsubsection{Spatial Neighbor Graph}
\begin{align}
    \mathcal{E} &\subseteq \mathcal{F} \times \mathcal{F} && \text{Set of farm neighbor pairs}
\end{align}

\textbf{Construction} (grid-based $k$-NN):
\begin{enumerate}
    \item Arrange farms in $\lceil \sqrt{F} \rceil \times \lceil \sqrt{F} \rceil$ grid
    \item Assign coordinates: $\text{pos}(f_i) = (i \bmod \text{side}, \lfloor i / \text{side} \rfloor)$
    \item For each farm, connect to $k=4$ nearest neighbors (Euclidean distance)
    \item Result: $|\mathcal{E}| \approx 2F$ (approximately 4 edges per farm)
\end{enumerate}

\subsubsection{Tuning Parameters}
\begin{align}
    \gamma &= 0.2 && \text{Rotation synergy strength} \\
    \gamma_s &= 0.1 && \text{Spatial synergy strength} \\
    \alpha &= 0.15 && \text{Diversity bonus coefficient} \\
    \lambda &= 3.0 && \text{One-hot penalty coefficient}
\end{align}

%==============================================================================
\section{Mathematical Formulation I: Mixed-Integer Program}
%==============================================================================

\subsection{Decision Variables}

\begin{equation}
Y_{f,c,t} \in \{0,1\} \quad \forall f \in \mathcal{F}, c \in \mathcal{C}, t \in \mathcal{T}
\end{equation}

\textbf{Interpretation}: $Y_{f,c,t} = 1$ if and only if farm $f$ plants crop $c$ in period $t$.

\textbf{Problem size}:
\begin{equation}
\text{Number of binary variables} = |\mathcal{F}| \times |\mathcal{C}| \times |\mathcal{T}| = F \cdot C \cdot 3
\end{equation}

\textbf{Example sizes}:
\begin{itemize}
    \item 5 farms, 6 families: $5 \times 6 \times 3 = 90$ variables
    \item 100 farms, 27 crops: $100 \times 27 \times 3 = 8,100$ variables  
    \item 1000 farms, 27 crops: $1000 \times 27 \times 3 = 81,000$ variables
\end{itemize}

\subsection{Objective Function}

\begin{equation}
\max_{\mathbf{Y}} \quad Z = Z_{\text{benefit}} + Z_{\text{rotation}} + Z_{\text{spatial}} + Z_{\text{diversity}} - Z_{\text{penalty}}
\end{equation}

\subsubsection{Component 1: Base Benefit}

\begin{equation}
Z_{\text{benefit}} = \sum_{f \in \mathcal{F}} \sum_{c \in \mathcal{C}} \sum_{t \in \mathcal{T}} B_c \cdot A_f \cdot Y_{f,c,t}
\end{equation}

\textbf{Interpretation}: Total agricultural value generated across all farms and periods, weighted by land area and crop benefit scores.

\subsubsection{Component 2: Temporal Rotation Synergies}

\begin{equation}
Z_{\text{rotation}} = \gamma \sum_{f \in \mathcal{F}} \sum_{c_1 \in \mathcal{C}} \sum_{c_2 \in \mathcal{C}} \sum_{t=1}^{2} R_{c_1,c_2} \cdot A_f \cdot Y_{f,c_1,t} \cdot Y_{f,c_2,t+1}
\end{equation}

\textbf{Interpretation}: Rewards beneficial crop sequences (e.g., legumes before grains) and penalizes unfavorable sequences (e.g., same family repeated).

\textbf{Quadratic terms}: $Y_{f,c_1,t} \cdot Y_{f,c_2,t+1}$ creates coupling between consecutive periods.

\subsubsection{Component 3: Spatial Neighbor Synergies}

\begin{equation}
Z_{\text{spatial}} = \gamma_s \sum_{(f_1,f_2) \in \mathcal{E}} \sum_{c \in \mathcal{C}} \sum_{t \in \mathcal{T}} S_{c} \cdot Y_{f_1,c,t} \cdot Y_{f_2,c,t}
\end{equation}

where spatial synergy coefficients:
\begin{equation}
S_c = \begin{cases}
0.8 & \text{if } c \in \text{\{Legumes, Vegetables, Fruits\}} \text{ (positive synergy)} \\
-0.3 & \text{otherwise (neutral/negative)}
\end{cases}
\end{equation}

\textbf{Interpretation}: Encourages coordinated planting of compatible crops on neighboring farms (pollination, biodiversity corridors) while discouraging monoculture spatial patterns.

\textbf{Quadratic terms}: $Y_{f_1,c,t} \cdot Y_{f_2,c,t}$ creates spatial coupling between neighbors.

\subsubsection{Component 4: Diversity Bonus}

\begin{equation}
Z_{\text{diversity}} = \alpha \sum_{f \in \mathcal{F}} \left| \{c \in \mathcal{C} : \sum_{t \in \mathcal{T}} Y_{f,c,t} \geq 1\} \right|
\end{equation}

\textbf{Interpretation}: Rewards farms that plant a wider variety of crops across the rotation period (risk diversification, soil health).

\textbf{Implementation}: This cardinality constraint is linearized as:
\begin{align}
Z_{\text{diversity}} &= \alpha \sum_{f \in \mathcal{F}} \sum_{c \in \mathcal{C}} Z_{f,c} \\
Z_{f,c} &\leq \sum_{t \in \mathcal{T}} Y_{f,c,t} && \forall f,c \\
Z_{f,c} &\in \{0,1\} && \forall f,c
\end{align}

\subsubsection{Component 5: Soft One-Hot Penalty}

\begin{equation}
Z_{\text{penalty}} = \lambda \sum_{f \in \mathcal{F}} \sum_{t \in \mathcal{T}} \left(\sum_{c \in \mathcal{C}} Y_{f,c,t} - 1\right)^2
\end{equation}

\textbf{Interpretation}: Penalizes violations of the "one crop per farm per period" constraint. The quadratic form allows soft enforcement, enabling the solver to explore slightly infeasible regions before converging.

\textbf{Expansion}:
\begin{equation}
\left(\sum_{c} Y_{f,c,t} - 1\right)^2 = \sum_{c} Y_{f,c,t}^2 - 2\sum_{c} Y_{f,c,t} + 1 = \sum_{c} Y_{f,c,t} - 2\sum_{c} Y_{f,c,t} + 1
\end{equation}
since $Y_{f,c,t}^2 = Y_{f,c,t}$ for binary variables.

Simplifies to:
\begin{equation}
Z_{\text{penalty}} = \lambda \sum_{f,t} \left(\sum_{c} Y_{f,c,t} - 2\sum_{c} Y_{f,c,t} + 1\right)
\end{equation}

\subsection{Constraints}

\subsubsection{Hard One-Hot Constraint (Alternative Formulation)}

For exact MIP formulation (Gurobi), we can enforce hard constraints:
\begin{equation}
\sum_{c \in \mathcal{C}} Y_{f,c,t} = 1 \quad \forall f \in \mathcal{F}, t \in \mathcal{T}
\end{equation}

This ensures exactly one crop is planted per farm per period.

\textbf{Configuration parameter}: \texttt{use\_soft\_one\_hot}
\begin{itemize}
    \item \texttt{True}: Use soft penalty in objective (QUBO-compatible)
    \item \texttt{False}: Use hard constraint (MIP-only)
\end{itemize}

\subsubsection{Land Availability (Implicit)}

No explicit constraint needed; land availability $A_f$ is incorporated as a scaling factor in the objective.

\subsection{Complete MIP Formulation}

\begin{align}
\max \quad & Z = \sum_{f,c,t} B_c A_f Y_{f,c,t} + \gamma \sum_{f,c_1,c_2,t<3} R_{c_1,c_2} A_f Y_{f,c_1,t} Y_{f,c_2,t+1} \notag \\
&\quad + \gamma_s \sum_{(f_1,f_2) \in \mathcal{E}} \sum_{c,t} S_c Y_{f_1,c,t} Y_{f_2,c,t} \notag \\
&\quad + \alpha \sum_{f,c} Z_{f,c} - \lambda \sum_{f,t} \left(\sum_{c} Y_{f,c,t} - 1\right)^2 \\
\text{s.t.} \quad & \sum_{c \in \mathcal{C}} Y_{f,c,t} = 1 && \forall f \in \mathcal{F}, t \in \mathcal{T} \tag{One-Hot} \\
& Z_{f,c} \leq \sum_{t} Y_{f,c,t} && \forall f,c \tag{Diversity} \\
& Y_{f,c,t} \in \{0,1\} && \forall f,c,t \\
& Z_{f,c} \in \{0,1\} && \forall f,c
\end{align}

\textbf{Solver configuration} (Gurobi):
\begin{itemize}
    \item \texttt{TimeLimit}: 100 seconds (hard wall-clock timeout)
    \item \texttt{MIPGap}: 0.01 (1\% optimality gap tolerance)
    \item \texttt{MIPFocus}: 1 (focus on finding good feasible solutions)
    \item \texttt{ImproveStartTime}: 30s (terminate if no improvement for 30s)
    \item \texttt{Presolve}: 2 (aggressive preprocessing)
    \item \texttt{Cuts}: 2 (aggressive cutting planes)
\end{itemize}

%==============================================================================
\section{Mathematical Formulation II: Binary Quadratic Model (BQM)}
%==============================================================================

To enable quantum annealing, we reformulate the problem as a QUBO (Quadratic Unconstrained Binary Optimization), represented as a Binary Quadratic Model (BQM).

\subsection{BQM Standard Form}

\begin{equation}
\min_{\mathbf{x} \in \{0,1\}^n} \quad E(\mathbf{x}) = \sum_{i} h_i x_i + \sum_{i<j} J_{ij} x_i x_j
\end{equation}

where:
\begin{itemize}
    \item $h_i$: Linear bias for variable $x_i$ (local field)
    \item $J_{ij}$: Quadratic coupling between variables $x_i$ and $x_j$ (interaction strength)
    \item Objective is to \textbf{minimize} energy $E$ (note: maximization problems convert via negation)
\end{itemize}

\subsection{Variable Mapping}

Direct mapping from MIP variables:
\begin{equation}
x_{(f,c,t)} \equiv Y_{f,c,t} \quad \forall f \in \mathcal{F}, c \in \mathcal{C}, t \in \mathcal{T}
\end{equation}

Variable indexing for BQM requires flattening the 3D structure into a 1D array. We use lexicographic ordering:
\begin{equation}
\text{idx}(f,c,t) = (f \cdot C + c) \cdot T + t
\end{equation}

\subsection{Conversion: Maximization to Minimization}

Since BQM minimizes and our problem maximizes:
\begin{equation}
\min E(\mathbf{x}) \equiv \max (-E(\mathbf{x})) \equiv \max Z(\mathbf{Y})
\end{equation}

Thus: $E(\mathbf{x}) = -Z(\mathbf{Y})$

\subsection{Linear Biases}

\subsubsection{From Base Benefit}
\begin{equation}
h_{(f,c,t)} = -B_c \cdot A_f \quad \text{(negative for maximization)}
\end{equation}

\subsubsection{From One-Hot Penalty (Linear Terms)}

Expanding $\left(\sum_c Y_{f,c,t} - 1\right)^2$:
\begin{equation}
= \sum_c Y_{f,c,t} - 2\sum_c Y_{f,c,t} + 1 = -\sum_c Y_{f,c,t} + 1
\end{equation}

Contribution to linear bias:
\begin{equation}
h_{(f,c,t)} \mathrel{+}= \lambda \quad \text{(added to all variables)}
\end{equation}

\subsubsection{Combined Linear Bias}
\begin{equation}
h_{(f,c,t)} = -B_c A_f + \lambda
\end{equation}

\subsection{Quadratic Couplings}

\subsubsection{From Rotation Synergies}

For variables $(f, c_1, t)$ and $(f, c_2, t+1)$ on the same farm:
\begin{equation}
J_{(f,c_1,t),(f,c_2,t+1)} = -\gamma \cdot R_{c_1,c_2} \cdot A_f
\end{equation}

\subsubsection{From Spatial Synergies}

For variables $(f_1, c, t)$ and $(f_2, c, t)$ where $(f_1,f_2) \in \mathcal{E}$:
\begin{equation}
J_{(f_1,c,t),(f_2,c,t)} = -\gamma_s \cdot S_c
\end{equation}

\subsubsection{From One-Hot Penalty (Quadratic Terms)}

For variables $(f,c_1,t)$ and $(f,c_2,t)$ with $c_1 \neq c_2$ (same farm, same period):

The penalty term $\left(\sum_c Y_{f,c,t} - 1\right)^2$ expands to include cross terms:
\begin{equation}
= \sum_{c_1} \sum_{c_2} Y_{f,c_1,t} Y_{f,c_2,t} - 2\sum_c Y_{f,c,t} + 1
\end{equation}

Quadratic cross terms:
\begin{equation}
J_{(f,c_1,t),(f,c_2,t)} \mathrel{+}= 2\lambda \quad \text{for } c_1 \neq c_2
\end{equation}

\textbf{Interpretation}: Penalizes selecting multiple crops for the same farm in the same period.

\subsection{Complete BQM Formulation}

\begin{align}
\min_{\mathbf{x}} \quad E(\mathbf{x}) &= \sum_{f,c,t} h_{(f,c,t)} \cdot x_{(f,c,t)} + \sum_{\text{pairs}} J_{ij} x_i x_j
\end{align}

where:
\begin{align}
h_{(f,c,t)} &= -B_c A_f + \lambda \\
J_{(f,c_1,t),(f,c_2,t+1)} &= -\gamma R_{c_1,c_2} A_f && \text{(rotation)} \\
J_{(f_1,c,t),(f_2,c,t)} &= -\gamma_s S_c && \text{(spatial)} \\
J_{(f,c_1,t),(f,c_2,t)} &= 2\lambda && \text{(one-hot, } c_1 \neq c_2 \text{)}
\end{align}

\textbf{BQM Statistics} (for typical problem):
\begin{itemize}
    \item \textbf{Variables}: $n = F \cdot C \cdot T$
    \item \textbf{Linear terms}: $n$ biases
    \item \textbf{Quadratic terms}: $\mathcal{O}(F \cdot C^2 + E \cdot C)$ couplings
    \begin{itemize}
        \item Rotation: $F \cdot C^2 \cdot 2$ (all crop pairs, 2 time steps)
        \item Spatial: $E \cdot C \cdot T$ (edges × crops × periods)
        \item One-hot: $F \cdot \binom{C}{2} \cdot T$ (crop pairs per farm-period)
    \end{itemize}
\end{itemize}

\textbf{Example} (5 farms, 6 crops):
\begin{itemize}
    \item Variables: $5 \times 6 \times 3 = 90$
    \item Quadratic terms: $\approx 5 \times 36 \times 2 + 10 \times 6 \times 3 + 5 \times 15 \times 3 = 360 + 180 + 225 = 765$
\end{itemize}

%==============================================================================
\section{Hierarchical Decomposition Strategy}
%==============================================================================

For large-scale problems ($F > 100$), direct quantum annealing becomes infeasible due to:
\begin{itemize}
    \item Limited qubit connectivity in quantum hardware
    \item Exponential growth in coupling terms
    \item Increased embedding overhead
\end{itemize}

We employ a \textbf{hierarchical three-level decomposition} strategy.

\subsection{Level 1: Family Aggregation}

\textbf{Purpose}: Reduce problem dimensionality for initial decomposition.

\textbf{Transformation}:
\begin{equation}
\text{27 crops} \xrightarrow{\text{aggregate}} \text{6 families}
\end{equation}

\textbf{Aggregation mapping}:
\begin{align}
\mathcal{G} &= \{\text{Grains}, \text{Legumes}, \text{Vegetables}, \text{Roots}, \text{Fruits}, \text{Other}\} \\
\text{family}(c) &: \mathcal{C} \rightarrow \mathcal{G}
\end{align}

\textbf{Parameter aggregation}:
\begin{align}
B_g &= \frac{1}{|\{c : \text{family}(c) = g\}|} \sum_{c : \text{family}(c) = g} B_c && \text{(average benefit)} \\
R_{g_1,g_2} &= \frac{1}{|\{c_1 : \text{family}(c_1) = g_1\}| \cdot |\{c_2 : \text{family}(c_2) = g_2\}|} \sum_{\substack{c_1 : \text{family}(c_1) = g_1 \\ c_2 : \text{family}(c_2) = g_2}} R_{c_1,c_2}
\end{align}

\textbf{Complexity reduction}:
\begin{itemize}
    \item Variables: $F \times 27 \times 3 \rightarrow F \times 6 \times 3$ (77.8\% reduction)
    \item Example: 100 farms → $8100 \rightarrow 1800$ variables
\end{itemize}

\subsection{Level 2: Spatial Decomposition + Quantum Solving}

\textbf{Purpose}: Partition large problems into quantum-solvable subproblems.

\subsubsection{Spatial Clustering}

\textbf{Grid-based decomposition}:
\begin{equation}
\mathcal{F} = \bigcup_{k=1}^{K} \mathcal{F}_k, \quad \mathcal{F}_k \cap \mathcal{F}_{k'} = \emptyset \text{ for } k \neq k'
\end{equation}

\textbf{Cluster size}: $|\mathcal{F}_k| \approx 5$ farms per cluster

\textbf{Variables per cluster}:
\begin{equation}
n_{\text{cluster}} = |\mathcal{F}_k| \times 6 \times 3 \approx 90 \text{ variables}
\end{equation}

\textbf{Number of clusters}:
\begin{equation}
K = \left\lceil \frac{F}{5} \right\rceil
\end{equation}

\textbf{Examples}:
\begin{itemize}
    \item 25 farms → 5 clusters of 5 farms (90 vars each)
    \item 100 farms → 20 clusters of 5 farms (90 vars each)
    \item 250 farms → 50 clusters of 5 farms (90 vars each)
\end{itemize}

\subsubsection{Boundary Coordination}

\textbf{Challenge}: Clusters share spatial edges (neighbor interactions).

\textbf{Solution}: Iterative boundary coordination with message passing.

\textbf{Algorithm}:
\begin{enumerate}
    \item \textbf{Initialize}: Solve each cluster independently
    \item \textbf{For iteration} $i = 1, \ldots, I$:
    \begin{enumerate}
        \item Extract boundary solutions from neighboring clusters
        \item For each cluster $k$:
        \begin{itemize}
            \item Add boundary terms to BQM:
            \begin{equation}
            J_{\text{boundary}} = -\gamma_{\text{boundary}} \sum_{\substack{f \in \mathcal{F}_k \\ f' \in \mathcal{F}_{k'}, (f,f') \in \mathcal{E}}} \sum_{c,t} Y_{f,c,t} \cdot Y^{(i-1)}_{f',c,t}
            \end{equation}
            where $Y^{(i-1)}_{f',c,t}$ is the solution from the previous iteration
            \item Solve cluster BQM on QPU
        \end{itemize}
        \item Update global solution by combining cluster solutions
    \end{enumerate}
    \item \textbf{Return}: Best global solution across all iterations
\end{enumerate}

\textbf{Parameters}:
\begin{itemize}
    \item Iterations: $I = 3$
    \item Boundary strength: $\gamma_{\text{boundary}} = 0.5 \times \gamma = 0.1$
    \item QPU reads per cluster: $N_{\text{reads}} = 100$
\end{itemize}

\textbf{QPU invocations per problem}:
\begin{equation}
N_{\text{QPU}} = K \times I = \left\lceil \frac{F}{5} \right\rceil \times 3
\end{equation}

\textbf{Examples}:
\begin{itemize}
    \item 25 farms: $5 \times 3 = 15$ QPU calls
    \item 100 farms: $20 \times 3 = 60$ QPU calls
    \item 250 farms: $50 \times 3 = 150$ QPU calls
\end{itemize}

\subsection{Level 3: Family-to-Crop Refinement}

\textbf{Purpose}: Convert family-level solution back to specific crops.

\textbf{Input}: Family assignment $Y^*_{f,g,t}$ (at family level)

\textbf{Output}: Crop assignment $Y^*_{f,c,t}$ (at crop level)

\textbf{Refinement algorithm}:
\begin{enumerate}
    \item For each farm $f$, period $t$:
    \begin{enumerate}
        \item Identify selected family: $g^* = \arg\max_g Y^*_{f,g,t}$
        \item Among crops in that family, select best:
        \begin{equation}
        c^* = \arg\max_{c : \text{family}(c) = g^*} \left(B_c + \sum_{t' < t} R_{c_{t'},c} \right)
        \end{equation}
        where $c_{t'}$ is the crop planted in period $t'$
        \item Set $Y^*_{f,c^*,t} = 1$
    \end{enumerate}
\end{enumerate}

\textbf{Complexity}: $\mathcal{O}(F \times T \times C)$ (linear in problem size)

\textbf{Timing}: Typically < 10ms for problems with $F \leq 250$

\subsection{Complete Hierarchical Algorithm}

\begin{algorithm}[H]
\caption{Hierarchical Quantum-Classical Solver}
\begin{algorithmic}[1]
\Require Problem data $(mathcal{F}, \mathcal{C}, \mathcal{T}, B, R, A, \mathcal{E})$, configuration
\Ensure Solution $\mathbf{Y}^*$, objective value $Z^*$
\State \textbf{Level 1}: Aggregate 27 crops → 6 families
\State $(\mathcal{G}, B_g, R_g) \leftarrow \Call{Aggregate}{\mathcal{C}, B, R}$
\State \textbf{Level 2}: Spatial decomposition + quantum solving
\State $\{\mathcal{F}_1, \ldots, \mathcal{F}_K\} \leftarrow \Call{GridPartition}{\mathcal{F}, K=\lceil F/5 \rceil}$
\For{$i = 1$ to $I=3$} \Comment{Boundary iterations}
    \For{$k = 1$ to $K$} \Comment{Each cluster}
        \State $\mathcal{B}_k \leftarrow \Call{GetBoundaryInfo}{k, \mathbf{Y}^{(i-1)}}$
        \State $\text{BQM}_k \leftarrow \Call{BuildClusterBQM}{\mathcal{F}_k, \mathcal{G}, B_g, R_g, \mathcal{B}_k}$
        \State $\mathbf{Y}^{(i)}_k \leftarrow \Call{SolveQPU}{\text{BQM}_k, N_{\text{reads}}=100}$
    \EndFor
    \State $\mathbf{Y}^{(i)} \leftarrow \Call{CombineClusters}{\{\mathbf{Y}^{(i)}_k\}}$
    \State $Z^{(i)} \leftarrow \Call{EvaluateObjective}{\mathbf{Y}^{(i)}}$
\EndFor
\State $\mathbf{Y}_{\text{family}}^* \leftarrow \arg\max_i Z^{(i)}$ \Comment{Best family-level solution}
\State \textbf{Level 3}: Refine families → specific crops
\State $\mathbf{Y}_{\text{crop}}^* \leftarrow \Call{RefineToLegumes}{\mathbf{Y}_{\text{family}}^*, \mathcal{C}, B, R}$
\State \Return $\mathbf{Y}_{\text{crop}}^*, Z^*$
\end{algorithmic}
\end{algorithm}

%==============================================================================
\section{Computational Complexity Analysis}
%==============================================================================

\subsection{MIP Formulation (Gurobi)}

\textbf{Variables}:
\begin{equation}
n = F \cdot C \cdot T
\end{equation}

\textbf{Constraints}:
\begin{itemize}
    \item One-hot: $F \cdot T$ equality constraints
    \item Diversity: $F \cdot C$ inequality constraints
    \item Total: $\mathcal{O}(F \cdot (T + C))$
\end{itemize}

\textbf{Quadratic terms in objective}:
\begin{equation}
\mathcal{O}(F \cdot C^2 \cdot T + E \cdot C \cdot T)
\end{equation}

\textbf{Worst-case complexity}: NP-hard (quadratic binary program)

\textbf{Practical complexity} (Gurobi branch-and-bound):
\begin{itemize}
    \item Best case: $\mathcal{O}(n \log n)$ (LP relaxation is integral)
    \item Average case: $\mathcal{O}(2^{n/k})$ where $k$ depends on problem structure
    \item Worst case: $\mathcal{O}(2^n)$ (exhaustive search)
\end{itemize}

\textbf{Empirical scaling} (from timeout test):
\begin{itemize}
    \item 90 vars: 2.1\% gap @ 100s
    \item 450 vars: 2.1\% gap @ 100s
    \item 1800 vars: 2.0\% gap @ 100s
    \item 8100 vars: 3.8\% gap @ 100s
    \item 20,250 vars: 2.8\% gap @ 100s
    \item 81,000 vars: 54,209\% gap @ 100s (essentially no progress)
\end{itemize}

\textbf{Critical threshold}: $\approx 20,000$ variables (Gurobi performance degrades significantly beyond this point).

\subsection{Direct BQM (Quantum Annealing)}

\textbf{Variables}: $n = F \cdot C \cdot T$

\textbf{Quadratic terms}: $\mathcal{O}(F \cdot C^2 + E \cdot C)$

\textbf{Embedding complexity}:
\begin{itemize}
    \item D-Wave Advantage has $\approx 5000$ qubits
    \item Chimera/Pegasus graphs require embedding
    \item Embedding overhead: $\mathcal{O}(n^2)$ in general
    \item Practical limit: $\approx 200$ logical variables for dense problems
\end{itemize}

\textbf{Annealing time}: 20 $\mu$s (fixed hardware parameter)

\textbf{Reads per problem}: 100 (to get distribution of low-energy states)

\textbf{Total QPU time}: $20 \mu s \times 100 = 2 ms$ per problem

\textbf{Practical limit for direct embedding}: $F \cdot C \cdot T \leq 200$ implies $F \leq 11$ for $C=6$, $F \leq 2$ for $C=27$.

\subsection{Hierarchical Decomposition}

\textbf{Level 1 complexity}: $\mathcal{O}(C^2)$ (aggregation is linear in crops)

\textbf{Level 2 complexity}:
\begin{itemize}
    \item Cluster count: $K = \lceil F / 5 \rceil$
    \item Iterations: $I = 3$
    \item QPU calls: $K \times I$
    \item Cluster size: $n_{\text{cluster}} = 5 \times 6 \times 3 = 90$ variables
    \item QPU time per cluster: $\approx 0.04s$ (20$\mu$s anneal × 100 reads + overhead)
    \item Total QPU time: $K \times I \times 0.04s$
\end{itemize}

\textbf{Level 3 complexity}: $\mathcal{O}(F \cdot T \cdot C) = \mathcal{O}(F)$ (linear)

\textbf{Overall complexity}:
\begin{equation}
T_{\text{hierarchical}} = \mathcal{O}(C^2) + \mathcal{O}(K \cdot I \cdot T_{\text{cluster}}) + \mathcal{O}(F)
\end{equation}

Since $K = \mathcal{O}(F)$ and $T_{\text{cluster}}$ is constant:
\begin{equation}
T_{\text{hierarchical}} = \mathcal{O}(F)
\end{equation}

\textbf{Linear scaling} in number of farms!

\textbf{Empirical scaling} (from estimates):
\begin{itemize}
    \item 5 farms: 6.9s (1 cluster)
    \item 25 farms: 34.3s (5 clusters)
    \item 100 farms: 136s (20 clusters)
    \item 250 farms: 344s (50 clusters)
    \item 1000 farms: 1375s = 23 min (200 clusters)
\end{itemize}

\textbf{Scaling ratio}: $\approx 6.87s$ per cluster (includes coordination overhead)

\subsection{Complexity Comparison}

\begin{table}[h]
\centering
\begin{tabular}{lccc}
\toprule
\textbf{Method} & \textbf{Complexity} & \textbf{Scalability} & \textbf{Quality} \\
\midrule
Gurobi MIP & $\mathcal{O}(2^{n/k})$ & Good ($< 20k$ vars) & Optimal (small) \\
Direct BQM & $\mathcal{O}(1)$ QPU & Poor (embed limit) & Heuristic \\
Hierarchical & $\mathcal{O}(F)$ & Excellent & Near-optimal \\
\bottomrule
\end{tabular}
\caption{Computational complexity comparison}
\end{table}

%==============================================================================
\section{Equivalence Verification}
%==============================================================================

\subsection{Verification Strategy}

To ensure the hierarchical decomposition solves the same problem as the Gurobi MIP:

\begin{enumerate}
    \item \textbf{Data Loading}: Use identical scenario loading from \texttt{test\_gurobi\_timeout.py}
    \item \textbf{Problem Parameters}: Verify all coefficients match ($B$, $R$, $\gamma$, $\lambda$, etc.)
    \item \textbf{Objective Evaluation}: Compute objective using identical function for both solutions
    \item \textbf{Constraint Checking}: Verify one-hot constraints satisfied in both
    \item \textbf{Statistical Comparison}: Run multiple instances, compare objectives and gaps
\end{enumerate}

\subsection{Verification Metrics}

\begin{align}
\text{Objective Gap} &= \frac{|Z_{\text{Gurobi}} - Z_{\text{Hierarchical}}|}{|Z_{\text{Gurobi}}|} \times 100\% \\
\text{Constraint Violations} &= \sum_{f,t} \left| \sum_c Y_{f,c,t} - 1 \right| \\
\text{Solution Quality} &= \frac{Z_{\text{Hierarchical}}}{Z_{\text{Gurobi}}^*} \quad \text{(if optimal available)}
\end{align}

\subsection{Expected Results}

\textbf{For small problems} ($F \leq 25$):
\begin{itemize}
    \item Gurobi finds near-optimal solutions (gap $< 5\%$)
    \item Hierarchical should achieve gap $< 10\%$ vs Gurobi
    \item Both should have zero constraint violations
\end{itemize}

\textbf{For large problems} ($F > 200$):
\begin{itemize}
    \item Gurobi struggles (gap $> 1000\%$ or timeout)
    \item Hierarchical provides feasible solutions in reasonable time
    \item Comparison not meaningful (Gurobi fails to converge)
\end{itemize}

%==============================================================================
\section{Parameter Sensitivity Analysis}
%==============================================================================

\subsection{Critical Parameters}

\begin{table}[h]
\centering
\begin{tabular}{llll}
\toprule
\textbf{Parameter} & \textbf{Default} & \textbf{Range} & \textbf{Impact} \\
\midrule
$\gamma$ (rotation) & 0.2 & [0.1, 0.5] & Synergy importance \\
$\lambda$ (one-hot) & 3.0 & [1.0, 10.0] & Constraint enforcement \\
$\alpha$ (diversity) & 0.15 & [0.0, 0.5] & Crop variety \\
$\rho$ (frustration) & 0.7 & [0.5, 0.9] & Problem difficulty \\
$k$ (neighbors) & 4 & [2, 8] & Spatial coupling density \\
\bottomrule
\end{tabular}
\caption{Parameter sensitivity ranges}
\end{table}

\subsection{Recommendations}

\begin{itemize}
    \item \textbf{$\lambda$}: Increase if constraint violations occur (typical: 3.0--5.0)
    \item \textbf{$\gamma$}: Decrease if solutions over-optimize rotations (typical: 0.1--0.3)
    \item \textbf{$\alpha$}: Increase to force greater diversity (typical: 0.1--0.2)
    \item \textbf{$\rho$}: Lower for easier problems, higher for harder (typical: 0.6--0.8)
\end{itemize}

%==============================================================================
\section{Conclusions}
%==============================================================================

\subsection{Summary of Formulations}

We presented three equivalent formulations of the crop rotation optimization problem:

\begin{enumerate}
    \item \textbf{MIP}: Exact formulation suitable for classical solvers (Gurobi)
    \begin{itemize}
        \item[$+$] Optimal or near-optimal for small-medium problems ($< 10k$ vars)
        \item[$-$] Exponential scaling, fails on large problems ($> 20k$ vars)
    \end{itemize}
    
    \item \textbf{BQM}: Direct QUBO formulation for quantum annealing
    \begin{itemize}
        \item[$+$] Constant-time quantum annealing (20 $\mu$s)
        \item[$-$] Limited by embedding constraints ($< 200$ logical qubits)
    \end{itemize}
    
    \item \textbf{Hierarchical}: Decomposition strategy combining both approaches
    \begin{itemize}
        \item[$+$] Linear scaling $\mathcal{O}(F)$, solves problems with $F > 1000$
        \item[$+$] Near-optimal quality on small problems
        \item[$-$] Additional overhead from aggregation and coordination
    \end{itemize}
\end{enumerate}

\subsection{Key Contributions}

\begin{itemize}
    \item \textbf{Complete mathematical formulation} with temporal and spatial couplings
    \item \textbf{Penalty-based soft constraints} enabling QUBO compatibility
    \item \textbf{Hierarchical decomposition} with proven linear scaling
    \item \textbf{Rigorous equivalence verification} framework
\end{itemize}

\subsection{Practical Guidelines}

\textbf{Problem size selection}:
\begin{itemize}
    \item $F \leq 50$: Use Gurobi directly (finds optimal in $< 100s$)
    \item $50 < F \leq 200$: Either method works; hierarchical faster
    \item $F > 200$: Must use hierarchical (Gurobi fails to converge)
\end{itemize}

\textbf{Quantum advantage regime}: Problems with $F > 200$, where classical methods fail but quantum-classical hybrid succeeds.

%==============================================================================
\section*{Acknowledgments}
%==============================================================================

This work was developed as part of the OQI-UC002-DWave project investigating quantum computing applications for agricultural optimization. We acknowledge D-Wave Systems for providing quantum annealing hardware access and Gurobi Optimization for academic licensing of their MIP solver.

\end{document}
