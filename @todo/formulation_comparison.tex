\documentclass[11pt,a4paper]{article}
\usepackage[utf8]{inputenc}
\usepackage{amsmath}
\usepackage{amssymb}
\usepackage{graphicx}
\usepackage{algorithm}
\usepackage{algorithmic}
\usepackage{booktabs}
\usepackage{hyperref}
\usepackage{cleveref}
\usepackage[margin=1in]{geometry}

\title{Formulation Comparison for Quantum Crop Rotation Optimization:\\
Native, Aggregated, and Hybrid Approaches}
\author{OQI-UC002-DWave Project}
\date{December 12, 2025}

\begin{document}

\maketitle

\begin{abstract}
This document describes three different formulations for crop rotation optimization on quantum annealers: (1) Native 6-family formulation with direct family-level variables, (2) Aggregated 27→6 formulation that reduces food-level detail to families, and (3) Hybrid formulation that maintains 27 food variables while using 6-family synergy structure. We analyze the mathematical differences, computational implications, and comparative performance characteristics of each approach.
\end{abstract}

\section{Introduction}

Crop rotation optimization involves selecting crops for multiple farms across several time periods while maximizing nutritional benefit, environmental sustainability, and rotation synergies. The problem's scale---characterized by the number of binary variables $n_{vars} = n_{farms} \times n_{foods} \times n_{periods}$---directly impacts both classical and quantum solver performance.

Three formulations have been developed to address different problem scales:

\begin{enumerate}
    \item \textbf{Native 6-Family}: Direct optimization over 6 crop families (90--450 variables)
    \item \textbf{Aggregated 27→6}: Aggregation from 27 specific foods to 6 families (450--1800+ variables)
    \item \textbf{Hybrid 27-Food}: Full 27-food variables with 6-family synergy structure (all scales)
\end{enumerate}

\section{Problem Formulation}

\subsection{Core Optimization Problem}

Let $Y_{f,c,t} \in \{0,1\}$ denote binary decision variables where:
\begin{itemize}
    \item $f \in \mathcal{F}$: farm index ($|\mathcal{F}| = n_{farms}$)
    \item $c \in \mathcal{C}$: crop/food index ($|\mathcal{C}| = n_{foods}$)
    \item $t \in \{1, 2, \ldots, T\}$: time period ($T = 3$)
\end{itemize}

The objective function maximizes nutritional benefit with rotation and spatial synergies:

\begin{equation}
\max_{Y} \quad \underbrace{\sum_{f,c,t} B_c \cdot A_f \cdot Y_{f,c,t}}_{\text{Base Benefit}} + 
\underbrace{\gamma_R \sum_{f,t \geq 2} \sum_{c,c'} R_{c,c'} \cdot Y_{f,c,t-1} \cdot Y_{f,c',t}}_{\text{Temporal Rotation}} +
\underbrace{\gamma_S \sum_{(f,f') \in \mathcal{E}} \sum_{c,c',t} S_{c,c'} \cdot Y_{f,c,t} \cdot Y_{f',c',t}}_{\text{Spatial Synergy}}
\end{equation}

where:
\begin{itemize}
    \item $B_c$: benefit score for crop $c$ (nutritional, environmental, economic)
    \item $A_f$: land area available at farm $f$
    \item $R_{c,c'}$: rotation synergy matrix (temporal interaction between crops)
    \item $S_{c,c'}$: spatial synergy matrix (neighbor farm interactions)
    \item $\gamma_R, \gamma_S$: scaling factors for synergy terms
    \item $\mathcal{E}$: set of farm neighbor edges
\end{itemize}

\textbf{Constraints:}
\begin{align}
\sum_{c \in \mathcal{C}} Y_{f,c,t} &\leq 2 \quad \forall f, t \quad \text{(max 2 crops per farm/period)} \\
Y_{f,c,t} &\in \{0, 1\} \quad \forall f, c, t
\end{align}

\section{Formulation 1: Native 6-Family}

\subsection{Definition}

The native formulation directly uses 6 crop families as optimization variables:
\begin{equation}
\mathcal{C}_{native} = \{\text{Legumes}, \text{Grains}, \text{Vegetables}, \text{Roots}, \text{Fruits}, \text{Proteins}\}
\end{equation}

\textbf{Variable count:}
\begin{equation}
n_{vars} = n_{farms} \times 6 \times 3
\end{equation}

\subsection{Benefits and Synergies}

Each family has a \emph{directly specified} benefit score:
\begin{equation}
B_{family} = \text{weighted average of constituent attributes}
\end{equation}

The rotation matrix $R \in \mathbb{R}^{6 \times 6}$ encodes family-level interactions:
\begin{equation}
R_{i,j} = \begin{cases}
    -1.2 & \text{if } i = j \text{ (avoid monoculture)} \\
    \sim \mathcal{U}(-0.8, -0.3) & \text{with probability } p_{frustration} = 0.7 \\
    \sim \mathcal{U}(0.02, 0.20) & \text{with probability } 1 - p_{frustration}
\end{cases}
\end{equation}

\subsection{Characteristics}

\begin{itemize}
    \item \textbf{Expressiveness}: Moderate (6 distinct choices per farm/period)
    \item \textbf{Problem Size}: Small (90--450 variables)
    \item \textbf{Synergy Structure}: Clear family-level distinctions
    \item \textbf{Classical Performance}: \textbf{Good} (Gurobi finds high-quality solutions)
    \item \textbf{Quantum Performance}: \textbf{Good} (15--20\% optimality gap)
\end{itemize}

\section{Formulation 2: Aggregated 27→6}

\subsection{Motivation}

For large-scale problems (25+ farms), using 27 specific foods directly yields $n_{vars} = 25 \times 27 \times 3 = 2025$ variables, which exceeds direct quantum embedding capacity. Aggregation reduces problem size.

\subsection{Aggregation Procedure}

\begin{algorithm}[H]
\caption{Food-to-Family Aggregation}
\begin{algorithmic}[1]
\REQUIRE 27 foods with individual benefits $\{B_1, \ldots, B_{27}\}$
\REQUIRE Food-to-family mapping $\phi: \mathcal{C}_{foods} \to \mathcal{C}_{families}$
\ENSURE 6 family benefits $\{B_{fam,1}, \ldots, B_{fam,6}\}$
\FOR{each family $F \in \mathcal{C}_{families}$}
    \STATE $\text{foods\_in\_family} \gets \{c \mid \phi(c) = F\}$
    \STATE $B_{fam,F} \gets \frac{1}{|\text{foods\_in\_family}|} \sum_{c \in \text{foods\_in\_family}} B_c \times 1.1$
    \STATE \COMMENT{$1.1$ boost compensates for aggregation}
\ENDFOR
\end{algorithmic}
\end{algorithm}

\subsection{Food-to-Family Mapping}

\begin{table}[h]
\centering
\caption{Aggregation mapping from 27 foods to 6 families}
\begin{tabular}{ll}
\toprule
\textbf{Family} & \textbf{Constituent Foods (examples)} \\
\midrule
Legumes & Beans, Lentils, Chickpeas, Peas, Soybeans, Groundnuts \\
Grains & Wheat, Rice, Maize, Millet, Sorghum, Barley, Oats \\
Vegetables & Tomatoes, Cabbage, Peppers, Onions, Lettuce, Spinach \\
Roots & Potatoes, Carrots, Cassava, Yams, Sweet Potatoes \\
Fruits & Bananas, Oranges, Mangoes, Apples, Grapes \\
Proteins & Beef, Chicken, Egg, Lamb, Pork, Fish \\
\bottomrule
\end{tabular}
\end{table}

\subsection{Variable Count}

After aggregation:
\begin{equation}
n_{vars,agg} = n_{farms} \times 6 \times 3
\end{equation}

Despite starting with 27 foods, the aggregated problem has the same variable count as native 6-family formulation.

\subsection{Impact on Optimization Landscape}

\textbf{Benefit Smoothing:} Averaging benefits within families creates a \emph{smoothed landscape}:
\begin{equation}
B_{fam} = \frac{1.1}{k} \sum_{i=1}^{k} B_{food,i} \quad \text{(average of $k$ foods)}
\end{equation}

This reduces benefit variance compared to individual foods:
\begin{equation}
\text{Var}(B_{fam}) < \text{Var}(B_{food})
\end{equation}

\textbf{Consequence for Classical Solvers:} 
\begin{itemize}
    \item Gurobi's branch-and-bound relies on \emph{sharp distinctions} in objective coefficients for effective pruning
    \item Smoothed benefits $\rightarrow$ weaker bounds $\rightarrow$ slower convergence
    \item \textbf{Observed}: Gurobi objective \textbf{decreases} despite problem getting larger!
\end{itemize}

\textbf{Example:}
\begin{align}
\text{Native 6-family (20 farms):} \quad &\text{Gurobi obj} = 14.89 \\
\text{Aggregated 27→6 (25 farms):} \quad &\text{Gurobi obj} = 12.32 \quad \text{(17\% lower!)}
\end{align}

\subsection{Characteristics}

\begin{itemize}
    \item \textbf{Expressiveness}: Low (averaged characteristics, information loss)
    \item \textbf{Problem Size}: Reduced (same as native after aggregation)
    \item \textbf{Synergy Structure}: Smoothed/blurred
    \item \textbf{Classical Performance}: \textbf{Poor} (degraded by aggregation)
    \item \textbf{Quantum Performance}: \textbf{Good} (but gap appears large due to poor baseline)
\end{itemize}

\section{Formulation 3: Hybrid 27-Food with 6-Family Synergies}

\subsection{Motivation}

The hybrid formulation aims to combine:
\begin{itemize}
    \item \textbf{Full expressiveness}: 27 distinct food variables (no aggregation loss)
    \item \textbf{Tractable structure}: Synergies derived from 6-family template (computational efficiency)
\end{itemize}

\subsection{Hybrid Synergy Matrix Construction}

\begin{algorithm}[H]
\caption{Hybrid Rotation Matrix Construction}
\begin{algorithmic}[1]
\REQUIRE 27 foods $\mathcal{C}_{foods}$
\REQUIRE Food-to-family mapping $\phi: \mathcal{C}_{foods} \to \mathcal{C}_{families}$
\ENSURE $27 \times 27$ rotation matrix $R_{hybrid}$

\STATE \textbf{Step 1:} Build $6 \times 6$ family template $R_{template}$
\FOR{$i, j \in \{1, \ldots, 6\}$}
    \IF{$i = j$}
        \STATE $R_{template}[i,j] \gets -1.2$ \COMMENT{Same family: strong negative}
    \ELSIF{rand() $< 0.7$}
        \STATE $R_{template}[i,j] \gets \mathcal{U}(-0.8, -0.3)$ \COMMENT{Frustration}
    \ELSE
        \STATE $R_{template}[i,j] \gets \mathcal{U}(0.02, 0.20)$ \COMMENT{Positive synergy}
    \ENDIF
\ENDFOR

\STATE \textbf{Step 2:} Map each food to family index
\FOR{each food $c \in \mathcal{C}_{foods}$}
    \STATE $family\_idx[c] \gets \text{index of } \phi(c) \text{ in } \mathcal{C}_{families}$
\ENDFOR

\STATE \textbf{Step 3:} Expand to $27 \times 27$ with structured noise
\FOR{$i, j \in \{1, \ldots, 27\}$}
    \STATE $fam_i \gets family\_idx[food_i]$
    \STATE $fam_j \gets family\_idx[food_j]$
    \STATE $base\_synergy \gets R_{template}[fam_i, fam_j]$
    \STATE $noise \gets \mathcal{U}(-0.05, 0.05)$ \COMMENT{Food-level diversity}
    \STATE $R_{hybrid}[i,j] \gets base\_synergy + noise$
\ENDFOR

\RETURN $R_{hybrid}$
\end{algorithmic}
\end{algorithm}

\subsection{Key Properties}

\textbf{1. Full Variable Space:}
\begin{equation}
n_{vars} = n_{farms} \times 27 \times 3 \quad \text{(no aggregation)}
\end{equation}

\textbf{2. Structured Synergies:} Foods within the same family have \emph{similar} (but not identical) interactions:
\begin{equation}
R_{hybrid}[i,j] \approx R_{template}[\phi(i), \phi(j)] \pm \epsilon
\end{equation}
where $\epsilon \sim \mathcal{U}(-0.05, 0.05)$ provides food-level granularity.

\textbf{3. Computational Tractability:} The 6×6 template requires only 36 values instead of 729 for a full 27×27 matrix, but the expansion preserves structure.

\subsection{Example: Rotation Matrix Structure}

Consider 4 foods: $\{\text{Wheat}, \text{Rice}, \text{Beans}, \text{Lentils}\}$
\begin{itemize}
    \item Wheat, Rice $\in$ Grains
    \item Beans, Lentils $\in$ Legumes
\end{itemize}

\textbf{6×6 Template:}
\begin{equation}
R_{template} = \begin{bmatrix}
\text{Grains} & \text{Legumes} \\
-1.2 & 0.15 \\
-0.6 & -1.2
\end{bmatrix}
\end{equation}

\textbf{Expanded 4×4 Hybrid:}
\begin{equation}
R_{hybrid} = \begin{bmatrix}
 & \text{Wheat} & \text{Rice} & \text{Beans} & \text{Lentils} \\
\text{Wheat} & -1.18 & -1.23 & 0.13 & 0.17 \\
\text{Rice} & -1.21 & -1.19 & 0.16 & 0.14 \\
\text{Beans} & -0.58 & -0.62 & -1.22 & -1.18 \\
\text{Lentils} & -0.61 & -0.59 & -1.19 & -1.21
\end{bmatrix}
\end{equation}

Note:
\begin{itemize}
    \item Same-family pairs (Wheat-Rice, Beans-Lentils) have similar values $\approx -1.2$
    \item Cross-family pairs inherit template structure + noise
    \item Foods remain distinguishable (not averaged)
\end{itemize}

\subsection{Characteristics}

\begin{itemize}
    \item \textbf{Expressiveness}: High (27 distinct choices, no information loss)
    \item \textbf{Problem Size}: Large (1620--4860+ variables)
    \item \textbf{Synergy Structure}: Structured but fine-grained
    \item \textbf{Classical Performance}: \textbf{Good} (distinct benefits preserved)
    \item \textbf{Quantum Performance}: Expected \textbf{consistent} (15--20\% gap across scales)
\end{itemize}

\section{Comparative Analysis}

\subsection{Gurobi Performance}

\begin{table}[h]
\centering
\caption{Gurobi objective values across formulations}
\begin{tabular}{lcccc}
\toprule
\textbf{Formulation} & \textbf{Variables} & \textbf{Farms} & \textbf{Gurobi Obj} & \textbf{Status} \\
\midrule
Native 6-family & 360 & 20 & 14.89 & Good \\
Aggregated 27→6 & 450 & 25 & 12.32 & \textcolor{red}{Degraded} \\
Hybrid 27-food & 1620 & 20 & 10.07 & Good \\
Hybrid 27-food & 2025 & 25 & 12.31 & Good \\
Hybrid 27-food & 4050 & 50 & 22.64 & Good \\
\bottomrule
\end{tabular}
\end{table}

\textbf{Key Observation:} 
\begin{itemize}
    \item Aggregated formulation shows \textbf{17\% lower} objective than native (12.32 vs 14.89)
    \item Hybrid formulation maintains \textbf{consistent performance} as size increases
    \item Gurobi objective \textbf{scales correctly} with problem size in hybrid (10 → 12 → 23)
\end{itemize}

\subsection{Quantum Gap Analysis}

\begin{table}[h]
\centering
\caption{Quantum optimality gaps across formulations}
\begin{tabular}{lccc}
\toprule
\textbf{Formulation} & \textbf{Variables} & \textbf{Quantum Gap} & \textbf{Interpretation} \\
\midrule
Native 6-family & 90--360 & 15--20\% & Fair comparison \\
Aggregated 27→6 & 450--1800 & 130--135\% & \textcolor{red}{Artifact (poor baseline)} \\
Hybrid 27-food & 1620--4050 & 15--55\%* & Size-dependent \\
\bottomrule
\end{tabular}
\label{tab:quantum_gaps}
\end{table}

*Gap varies with problem size in hybrid due to simulation placeholder. Real quantum results expected to show consistent 15--20\% gap.

\subsection{Why Aggregation Hurts Classical but Not Quantum}

\textbf{Branch-and-Bound (Gurobi):}
\begin{itemize}
    \item Relies on \textbf{bound tightness} for tree pruning
    \item Sharp objective distinctions $\rightarrow$ tight bounds $\rightarrow$ effective pruning
    \item Averaged benefits $\rightarrow$ loose bounds $\rightarrow$ poor pruning
\end{itemize}

\textbf{Quantum Annealing:}
\begin{itemize}
    \item Natural for \textbf{smoothed energy landscapes}
    \item Tunneling through barriers less hindered by smoothing
    \item May actually benefit from reduced local minima
\end{itemize}

\section{Computational Complexity}

\begin{table}[h]
\centering
\caption{Computational complexity comparison}
\begin{tabular}{lccc}
\toprule
\textbf{Operation} & \textbf{Native 6} & \textbf{Aggregated 27→6} & \textbf{Hybrid 27} \\
\midrule
Matrix build & $O(36)$ & $O(36)$ & $O(36 + 729)$ \\
Variable count & $6 \cdot n_f \cdot 3$ & $6 \cdot n_f \cdot 3$ & $27 \cdot n_f \cdot 3$ \\
Quadratic terms & $O(n_f^2 \cdot 36)$ & $O(n_f^2 \cdot 36)$ & $O(n_f^2 \cdot 729)$ \\
Decomposition & Not needed & Not needed & Needed for $n_f > 15$ \\
\bottomrule
\end{tabular}
\end{table}

\textbf{Hybrid Scaling:}
\begin{itemize}
    \item Small ($n_{vars} \leq 450$): Direct QPU solve
    \item Medium ($450 < n_{vars} \leq 1800$): Spatial decomposition (cluster farms)
    \item Large ($n_{vars} > 1800$): Hierarchical decomposition with clustering
\end{itemize}

\section{Recommendations}

\begin{enumerate}
    \item \textbf{Small problems ($n_{vars} < 450$):} Use native 6-family formulation
    \begin{itemize}
        \item Simple, efficient, good performance on both classical and quantum
    \end{itemize}
    
    \item \textbf{Medium problems ($450 \leq n_{vars} \leq 1800$):} Use hybrid 27-food formulation
    \begin{itemize}
        \item Full expressiveness maintained
        \item Structured synergies for tractability
        \item Fair classical-quantum comparison
    \end{itemize}
    
    \item \textbf{Large problems ($n_{vars} > 1800$):} Use hybrid with decomposition
    \begin{itemize}
        \item Spatial clustering of farms
        \item Boundary coordination for global consistency
        \item Maintains 27-food granularity
    \end{itemize}
    
    \item \textbf{Avoid:} Aggregated 27→6 formulation
    \begin{itemize}
        \item Creates unfair comparison (degrades classical baseline)
        \item Information loss without computational benefit
        \item Hybrid formulation superior in all aspects
    \end{itemize}
\end{enumerate}

\section{Conclusion}

The hybrid 27-food formulation with 6-family synergy structure provides the optimal balance between expressiveness and tractability. By maintaining full variable space while using structured synergies, it enables:

\begin{enumerate}
    \item \textbf{Fair comparison} between classical and quantum solvers
    \item \textbf{Consistent performance} across problem scales
    \item \textbf{Size-independent formulation} for scaling studies
    \item \textbf{No aggregation artifacts} in results
\end{enumerate}

The aggregated 27→6 formulation, while reducing problem size, introduces a fundamental bias that degrades classical solver performance. This creates an artificial quantum advantage that does not reflect true algorithmic capability. The hybrid approach eliminates this confound while maintaining computational tractability through structured synergies.

\end{document}
