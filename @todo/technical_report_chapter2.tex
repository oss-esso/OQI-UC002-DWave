% Chapter 2: Mathematical Problem Formulation

\chapter{Mathematical Problem Formulation}

This chapter provides a rigorous mathematical formulation of the agricultural resource allocation optimization problem. We present both the continuous (farm-based) and binary (patch-based) formulations, detail the constraint structures, and specify the multi-objective function.

\section{Problem Statement}

\subsection{Overview}

Given a set of agricultural units (farms or patches) with specified land areas and a set of potential crops with associated characteristics, determine the optimal allocation of crops to land units such that:

\begin{itemize}
    \item All land availability constraints are satisfied
    \item Minimum planting area requirements are met
    \item Food group diversity constraints are enforced
    \item The multi-objective utility function is maximized
\end{itemize}

\subsection{Notation and Variable Definitions}

\begin{table}[H]
\centering
\caption{Problem Notation and Definitions}
\label{tab:notation}
\begin{tabular}{ll}
\toprule
\textbf{Symbol} & \textbf{Description} \\
\midrule
$\mathcal{F}$ & Set of farms/patches, $|\mathcal{F}| = 25$ \\
$\mathcal{C}$ & Set of crops, $|\mathcal{C}| = 27$ \\
$\mathcal{G}$ & Set of food groups \\
$L_f$ & Land availability for farm $f \in \mathcal{F}$ (hectares) \\
$M_c$ & Minimum planting area for crop $c \in \mathcal{C}$ (hectares) \\
$M^{\text{max}}_c$ & Maximum planting area for crop $c \in \mathcal{C}$ (hectares) \\
$G(c)$ & Food group assignment function: $\mathcal{C} \rightarrow \mathcal{G}$ \\
$N_c$ & Nutritional value of crop $c$ (score) \\
$D_c$ & Nutrient density of crop $c$ (score) \\
$E_c$ & Environmental impact of crop $c$ (negative score) \\
$\$_c$ & Affordability of crop $c$ (score) \\
$S_c$ & Sustainability of crop $c$ (score) \\
$w_1, \ldots, w_5$ & Objective function weights \\
\bottomrule
\end{tabular}
\end{table}

\section{Continuous Formulation (Farm Scenario)}

\subsection{Decision Variables}

For the continuous formulation with farms, we define:

\begin{itemize}
    \item $A_{f,c} \in \mathbb{R}_{\geq 0}$: Area (hectares) of crop $c$ planted on farm $f$
    \item $Y_{f,c} \in \{0,1\}$: Binary indicator whether crop $c$ is planted on farm $f$
\end{itemize}

The total number of variables is:
\begin{equation}
|\text{Vars}| = |\mathcal{F}| \times |\mathcal{C}| \times 2 = 25 \times 27 \times 2 = 1350 \text{ variables}
\end{equation}

comprising 675 continuous variables ($A_{f,c}$) and 675 binary variables ($Y_{f,c}$).

\subsection{Constraints}

\subsubsection{Land Availability Constraints}

For each farm $f \in \mathcal{F}$, the total allocated land cannot exceed available land:

\begin{equation}
\sum_{c \in \mathcal{C}} A_{f,c} \leq L_f \quad \forall f \in \mathcal{F}
\label{eq:land_availability}
\end{equation}

This yields 25 constraints (one per farm).

\subsubsection{Minimum Planting Area Constraints}

If crop $c$ is planted on farm $f$ (i.e., $Y_{f,c} = 1$), the planted area must exceed the minimum:

\begin{equation}
A_{f,c} \geq M_c \cdot Y_{f,c} \quad \forall f \in \mathcal{F}, \forall c \in \mathcal{C}
\label{eq:min_planting}
\end{equation}

This yields $25 \times 27 = 675$ constraints.

\subsubsection{Food Group Diversity Constraints}

For each food group $g \in \mathcal{G}$, enforce minimum and maximum diversity requirements based on \textbf{selection counts} (not area):

\begin{equation}
\alpha^{\text{min}}_g \leq \sum_{f \in \mathcal{F}} \sum_{c \in G^{-1}(g)} Y_{f,c} \leq \alpha^{\text{max}}_g \quad \forall g \in \mathcal{G}
\label{eq:food_group_farm}
\end{equation}

where $G^{-1}(g)$ denotes crops in food group $g$, and $\alpha^{\text{min}}_g$, $\alpha^{\text{max}}_g$ specify minimum and maximum number of different foods from group $g$.

\textbf{Important}: These constraints count \textit{selections} ($Y$ variables), not total area allocated. This ensures diversity in crop variety selection.

With 5 food groups and 2 constraints per group (min and max), this yields 10 constraints.

\subsubsection{Maximum Planting Area Constraints}

If crop $c$ is planted on farm $f$, the planted area cannot exceed an explicit maximum (when specified):

\begin{equation}
A_{f,c} \leq M^{\text{max}}_c \cdot Y_{f,c} \quad \forall f \in \mathcal{F}, \forall c \in \mathcal{C} \quad \text{(when } M^{\text{max}}_c \text{ defined)}
\label{eq:max_planting}
\end{equation}

For crops without an explicit maximum, the farm capacity provides the upper bound:

\begin{equation}
A_{f,c} \leq L_f \cdot Y_{f,c} \quad \forall f \in \mathcal{F}, \forall c \in \mathcal{C} \quad \text{(when } M^{\text{max}}_c \text{ undefined)}
\label{eq:linking}
\end{equation}

These linking and maximum constraints total $25 \times 27 = 675$ constraints and ensure that $Y_{f,c} = 0 \implies A_{f,c} = 0$.

\subsubsection{(Revised) Food Group Constraints}

In the corrected formulation we enforce food-group diversity by counting crop \emph{selections} rather than area. For each food group $g \in \mathcal{G}$ we require:

\begin{equation}
\alpha^{\text{min}}_g \leq \sum_{f \in \mathcal{F}} \sum_{c \in G^{-1}(g)} Y_{f,c} \leq \alpha^{\text{max}}_g \quad \forall g \in \mathcal{G}
\label{eq:food_group_count}
\end{equation}

Here $G^{-1}(g)$ denotes the set of crops in food group $g$, and $\alpha^{\text{min}}_g$, $\alpha^{\text{max}}_g$ are integer bounds on the number of different foods from group $g$ to be selected across all farms. Counting selections (binary variables) enforces variety without constraining area allocation directly.

\subsection{Objective Function}

The multi-objective function maximizes weighted benefits:

\begin{equation}
\begin{aligned}
\text{maximize} \quad & \frac{1}{\sum_{f} L_f} \sum_{f \in \mathcal{F}} \sum_{c \in \mathcal{C}} A_{f,c} \cdot B_c \\
\text{where} \quad B_c = & \, w_1 N_c + w_2 D_c - w_3 E_c + w_4 \$_c + w_5 S_c
\end{aligned}
\label{eq:objective_continuous}
\end{equation}

The normalization by total land ($\sum_f L_f = 100$ hectares) ensures scale-independent objective values.

\subsection{Complete Continuous Formulation}

The complete mixed-integer nonlinear program (MINLP) is:

\begin{equation}
\begin{aligned}
\max \quad & \frac{1}{100} \sum_{f=1}^{25} \sum_{c=1}^{27} A_{f,c} \cdot B_c \\
\text{s.t.} \quad & \sum_{c=1}^{27} A_{f,c} \leq L_f & \forall f \in \{1, \ldots, 25\} \\
& A_{f,c} \geq M_c \cdot Y_{f,c} & \forall f,c \\
& A_{f,c} \leq L_f \cdot Y_{f,c} & \forall f,c \\
& \alpha^{\text{min}}_g \leq \sum_{f,c: G(c)=g} Y_{f,c} \leq \alpha^{\text{max}}_g & \forall g \in \mathcal{G} \\
& A_{f,c} \geq 0 & \forall f,c \\
& Y_{f,c} \in \{0,1\} & \forall f,c
\end{aligned}
\label{eq:minlp_complete}
\end{equation}

\section{Binary Formulation (Patch Scenario)}

\subsection{Decision Variables}

For the binary formulation with patches, we use only binary selection variables:

\begin{equation}
Y_{p,c} \in \{0,1\} \quad \text{indicates whether crop $c$ is planted on patch $p$}
\end{equation}

The total number of variables is:
\begin{equation}
|\text{Vars}| = |\mathcal{P}| \times |\mathcal{C}| = 25 \times 27 = 675 \text{ binary variables}
\end{equation}

Each patch has a fixed area $s_p$, and if crop $c$ is selected for patch $p$, the entire patch area is allocated to that crop.

\subsection{Constraints}

\subsubsection{One-Crop-Per-Patch Constraint}

Each patch can be allocated to at most one crop:

\begin{equation}
\sum_{c \in \mathcal{C}} Y_{p,c} \leq 1 \quad \forall p \in \mathcal{P}
\label{eq:one_crop}
\end{equation}

This yields 25 constraints (one per patch).

\subsubsection{Food Group Constraints}

Food group diversity in the patch scenario is enforced via selection counts (not aggregate area):

\begin{equation}
\alpha^{\text{min}}_g \leq \sum_{p \in \mathcal{P}} \sum_{c: G(c)=g} Y_{p,c} \leq \alpha^{\text{max}}_g \quad \forall g \in \mathcal{G}
\label{eq:food_group_binary}
\end{equation}

\subsection{Objective Function}

The binary objective maximizes total benefit across all patches:

\begin{equation}
\text{maximize} \quad \frac{1}{\sum_p s_p} \sum_{p \in \mathcal{P}} \sum_{c \in \mathcal{C}} s_p \cdot Y_{p,c} \cdot B_c
\label{eq:objective_binary}
\end{equation}

\subsection{Complete Binary Formulation}

The complete binary integer program (BIP) is:

\begin{equation}
\begin{aligned}
\max \quad & \frac{1}{100} \sum_{p=1}^{25} \sum_{c=1}^{27} s_p \cdot Y_{p,c} \cdot B_c \\
\text{s.t.} \quad & \sum_{c=1}^{27} Y_{p,c} \leq 1 & \forall p \in \{1, \ldots, 25\} \\
& \alpha^{\text{min}}_g \leq \sum_{p,c: G(c)=g} Y_{p,c} \leq \alpha^{\text{max}}_g & \forall g \in \mathcal{G} \\
& Y_{p,c} \in \{0,1\} & \forall p,c
\end{aligned}
\label{eq:bip_complete}
\end{equation}

\section{Constrained Quadratic Model (CQM) Representation}

Both formulations are encoded as Constrained Quadratic Models (CQMs) for submission to D-Wave solvers.

\subsection{CQM Variable Types}

The D-Wave CQM framework supports:

\begin{itemize}
    \item \texttt{Real} variables: $A_{f,c} \in [0, L_f]$
    \item \texttt{Binary} variables: $Y_{f,c} \in \{0,1\}$
\end{itemize}

\subsection{CQM Constraint Encoding}

Constraints are encoded with symbolic labels and sense specifications:

\begin{lstlisting}[caption={CQM Constraint Encoding Example},label={lst:cqm_constraints}]
from dimod import ConstrainedQuadraticModel, Real, Binary

cqm = ConstrainedQuadraticModel()

# Land availability constraint for farm f
constraint_expr = sum(A[f,c] for c in crops) <= L[f]
cqm.add_constraint(constraint_expr, 
                   label=f"land_avail_farm_{f}")

# Minimum planting constraint
constraint_expr = A[f,c] >= M[c] * Y[f,c]
cqm.add_constraint(constraint_expr,
                   label=f"min_plant_f{f}_c{c}")
\end{lstlisting}

\subsection{CQM to BQM Conversion}

For quantum annealing on D-Wave QPUs, CQMs must be converted to Binary Quadratic Models (BQMs):

\begin{definition}[Binary Quadratic Model]
A BQM is defined by:
\begin{equation}
E(\mathbf{x}) = \sum_{i} h_i x_i + \sum_{i<j} J_{ij} x_i x_j + c
\end{equation}
where $x_i \in \{0,1\}$, $h_i$ are linear coefficients, $J_{ij}$ are quadratic coefficients, and $c$ is an offset constant.
\end{definition}

The conversion process involves:

\begin{enumerate}
    \item \textbf{Discretization}: Real variables are discretized into binary representations
    \item \textbf{Constraint Relaxation}: Hard constraints are converted to penalty terms
    \item \textbf{Quadratization}: Higher-order terms are reduced to quadratic form
    \item \textbf{Penalty Tuning}: Penalty coefficients are scaled to ensure constraint satisfaction
\end{enumerate}

The conversion is performed automatically via:

\begin{lstlisting}[caption={CQM to BQM Conversion},label={lst:cqm_to_bqm}]
from dimod import cqm_to_bqm

bqm, invert = cqm_to_bqm(cqm)
# bqm: BinaryQuadraticModel for quantum annealing
# invert: Function to map BQM solution back to CQM space
\end{lstlisting}

\section{Problem Complexity Analysis}

\subsection{Continuous Formulation Complexity}

The continuous MINLP has:

\begin{itemize}
    \item \textbf{Variables}: 1350 (675 continuous + 675 binary)
    \item \textbf{Constraints}: $\approx$ 1385 (25 land + 675 min planting + 675 max/linking + 10 food group bounds)
    \item \textbf{Complexity Class}: NP-hard (due to binary variables)
    \item \textbf{Solver Type}: Mixed-integer nonlinear programming (MINLP)
\end{itemize}

\subsection{Binary Formulation Complexity}

The binary BIP has:

\begin{itemize}
    \item \textbf{Variables}: 675 binary
    \item \textbf{Constraints}: $\approx$ 30-35 (25 one-crop-per-patch + food groups)
    \item \textbf{Complexity Class}: NP-hard (binary integer programming)
    \item \textbf{Solver Type}: Binary integer programming (BIP) / QUBO
\end{itemize}

\subsection{Solution Space Size}

The theoretical solution space sizes are:

\begin{equation}
\begin{aligned}
\text{Continuous:} \quad & |\Omega| \approx 2^{675} \times \mathbb{R}^{675} \\
\text{Binary:} \quad & |\Omega| = 2^{675} \approx 10^{203}
\end{aligned}
\end{equation}

Exhaustive enumeration is computationally intractable, necessitating advanced optimization techniques.

\section{Multi-Objective Optimization Framework}

\subsection{Objective Components}

The composite objective function balances five criteria:

\begin{table}[H]
\centering
\caption{Multi-Objective Components}
\label{tab:objectives}
\begin{tabular}{lll}
\toprule
\textbf{Component} & \textbf{Symbol} & \textbf{Interpretation} \\
\midrule
Nutritional Value & $N_c$ & Caloric and macronutrient content \\
Nutrient Density & $D_c$ & Micronutrient concentration \\
Environmental Impact & $E_c$ & Carbon footprint, water usage (minimize) \\
Affordability & $\$_c$ & Economic accessibility \\
Sustainability & $S_c$ & Long-term viability \\
\bottomrule
\end{tabular}
\end{table}

\subsection{Weight Configuration}

Default weight configuration for the full\_family scenario:

\begin{equation}
\mathbf{w} = [w_1, w_2, w_3, w_4, w_5]^T
\end{equation}

Weights are normalized such that $\sum_{i=1}^5 w_i = 1.0$ and are configured based on policy priorities.

\subsection{Pareto Optimality}

The weighted sum approach yields a single solution on the Pareto frontier. Alternative weight configurations explore different trade-offs:

\begin{definition}[Pareto Optimal Solution]
A solution $\mathbf{x}^*$ is Pareto optimal if there exists no other feasible solution $\mathbf{x}$ such that $O_i(\mathbf{x}) \geq O_i(\mathbf{x}^*)$ for all $i$ and $O_j(\mathbf{x}) > O_j(\mathbf{x}^*)$ for at least one $j$.
\end{definition}

\section{Constraint Satisfaction Verification}

\subsection{Feasibility Checking}

A solution is feasible if and only if all constraints are satisfied within numerical tolerance $\epsilon = 10^{-6}$:

\begin{equation}
\begin{aligned}
\text{Feasible} \iff & \sum_c A_{f,c} \leq L_f + \epsilon \quad \forall f \\
& A_{f,c} \geq M_c \cdot Y_{f,c} - \epsilon \quad \forall f,c \\
& A_{f,c} \leq L_f \cdot Y_{f,c} + \epsilon \quad \forall f,c \\
& \sum_{f,c: G(c)=g} Y_{f,c} \geq \alpha^{\text{min}}_g - \epsilon \quad \forall g \\
& \sum_{f,c: G(c)=g} Y_{f,c} \leq \alpha^{\text{max}}_g + \epsilon \quad \forall g
\end{aligned}
\end{equation}

\subsection{Constraint Violation Metrics}

For infeasible solutions, we measure violation magnitude:

\begin{equation}
V(\mathbf{x}) = \sum_{i \in \mathcal{I}} \max(0, g_i(\mathbf{x})) + \sum_{j \in \mathcal{E}} |h_j(\mathbf{x})|
\end{equation}

where $g_i$ are inequality constraints and $h_j$ are equality constraints.
