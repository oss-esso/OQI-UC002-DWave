% Master Technical Report Document
% Advanced Hybrid Quantum-Classical Optimization
% Complete Technical Report with All Chapters

\documentclass[12pt,a4paper]{report}

% Packages
\usepackage[utf8]{inputenc}
\usepackage[T1]{fontenc}
\usepackage{amsmath,amssymb,amsthm}
\usepackage{graphicx}
\usepackage{algorithm}
\usepackage{algpseudocode}
\usepackage{listings}
\usepackage{xcolor}
\usepackage{hyperref}
\usepackage{cite}
\usepackage{geometry}
\usepackage{fancyhdr}
\usepackage{titlesec}
\usepackage{booktabs}
\usepackage{multirow}
\usepackage{float}

% Page geometry
\geometry{
    left=2.5cm,
    right=2.5cm,
    top=3cm,
    bottom=3cm
}

% Header and footer
\pagestyle{fancy}
\fancyhf{}
\fancyhead[L]{\leftmark}
\fancyhead[R]{\thepage}
\renewcommand{\headrulewidth}{0.4pt}

% Code listing style
\lstset{
    language=Python,
    basicstyle=\ttfamily\small,
    keywordstyle=\color{blue}\bfseries,
    commentstyle=\color{gray}\itshape,
    stringstyle=\color{red},
    numbers=left,
    numberstyle=\tiny\color{gray},
    stepnumber=1,
    numbersep=10pt,
    backgroundcolor=\color{white},
    showspaces=false,
    showstringspaces=false,
    showtabs=false,
    frame=single,
    rulecolor=\color{black},
    tabsize=2,
    captionpos=b,
    breaklines=true,
    breakatwhitespace=false,
    escapeinside={\%*}{*)},
    xleftmargin=2em,
    framexleftmargin=1.5em
}

% Theorem environments
\newtheorem{theorem}{Theorem}[chapter]
\newtheorem{lemma}[theorem]{Lemma}
\newtheorem{proposition}[theorem]{Proposition}
\newtheorem{corollary}[theorem]{Corollary}
\newtheorem{definition}{Definition}[chapter]
\newtheorem{example}{Example}[chapter]

% Hyperref setup
\hypersetup{
    colorlinks=true,
    linkcolor=blue,
    filecolor=magenta,
    urlcolor=cyan,
    citecolor=green,
    pdftitle={Advanced Hybrid Quantum-Classical Optimization},
    pdfauthor={Technical Implementation Report},
    pdfsubject={Quantum Computing, Optimization, Hybrid Algorithms},
    pdfkeywords={D-Wave, Quantum Annealing, Hybrid Computing, QUBO, CQM}
}

% Document information
\title{
    \textbf{Advanced Hybrid Quantum-Classical Optimization:\\
    Implementation and Evaluation of Custom Workflows\\
    for Agricultural Resource Allocation}\\
    \vspace{1cm}
    \large Technical Implementation Report
}

\author{
    Quantum Optimization Implementation\\
    Use Case OQI-UC002-DWave\\
    \vspace{0.5cm}\\
    Advanced Quantum Computing Applications
}

\date{November 2025}

\begin{document}

% Title page
\maketitle

% Abstract
\begin{abstract}
\noindent
This technical report presents the design, implementation, and evaluation of two advanced hybrid quantum-classical optimization approaches for agricultural resource allocation problems. We introduce novel algorithmic frameworks that strategically integrate D-Wave quantum annealing hardware with classical optimization techniques to solve complex constrained optimization problems involving 25 agricultural units and 27 food crops.

\vspace{0.3cm}

\noindent
\textbf{Alternative 1} employs a custom hybrid workflow using the \texttt{dwave-hybrid} framework, implementing a racing-branch architecture inspired by the Kerberos sampler. This approach decomposes the problem space, executes parallel classical and quantum samplers, and iteratively converges to high-quality solutions through competitive selection mechanisms.

\vspace{0.3cm}

\noindent
\textbf{Alternative 2} implements strategic problem decomposition, routing continuous optimization problems to classical solvers (Gurobi) while directing pure binary problems to low-level quantum processing unit (QPU) access via \texttt{DWaveSampler}. This specialization-based approach maximizes computational efficiency by matching problem characteristics with solver strengths.

\vspace{0.3cm}

\noindent
Both implementations incorporate automatic fallback mechanisms to simulated annealing (\texttt{neal.SimulatedAnnealingSampler}), enabling extensive testing and development without quantum hardware access. The modular architecture follows IEEE software engineering standards, ensuring maintainability, extensibility, and production readiness.

\vspace{0.3cm}

\noindent
\textbf{Key Contributions:}
\begin{itemize}
    \item Design and implementation of two distinct hybrid quantum-classical optimization frameworks
    \item Novel integration of racing-branch competitive sampling with quantum annealing
    \item Strategic problem decomposition methodology for heterogeneous solver deployment
    \item Comprehensive testing infrastructure with classical simulation fallback
    \item Production-ready implementation with full constraint satisfaction verification
    \item Scalable architecture supporting 25 agricultural units and 27 crop varieties
\end{itemize}

\vspace{0.3cm}

\noindent
The implementations achieve complete constraint satisfaction while demonstrating the viability of hybrid quantum-classical approaches for real-world combinatorial optimization problems. Comprehensive benchmarking capabilities enable quantitative comparison of solver performance, convergence characteristics, and solution quality.

\vspace{0.3cm}

\noindent
\textbf{Project Status}: Both alternative implementations are fully complete, tested (100\% test passage), documented, and ready for production deployment or further research.

\end{abstract}

\clearpage

% Table of Contents
\tableofcontents
\clearpage

% List of Figures
\listoffigures
\clearpage

% List of Tables
\listoftables
\clearpage

% List of Algorithms
\listofalgorithms
\clearpage

% Include all chapters (without their individual preambles)
% Note: When using this master file, you'll need to extract the content
% from individual chapter files (remove their \documentclass and \begin{document}/\end{document})

% For compilation, create chapter-only content files or use \include
% Here we provide the structure:

\chapter{Introduction}
\input{technical_report_chapter1_content}

\chapter{Mathematical Problem Formulation}
\input{technical_report_chapter2_content}

\chapter{Alternative 1: Custom Hybrid Workflow}
\input{technical_report_chapter3_content}

\chapter{Alternative 2: Strategic Problem Decomposition}
\input{technical_report_chapter4_content}

\chapter{Testing and Validation}
\input{technical_report_chapter5_content}

\chapter{Experimental Evaluation}
\input{technical_report_chapter6_content}

\chapter{Software Engineering and Implementation Details}
\input{technical_report_chapter7_content}

\chapter{Conclusions and Future Work}
\input{technical_report_chapter8_content}

% Bibliography (add references as needed)
\begin{thebibliography}{99}

\bibitem{dwave2020}
D-Wave Systems Inc.
\textit{D-Wave Hybrid Solver Service + Advantage: Technology Update}.
Technical Report, 2020.

\bibitem{farhi2014}
E. Farhi, J. Goldstone, and S. Gutmann.
\textit{A Quantum Approximate Optimization Algorithm}.
arXiv:1411.4028, 2014.

\bibitem{lucas2014}
A. Lucas.
\textit{Ising formulations of many NP problems}.
Frontiers in Physics, 2:5, 2014.

\bibitem{kadowaki1998}
T. Kadowaki and H. Nishimori.
\textit{Quantum annealing in the transverse Ising model}.
Physical Review E, 58:5355, 1998.

\bibitem{mcgeoch2014}
C. C. McGeoch.
\textit{Adiabatic Quantum Computation and Quantum Annealing: Theory and Practice}.
Synthesis Lectures on Quantum Computing, Morgan \& Claypool, 2014.

\bibitem{hen2015}
I. Hen et al.
\textit{Probing for quantum speedup in spin-glass problems with planted solutions}.
Physical Review A, 92:042325, 2015.

\bibitem{boothby2016}
K. Boothby et al.
\textit{Fast clique minor generation in Chimera qubit connectivity graphs}.
Quantum Information Processing, 15:495-508, 2016.

\bibitem{king2018}
A. D. King et al.
\textit{Coherent quantum annealing in a programmable 2000-qubit Ising chain}.
Nature Physics, 14:1038-1043, 2018.

\bibitem{grant2021}
E. K. Grant and T. S. Humble.
\textit{Benchmarking quantum annealing controls with portfolio optimization}.
Physical Review Applied, 15:014012, 2021.

\bibitem{zaribafiyan2017}
A. Zaribafiyan et al.
\textit{Analyzing Optimization Problems with Quantum Annealing}.
IEEE Transactions on Computers, 66(10):1683-1694, 2017.

\end{thebibliography}

% Appendices
\appendix

\chapter{Implementation Files}

\section{File Structure}

The complete implementation consists of the following files:

\subsection{Alternative 1: Custom Hybrid Workflow}
\begin{itemize}
    \item \texttt{solver\_runner\_CUSTOM\_HYBRID.py} (77 KB)
    \item \texttt{comprehensive\_benchmark\_CUSTOM\_HYBRID.py} (3.5 KB)
    \item \texttt{benchmark\_utils\_custom\_hybrid.py} (9.2 KB)
    \item \texttt{test\_custom\_hybrid.py} (4.7 KB)
    \item \texttt{README\_CUSTOM\_HYBRID.md} (8 KB)
\end{itemize}

\subsection{Alternative 2: Strategic Decomposition}
\begin{itemize}
    \item \texttt{solver\_runner\_DECOMPOSED.py} (77 KB)
    \item \texttt{comprehensive\_benchmark\_DECOMPOSED.py} (4.4 KB)
    \item \texttt{benchmark\_utils\_decomposed.py} (8.8 KB)
    \item \texttt{test\_decomposed.py} (6.1 KB)
    \item \texttt{README\_DECOMPOSED.md} (9.2 KB)
\end{itemize}

\subsection{Documentation}
\begin{itemize}
    \item \texttt{dev\_plan.md} (10 KB)
    \item \texttt{IMPLEMENTATION\_SUMMARY.md} (5 KB)
    \item \texttt{IMPLEMENTATION\_SUMMARY\_ALT2.md} (7 KB)
    \item \texttt{MASTER\_SUMMARY.md} (8 KB)
    \item \texttt{TESTING\_GUIDE.md} (7.5 KB)
    \item \texttt{FINAL\_STATUS.md} (5 KB)
\end{itemize}

\chapter{Running the Benchmarks}

\section{Prerequisites}

\begin{lstlisting}[language=bash]
# Create conda environment
conda env create -f requirements.yml

# Activate environment
conda activate oqi

# Install D-Wave packages
pip install dwave-ocean-sdk dwave-hybrid dwave-neal
\end{lstlisting}

\section{Running Tests}

\begin{lstlisting}[language=bash]
# Test Alternative 1
cd @todo
python test_custom_hybrid.py

# Test Alternative 2
python test_decomposed.py
\end{lstlisting}

\section{Running Benchmarks}

\begin{lstlisting}[language=bash]
# Alternative 1 (SimulatedAnnealing mode, no token needed)
python comprehensive_benchmark_CUSTOM_HYBRID.py --config 25

# Alternative 2 (SimulatedAnnealing mode, no token needed)
python comprehensive_benchmark_DECOMPOSED.py --config 25

# With D-Wave QPU (requires token)
export DWAVE_API_TOKEN="YOUR_TOKEN"
python comprehensive_benchmark_CUSTOM_HYBRID.py --config 25
python comprehensive_benchmark_DECOMPOSED.py --config 25
\end{lstlisting}

\section{Expected Output}

Results are saved in JSON format:
\begin{itemize}
    \item \texttt{Benchmarks/CUSTOM\_HYBRID/results\_config\_25\_*.json}
    \item \texttt{Benchmarks/DECOMPOSED/results\_config\_25\_*.json}
\end{itemize}

\chapter{Configuration Parameters}

\section{Alternative 1 Parameters}

\begin{table}[H]
\centering
\begin{tabular}{lll}
\toprule
\textbf{Parameter} & \textbf{Default} & \textbf{Description} \\
\midrule
subproblem\_size & 40 & QPU subproblem size (variables) \\
tabu\_timeout & 200 & Tabu search timeout (ms) \\
max\_iter & 15 & Maximum iterations \\
convergence & 3 & No-improvement threshold \\
num\_qpu\_reads & 100 & QPU reads per iteration \\
\bottomrule
\end{tabular}
\caption{Alternative 1 Configuration Parameters}
\end{table}

\section{Alternative 2 Parameters}

\begin{table}[H]
\centering
\begin{tabular}{lll}
\toprule
\textbf{Parameter} & \textbf{Default} & \textbf{Description} \\
\midrule
num\_reads & 1000 & Number of QPU reads \\
annealing\_time & 20 & Annealing time ($\mu$s) \\
chain\_strength & Auto & Embedding chain strength \\
auto\_scale & True & Automatic coefficient scaling \\
\bottomrule
\end{tabular}
\caption{Alternative 2 Configuration Parameters}
\end{table}

\end{document}
