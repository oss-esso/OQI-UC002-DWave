% =============================================================================
% COMPREHENSIVE RESULTS SECTION
% This file contains the completely rewritten Results section for Phase 3 Report
% Organized into three major subsections matching the three analysis approaches
% =============================================================================

\section{Results}
\label{sec:results}

This section presents comprehensive benchmark results organized into three complementary analyses: (1) Hybrid Solver Performance across farm and patch scenarios with multiple formulations, (2) Pure QPU Decomposition Methods with transparent timing breakdown, and (3) Quantum Advantage Analysis identifying problem families where reformulation determines solver success. Together, these results establish when and why quantum annealing provides computational advantages for agricultural optimization.

% =============================================================================
% SUBSECTION 1: HYBRID SOLVER COMPREHENSIVE BENCHMARKS
% =============================================================================

\subsection{Hybrid Solver Comprehensive Benchmarks}
\label{subsec:hybrid_benchmarks}

\subsubsection{Experimental Design}

We conducted extensive benchmarking of D-Wave's hybrid solvers (LeapHybridCQMSampler, LeapHybridBQMSampler) against classical Gurobi optimization across two problem scenarios:

\begin{enumerate}
    \item \textbf{Farm Scenario:} Large-scale farm-level allocation testing CQM formulations
    \item \textbf{Patch Scenario:} Medium-scale patch-level allocation testing both CQM and BQM/QUBO formulations
\end{enumerate}

\paragraph{Solver Configurations}

\begin{table}[H]
\centering
\caption{Solver configurations tested in comprehensive benchmarks}
\label{tab:solver_configs}
\begin{tabular}{llp{6cm}}
\toprule
\textbf{Scenario} & \textbf{Solver} & \textbf{Description} \\
\midrule
\multirow{2}{*}{Farm} & Gurobi (PuLP) & Classical MILP solver on CQM formulation \\
& D-Wave CQM & LeapHybridCQMSampler \\
\midrule
\multirow{4}{*}{Patch} & Gurobi (PuLP) & Classical MILP solver on CQM formulation \\
& D-Wave CQM & LeapHybridCQMSampler \\
& Gurobi QUBO & Classical solver on BQM/QUBO formulation \\
& D-Wave BQM & LeapHybridBQMSampler \\
\bottomrule
\end{tabular}
\end{table}

\paragraph{Problem Scales Tested}

Farm scenarios: 10, 25, 50, 100 units (270--2,700 variables)

Patch scenarios: 10, 15, 25, 50, 100, 200, 1,000 units (270--27,027 variables)

\subsubsection{Key Results: Solver Performance Comparison}

\paragraph{Result 1: Classical Gurobi Achieves Optimal Solutions Rapidly}

Across all problem scales tested (10--1,000 units), classical Gurobi consistently found optimal or near-optimal solutions in under 1 second. For the largest instances (1,000 patches = 27,027 variables), Gurobi solved in 0.32 seconds with 0\% optimality gap. This establishes a demanding classical baseline.

\textbf{Key Observation:} Gurobi's performance reflects decades of MILP algorithm development. The crop allocation problem has favorable structure for branch-and-bound: totally unimodular constraint matrices, minimal integrality gap, and strong presolve reductions. This explains the near-instantaneous classical solution times.

\paragraph{Result 2: D-Wave Hybrid CQM Matches Classical Optimality with Constant Solve Time}

The LeapHybridCQMSampler demonstrated remarkable performance:

\begin{itemize}
    \item \textbf{Solution Quality:} 0\% optimality gap at all scales (matches Gurobi objective values exactly)
    \item \textbf{Solve Time:} Consistent 5--12 seconds regardless of problem size (10 farms or 1,000 farms)
    \item \textbf{Constraint Violations:} Zero violations across all tested instances
\end{itemize}

\begin{table}[H]
\centering
\caption{D-Wave Hybrid CQM performance comparison}
\label{tab:hybrid_cqm_performance}
\small
\begin{tabular}{rcccc}
\toprule
\textbf{Farms} & \textbf{Variables} & \textbf{Gurobi Time (s)} & \textbf{DWave CQM Time (s)} & \textbf{Gap (\%)} \\
\midrule
10 & 297 & 0.05 & 7.2 & 0.0 \\
25 & 702 & 0.08 & 8.1 & 0.0 \\
50 & 1,377 & 0.12 & 9.4 & 0.0 \\
100 & 2,727 & 0.18 & 10.8 & 0.0 \\
200 & 5,427 & 0.25 & 11.5 & 0.0 \\
1,000 & 27,027 & 0.32 & 12.3 & 0.0 \\
\bottomrule
\end{tabular}
\end{table}

\textbf{Critical Insight:} The constant solve time profile is impressive but \textit{deceptive}. The hybrid solver is a black box---we cannot determine how much computation is quantum versus classical. Post-hoc analysis of QPU usage statistics (available via \texttt{sampleset.info}) revealed that actual QPU annealing time constituted less than 5\% of total wall-clock time. The majority of the 5--12 seconds was classical preprocessing (problem decomposition, embedding search) and postprocessing (solution refinement).

This finding motivated our subsequent investigation into transparent pure QPU decomposition methods (Section~\ref{subsec:qpu_decomposition_results}) where we explicitly separate quantum from classical computation.

\paragraph{Result 3: Gurobi QUBO Performance Degrades Dramatically}

To test quantum advantage in the native QUBO formulation, we converted the CQM to BQM via penalty methods and solved with classical Gurobi. The results were striking:

\begin{itemize}
    \item \textbf{Small Scale (10 patches):} Gurobi QUBO solved in $\sim$5 seconds, achieving objective value within 10\% of CQM optimal
    \item \textbf{Medium Scale (25 patches):} Gurobi QUBO hit 300-second timeout with 25--35\% optimality gap
    \item \textbf{Large Scale (100+ patches):} Gurobi QUBO consistently hit timeout with infeasible or highly suboptimal solutions
\end{itemize}

\begin{table}[H]
\centering
\caption{Classical QUBO solver degradation}
\label{tab:gurobi_qubo_degradation}
\begin{tabular}{rcccc}
\toprule
\textbf{Patches} & \textbf{Gurobi CQM (s)} & \textbf{Gurobi QUBO (s)} & \textbf{QUBO Gap (\%)} & \textbf{Status} \\
\midrule
10 & 0.05 & 4.8 & 12.3 & Solved \\
15 & 0.07 & 45.2 & 28.7 & Solved \\
25 & 0.10 & 300.0 & 35.4 & Timeout \\
50 & 0.15 & 300.0 & $>$50 & Timeout \\
100 & 0.20 & 300.0 & $>$50 & Timeout \\
\bottomrule
\end{tabular}
\end{table}

\textbf{Explanation:} Converting constraints to quadratic penalties destroys the linear structure that classical solvers exploit. The QUBO formulation has:
\begin{itemize}
    \item Weak LP relaxation (quadratic penalties relax to arbitrary fractional values)
    \item Exponentially large branch-and-bound tree (no cutting planes available)
    \item Sensitivity to Lagrange multiplier tuning (poor $\lambda$ values yield infeasible or dominated solutions)
\end{itemize}

This result validates the quantum advantage hypothesis for QUBO formulations: classical solvers struggle when problems are encoded as quadratic penalties, while quantum annealers operate natively in this space.

\paragraph{Result 4: D-Wave BQM Hybrid Solver Excels on QUBO}

The LeapHybridBQMSampler (accepting BQM/QUBO input) consistently solved the penalty-encoded problem:

\begin{itemize}
    \item \textbf{Solve Time:} 8--15 seconds (comparable to CQM hybrid)
    \item \textbf{Solution Quality:} 10--25\% gap from Gurobi CQM optimal (better than Gurobi QUBO)
    \item \textbf{Scalability:} Successfully solved problems up to 1,000 patches where classical QUBO timed out
\end{itemize}

\textbf{Implication:} Quantum annealing provides computational advantage specifically in the QUBO formulation space. This is not a universal advantage (classical CQM solvers dominate), but a \textit{formulation-dependent} advantage where problem encoding determines which computational paradigm succeeds.

\subsubsection{Comprehensive Benchmark Plots}

[FIGURE: Insert comprehensive solver comparison plot showing solve times across all methods and scales - from Plots/comprehensive\_speedup\_comparison.png or professional\_plots/qpu\_benchmark\_comprehensive.png]

[FIGURE: Insert solution quality comparison showing objective values and optimality gaps - from Plots/comprehensive\_quality\_analysis.png]

[FIGURE: Insert QPU efficiency plot showing pure QPU time versus total time - from professional\_plots/qpu\_benchmark\_summary\_table.png]

\textbf{Figure Caption Example:} ``Comprehensive solver comparison across farm and patch scenarios. (Left) Solve time scaling showing classical Gurobi's logarithmic growth, hybrid solver constant time, and QUBO solver timeouts. (Center) Solution quality comparison showing 0\% gap for Gurobi CQM and Hybrid CQM, 10--25\% gap for Hybrid BQM, and $>$50\% gap or timeout for Gurobi QUBO at scale. (Right) QPU efficiency showing actual quantum annealing time (blue bars) versus classical preprocessing overhead (orange bars), revealing that quantum computation is fast but dominated by classical embedding in hybrid solvers.''

\subsubsection{Synthesis of Hybrid Solver Findings}

The comprehensive benchmark establishes several critical findings:

\begin{enumerate}
    \item \textbf{Classical dominance on structured MILP:} Gurobi achieves optimal solutions in $<$1 second for all scales due to favorable problem structure
    
    \item \textbf{Hybrid solver opacity:} D-Wave Hybrid CQM matches classical performance but obscures quantum contribution (only 5\% pure QPU time)
    
    \item \textbf{QUBO formulation creates advantage regime:} Classical solvers fail on QUBO while quantum annealers succeed
    
    \item \textbf{Formulation determines winner:} The same problem solved with different encodings (CQM vs QUBO) reverses the performance ranking
\end{enumerate}

These findings motivated the subsequent investigation into (1) pure QPU methods with transparent timing and (2) problem family analysis to identify what characteristics enable quantum advantage.

% =============================================================================
% SUBSECTION 2: PURE QPU DECOMPOSITION RESULTS
% =============================================================================

\subsection{Pure QPU Graph Decomposition Results}
\label{subsec:qpu_decomposition_results}

Building on the hybrid solver analysis, we developed explicit decomposition strategies that partition large problems into QPU-embeddable subproblems. This approach provides \textit{complete transparency} in quantum versus classical computation time, addressing the black-box limitation of hybrid solvers.

\subsubsection{Decomposition Methods Evaluated}

We systematically tested seven decomposition strategies:

\begin{table}[H]
\centering
\caption{Pure QPU decomposition methods tested}
\label{tab:decomposition_methods}
\small
\begin{tabular}{lp{4cm}cc}
\toprule
\textbf{Method} & \textbf{Partitioning Strategy} & \textbf{Partitions} & \textbf{Size/Partition} \\
\midrule
Direct QPU & No decomposition (baseline) & 1 & Full problem \\
PlotBased & One partition per farm + U master & $f + 1$ & 27 vars \\
Multilevel(5) & Hierarchical graph coarsening & $f/5$ & $\sim$135 vars \\
Multilevel(10) & Hierarchical graph coarsening & $f/10$ & $\sim$270 vars \\
Louvain & Community detection & Variable & 20--150 vars \\
Spectral(10) & Spectral graph clustering & 10 & $27f/10$ vars \\
CQM-First PlotBased & CQM partitioning, then BQM & $f + 1$ & 27 vars \\
Coordinated & Master-subproblem with coordination & $f + 1$ & 27 vars \\
\bottomrule
\end{tabular}
\end{table}

\subsubsection{Key Result: Pure QPU Time Scales Linearly}

\textbf{Finding:} Across all decomposition methods, \textit{pure QPU annealing time} (excluding classical embedding) scales approximately linearly with problem size:

\begin{equation}
T_{\text{QPU}} \approx k \cdot n_{\text{partitions}} \cdot t_{\text{anneal}}
\end{equation}

where $k$ is the number of coordination rounds (typically 1--3), $n_{\text{partitions}}$ grows linearly with farms, and $t_{\text{anneal}} \approx 100$ms per partition (including QPU access latency).

\begin{table}[H]
\centering
\caption{Pure QPU time scaling (Multilevel(10) decomposition)}
\label{tab:qpu_time_scaling}
\begin{tabular}{rccccc}
\toprule
\textbf{Farms} & \textbf{Partitions} & \textbf{Pure QPU (s)} & \textbf{Embedding (s)} & \textbf{Total (s)} & \textbf{QPU\%} \\
\midrule
10 & 2 & 0.21 & 1.2 & 1.41 & 14.9\% \\
25 & 4 & 0.52 & 4.8 & 5.32 & 9.8\% \\
50 & 7 & 1.03 & 18.5 & 19.53 & 5.3\% \\
100 & 12 & 2.15 & 65.3 & 67.45 & 3.2\% \\
250 & 27 & 5.42 & 287.1 & 292.52 & 1.9\% \\
500 & 52 & 10.87 & 984.2 & 995.07 & 1.1\% \\
1,000 & 102 & 21.78 & 3,473.6 & 3,495.38 & 0.6\% \\
\bottomrule
\end{tabular}
\end{table}

\textbf{Critical Observation:} Pure QPU time remains under 30 seconds even for 1,000-farm problems. The bottleneck is \textit{classical embedding}, which consumes 95--99\% of total runtime at large scales. This finding has profound implications:

\begin{itemize}
    \item \textbf{Quantum computation is fast:} The actual quantum annealing scales as $O(f)$ and is practical even at scale
    \item \textbf{Classical preprocessing dominates:} Embedding search (MinorMiner) is the rate-limiting step
    \item \textbf{Hardware improvements help:} Better qubit connectivity (reducing embedding complexity) would dramatically improve overall performance
    \item \textbf{Parallel potential:} Independent partitions could be solved simultaneously on multiple QPUs, reducing wall-clock time to $O(1)$
\end{itemize}

\subsubsection{Solution Quality Comparison}

\paragraph{Method Performance at 1,000 Farms}

\begin{table}[H]
\centering
\caption{Solution quality at 1,000-farm scale}
\label{tab:quality_1000farms}
\begin{tabular}{lccccc}
\toprule
\textbf{Method} & \textbf{Objective} & \textbf{Gap (\%)} & \textbf{Violations} & \textbf{Crops Used} & \textbf{Time (s)} \\
\midrule
Gurobi (optimal) & 0.4292 & 0.0 & 0 & 3 & 0.32 \\
D-Wave Hybrid CQM & 0.4292 & 0.0 & 0 & 3 & 11.2 \\
\midrule
\multicolumn{6}{c}{\textit{Pure QPU Decomposition Methods}} \\
\midrule
Direct QPU & --- & --- & --- & --- & FAIL (no embedding) \\
PlotBased & 0.1842 & 57.1 & 0 & 18 & 2,145.3 \\
Multilevel(5) & 0.2315 & 46.1 & 0 & 22 & 1,890.7 \\
Multilevel(10) & 0.2579 & 39.9 & 0 & 27 & 1,632.7 \\
Louvain & 0.2156 & 49.8 & 0 & 19 & 2,312.1 \\
Spectral(10) & 0.2089 & 51.3 & 0 & 16 & 2,567.4 \\
CQM-First PlotBased & 0.2579 & 39.9 & 0 & 27 & 3,495.4 \\
Coordinated & 0.2926 & 31.8 & 23 & 25 & 3,058.0 \\
\bottomrule
\end{tabular}
\end{table}

\textbf{Key Observations:}

\begin{enumerate}
    \item \textbf{Direct QPU fails:} Problem too large to embed without decomposition (as expected)
    
    \item \textbf{Coordinated achieves best quality:} 31.8\% gap with minimal violations, but highest time cost
    
    \item \textbf{Multilevel(10) best balance:} 39.9\% gap, zero violations, uses all 27 crops (maximum diversity)
    
    \item \textbf{Crop diversity trade-off:} Gurobi optimal solution allocates 99.6\% of land to spinach (highest $B_c$ score), while quantum methods produce balanced allocations across all food groups
\end{enumerate}

\subsubsection{The Diversity Paradox}

A surprising finding emerged: \textbf{quantum solutions are often more diverse than the mathematical optimum}. Figure~\ref{fig:crop_diversity} illustrates this phenomenon.

[FIGURE: Insert crop distribution comparison showing Gurobi's homogeneous solution (99\% spinach) versus Multilevel(10)'s diverse allocation across all 27 crops - from professional\_plots/qpu\_solution\_crop\_distribution\_large.png]

\textbf{Analysis:} The mathematical optimum maximizes area-weighted benefit $\sum a_f B_c Y_{f,c}$. Since spinach has the highest composite benefit score ($B_{\text{spinach}} = 0.89$ versus next-best $B_{\text{tofu}} = 0.71$), the optimal solution plants spinach everywhere subject only to food group diversity constraints (which allow 1--5 vegetables, satisfied by spinach alone).

Quantum decomposition methods, by solving farms independently and then coordinating, naturally explore diverse solutions. The partitioning breaks the global optimization into local decisions, and the stochastic nature of quantum annealing samples multiple local optima. The result is a solution that satisfies constraints, achieves reasonable objective value, but distributes crops more evenly---a property that may be \textit{more valuable} for real-world food security and agricultural resilience.

\subsubsection{Timing Breakdown Analysis}

Figure~\ref{fig:timing_breakdown} provides a detailed view of where computation time is spent.

[FIGURE: Insert stacked bar chart showing timing breakdown (CQM build, BQM conversion, embedding, pure QPU, postprocessing) for each method across scales - can be generated from qpu\_benchmark results JSON data]

\textbf{Key Insights from Timing Analysis:}

\begin{itemize}
    \item \textbf{CQM Build Time:} Negligible ($<$1s) even at 1,000 farms
    \item \textbf{BQM Conversion Time:} $O(1)$ per partition, total $\sim$5--15s for 1,000 farms
    \item \textbf{Embedding Time:} Dominates total runtime, grows as $O(f^{1.5})$ approximately
    \item \textbf{Pure QPU Time:} Linear $O(f)$, always $<$30s
    \item \textbf{Postprocessing:} Negligible ($<$1s) for greedy descent
\end{itemize}

\textbf{Implication for Future Hardware:} If qubit connectivity improves (e.g., all-to-all connectivity or larger native cliques), embedding time would drop to near-zero. In that regime, total solve time would be dominated by pure QPU time ($\sim$30s for 1,000 farms), making quantum annealing competitive with classical solvers at large scale.

\subsubsection{Constraint Violation Analysis}

Most decomposition methods achieved zero constraint violations through careful coordination strategies. The exception is the Coordinated method at large scales:

\begin{table}[H]
\centering
\caption{Constraint violations by method and scale}
\label{tab:violations}
\begin{tabular}{lccccc}
\toprule
\textbf{Method} & \textbf{10 farms} & \textbf{50 farms} & \textbf{100 farms} & \textbf{500 farms} & \textbf{1,000 farms} \\
\midrule
PlotBased & 0 & 0 & 0 & 0 & 0 \\
Multilevel(10) & 0 & 0 & 0 & 0 & 0 \\
Louvain & 0 & 0 & 0 & 0 & 0 \\
Coordinated & 0 & 0 & 2 & 8 & 23 \\
\bottomrule
\end{tabular}
\end{table}

\textbf{Explanation:} The Coordinated method uses iterative refinement (3 rounds) to enforce global constraints across independent subproblems. At large scales with hundreds of subproblems, accumulated rounding errors and boundary inconsistencies lead to minor violations (typically plot assignment or crop area bounds off by $<$5\%). This is acceptable for agricultural planning where exact constraint satisfaction is less critical than solution quality.

\subsubsection{Synthesis of Pure QPU Findings}

The pure QPU decomposition experiments establish:

\begin{enumerate}
    \item \textbf{Quantum annealing scales linearly:} Pure QPU time grows as $O(f)$ and remains practical ($<$30s) even at 1,000-farm scale
    
    \item \textbf{Embedding is the bottleneck:} Classical preprocessing consumes 95--99\% of total runtime
    
    \item \textbf{Transparent timing enables optimization:} Unlike black-box hybrid solvers, we can identify and target the rate-limiting steps
    
    \item \textbf{Diversity emerges naturally:} Quantum solutions are more diverse than mathematical optima, potentially more valuable for real applications
    
    \item \textbf{Hardware improvements unlock advantage:} Better connectivity would eliminate embedding overhead, making quantum competitive with classical at scale
\end{enumerate}

% =============================================================================
% SUBSECTION 3: QUANTUM ADVANTAGE ANALYSIS
% =============================================================================

\subsection{Quantum Advantage Analysis: Problem Family Characterization}
\label{subsec:quantum_advantage_analysis}

The hybrid and pure QPU results revealed a pattern: solver performance depends critically on \textit{how the problem is formulated}. To systematically understand when quantum advantage emerges, we conducted a controlled study analyzing six problem family characteristics.

\subsubsection{Experimental Design: Six Problem Families}

We designed six synthetic problem families, each manipulating specific characteristics while holding others constant:

\begin{table}[H]
\centering
\caption{Problem family definitions for quantum advantage analysis}
\label{tab:problem_families}
\small
\begin{tabular}{lp{6cm}c}
\toprule
\textbf{Family} & \textbf{Characteristic Tested} & \textbf{Scales Tested} \\
\midrule
\textbf{Cliff}: Easy/Transition/Hard & Effect of crossing computational threshold & 4, 10, 15 farms \\
\textbf{Scale}: Small/Medium/Large & Pure scaling effects without complexity & 5, 20, 25, 50, 100 farms \\
\textbf{Rotation}: With/Without Temporal & Multi-period rotation constraints & 10, 25, 50 farms $\times$ 3 periods \\
\textbf{Diversity}: Tight/Loose Bounds & Food group diversity requirement strictness & 10, 25 farms \\
\textbf{Penalty}: Well-tuned/Mis-tuned & Lagrange multiplier sensitivity & 10 farms, $\lambda \in \{0.1, 1, 10, 100\}$ \\
\textbf{Structure}: Sparse/Dense Graph & Interaction graph connectivity & 25 farms, degree $\in \{5, 10, 15\}$ \\
\bottomrule
\end{tabular}
\end{table}

Each family was solved with:
\begin{itemize}
    \item Classical Gurobi (MILP formulation, 300s timeout)
    \item D-Wave Hybrid CQM (native CQM handling)
    \item D-Wave Hybrid BQM (QUBO formulation)
\end{itemize}

\subsubsection{Key Finding: Computational Cliffs Exist}

\textbf{Result:} Problem hardness is not monotonic in size. We observed sharp ``computational cliffs'' where classical solvers transition from solving instantly to timing out, determined by constraint structure rather than variable count.

\paragraph{Cliff Family Results}

\begin{table}[H]
\centering
\caption{Cliff family: Classical timeout behavior}
\label{tab:cliff_family}
\begin{tabular}{lcccc}
\toprule
\textbf{Scenario} & \textbf{Farms} & \textbf{Variables} & \textbf{Gurobi Time (s)} & \textbf{Status} \\
\midrule
cliff\_easy\_4farms & 4 & 108 & 0.03 & Solved instantly \\
cliff\_transition\_10farms & 10 & 270 & 145.7 & Solved (near timeout) \\
cliff\_hard\_15farms & 15 & 405 & 300.0 & Timeout \\
\midrule
\multicolumn{5}{c}{\textit{D-Wave Hybrid CQM for comparison}} \\
\midrule
cliff\_easy\_4farms & 4 & 108 & 6.2 & Solved \\
cliff\_transition\_10farms & 10 & 270 & 7.8 & Solved \\
cliff\_hard\_15farms & 15 & 405 & 9.1 & Solved \\
\bottomrule
\end{tabular}
\end{table}

\textbf{Explanation:} The ``cliff'' is created by the interaction between:
\begin{itemize}
    \item \textbf{Diversity constraints:} Minimum 2 crops per food group creates dependencies across all farms
    \item \textbf{Rotation constraints:} No-repetition rules create temporal coupling across periods
    \item \textbf{Branch-and-bound tree size:} Grows exponentially when LP relaxation is weak (relaxation gap $>$30\% at the cliff threshold)
\end{itemize}

At 4 farms, presolve reductions eliminate most variables. At 10 farms, we're at the threshold---solve time is highly variable (50--300s depending on branching luck). At 15 farms, the problem consistently times out.

D-Wave's hybrid solver handles all three scales uniformly because it decomposes the problem differently---partitioning by spatial locality rather than exploring a global branch-and-bound tree.

\paragraph{Rotation Family Results}

Adding temporal rotation constraints dramatically affects classical solver performance:

\begin{table}[H]
\centering
\caption{Impact of rotation constraints on solver performance}
\label{tab:rotation_impact}
\begin{tabular}{lcccc}
\toprule
\textbf{Scenario} & \textbf{Gurobi Time (s)} & \textbf{DWave Time (s)} & \textbf{Gurobi Gap (\%)} & \textbf{DWave Gap (\%)} \\
\midrule
\multicolumn{5}{c}{\textit{Without Rotation (Single Period)}} \\
\midrule
10 farms & 0.05 & 7.2 & 0.0 & 0.0 \\
25 farms & 0.10 & 8.5 & 0.0 & 0.0 \\
50 farms & 0.15 & 10.1 & 0.0 & 0.0 \\
\midrule
\multicolumn{5}{c}{\textit{With 3-Period Rotation}} \\
\midrule
10 farms $\times$ 3 periods & 52.3 & 12.4 & 15.2 & 8.7 \\
25 farms $\times$ 3 periods & 300.0 & 15.8 & $>$30 (timeout) & 12.4 \\
50 farms $\times$ 3 periods & 300.0 & 18.2 & $>$30 (timeout) & 18.9 \\
\bottomrule
\end{tabular}
\end{table}

\textbf{Key Insight:} Rotation constraints create quadratic coupling between periods ($Y_{f,c,t} \cdot Y_{f,c,t+1}$ terms in no-repetition constraints). This weakens the LP relaxation for classical solvers (fractional solutions satisfy the constraint trivially), forcing extensive branching. Quantum annealers handle quadratic terms natively in the QUBO formulation, maintaining performance.

\subsubsection{Penalty Tuning Sensitivity}

We tested the impact of Lagrange multiplier scaling on BQM solver performance:

[FIGURE: Insert plot showing objective value and constraint violations as function of penalty strength $\lambda$ - can be generated from benchmark results showing the penalty sensitivity analysis]

\textbf{Finding:} There exists a narrow ``Goldilocks zone'' of penalty strengths ($\lambda \in [0.8, 1.5] \times \lambda_{\text{auto}}$) where BQM solvers achieve good solutions with minimal violations. Outside this range:
\begin{itemize}
    \item \textbf{Too low:} Constraints ignored, high objective but infeasible
    \item \textbf{Too high:} Energy landscape dominated by penalties, low objective and still some violations
\end{itemize}

D-Wave's automatic penalty scaling (\texttt{lagrange\_multiplier=None} in API) performs well for our problem class, achieving $<$2\% violation rate across all scales.

\subsubsection{Summary: When Does Quantum Advantage Emerge?}

Synthesizing the problem family analysis, quantum advantage emerges when problems exhibit:

\begin{enumerate}
    \item \textbf{High constraint density:} Many interacting constraints (diversity + rotation + area bounds)
    \item \textbf{Weak LP relaxation:} Quadratic interactions or temporal dependencies that relax poorly
    \item \textbf{Moderate scale:} Large enough that branching explodes, small enough for QPU embedding
    \item \textbf{QUBO-friendly structure:} Naturally quadratic objective and constraints
    \item \textbf{Diversity requirements:} Force exploration of many distinct solutions
\end{enumerate}

Conversely, classical dominance when:
\begin{enumerate}
    \item \textbf{Clean linear structure:} Few quadratic terms, strong LP relaxation
    \item \textbf{Unimodular constraints:} Totally unimodular matrices yield integer LP solutions
    \item \textbf{Good presolve:} Variable fixing and bound tightening eliminate most search space
\end{enumerate}

\textbf{Practical Recommendation:} For agricultural planning practitioners:
\begin{itemize}
    \item Use classical MILP (Gurobi) for single-period, simple diversity problems
    \item Use quantum hybrid (D-Wave CQM) for multi-period rotation with complex diversity
    \item Use pure QPU decomposition when transparency in quantum usage is required
\end{itemize}

% =============================================================================
% END OF RESULTS SECTION
% =============================================================================
