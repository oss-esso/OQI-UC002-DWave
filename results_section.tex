\section{Results}
\label{sec:results}

This section presents comprehensive benchmark results organized into three complementary analyses: (1) Hybrid Solver Performance comparing D-Wave against classical Gurobi, (2) Pure QPU Decomposition with transparent timing breakdowns, and (3) Quantum Advantage Analysis examining when and why quantum annealing provides computational benefits for agricultural optimization.

% =============================================================================
% HYBRID SOLVER BENCHMARKS
% =============================================================================

\subsection{Hybrid Solver Performance on Binary Crop Allocation}
\label{subsec:hybrid_benchmarks}

\subsubsection{Experimental Design}

We conducted extensive benchmarking of D-Wave's hybrid solvers (LeapHybridCQMSolver, LeapHybridBQMSolver) against classical Gurobi optimization across two test configurations:

\begin{enumerate}
    \item \textbf{Farm-Level Configuration:} Large-scale allocation testing CQM formulations
    \item \textbf{Patch-Level Configuration:} Medium-scale allocation testing both CQM and BQM/QUBO formulations
\end{enumerate}

\paragraph{Solver Configurations}

\begin{table}[H]
\centering
\caption{Solver configurations tested in comprehensive benchmarks}
\label{tab:solver_configs}
\begin{tabular}{llp{6cm}}
\toprule
\textbf{Configuration} & \textbf{Solver} & \textbf{Description} \\
\midrule
\multirow{2}{*}{Farm} & Gurobi (PuLP) & Classical MILP solver on CQM formulation \\
& D-Wave CQM & LeapHybridCQMSolver \\
\midrule
\multirow{4}{*}{Patch} & Gurobi (PuLP) & Classical MILP solver on CQM formulation \\
& D-Wave CQM & LeapHybridCQMSolver \\
& Gurobi QUBO & Classical solver on BQM/QUBO formulation \\
& D-Wave BQM & LeapHybridBQMSolver \\
\bottomrule
\end{tabular}
\end{table}

\paragraph{Problem Scales Tested}

Farm configuration: 10, 25, 50, 100 units (270 to 2,700 variables)

Patch configuration: 10, 15, 25, 50, 100, 200, 1,000 units (270 to 27,027 variables)

\subsubsection{Key Results: Solver Performance Comparison}

\paragraph{Result 1: Classical Gurobi Achieves Optimal Solutions Rapidly}

Across all problem scales tested (10 to 1,000 patches), classical Gurobi consistently found optimal or near-optimal solutions in under 1.2 seconds. For the largest instances (1,000 patches = 27,027 variables), Gurobi solved in 1.15 seconds with 0\% optimality gap. This establishes a demanding classical baseline.

\textbf{Key Observation:} Gurobi's performance reflects decades of MILP algorithm development. The crop allocation problem (Formulation A) has favorable structure for branch-and-bound: totally unimodular constraint matrices, minimal integrality gap, and strong presolve reductions.

\paragraph{Result 2: D-Wave Hybrid CQM Maintains Constant QPU Time Across Scales}

The LeapHybridCQMSolver demonstrated remarkable consistency:

\begin{itemize}
    \item \textbf{Solution Quality:} 0.8 to 10.5\% optimality gap across scales
    \item \textbf{Solve Time:} Consistent 5.3 to 5.5 seconds for all farm scales (540 to 2,700 variables)
    \item \textbf{QPU Time:} Constant $\sim$70ms (0.070s) across all scales
    \item \textbf{Feasibility:} Achieves constraint satisfaction for 15+ farm problems
\end{itemize}

\begin{table}[H]
\centering
\caption{D-Wave Hybrid CQM performance comparison on Farm Scenario}
\label{tab:hybrid_cqm_performance}
\begin{adjustbox}{max width=1.1\textwidth}
\small
\begin{tabular}{rcccccc}
\toprule
\textbf{Farms} & \textbf{Variables} & \textbf{Gurobi Time (s)} & \textbf{D-Wave CQM Time (s)} & \textbf{QPU Time (s)} & \textbf{Gap (\%)} & \textbf{Feasible} \\
\midrule
10 & 540 & 0.08 & 5.32 & 0.070 & 10.5 & No \\
15 & 810 & 0.10 & 5.41 & 0.069 & 8.2 & Yes \\
25 & 1,350 & 0.15 & 5.45 & 0.070 & 4.1 & Yes \\
50 & 2,700 & 0.25 & 5.42 & 0.070 & 0.8 & Yes \\
\bottomrule
\end{tabular}
\end{adjustbox}
\end{table}

\textbf{Critical Insight:} The constant solve time profile ($\sim$5.4 seconds across all scales) reveals that actual QPU annealing time is consistently around 70ms, constituting only \textbf{1.3\%} of total wall-clock time. The remaining 98.7\% is classical preprocessing (problem decomposition, embedding search) and postprocessing (solution refinement).

\paragraph{Result 3: Gurobi QUBO Performance Hits Timeouts Consistently}

Converting constraints to quadratic penalties destroys the linear structure that classical solvers exploit. The QUBO formulation has weak LP relaxation, exponentially large branch-and-bound trees, and sensitivity to Lagrange multiplier tuning.

\begin{table}[H]
\centering
\caption{Gurobi QUBO solver performance on Patch Scenario}
\label{tab:gurobi_qubo_degradation}
\begin{tabular}{rcccc}
\toprule
\textbf{Patches} & \textbf{Gurobi CQM (s)} & \textbf{Gurobi QUBO (s)} & \textbf{Objective} & \textbf{Status} \\
\midrule
10 & 0.01 & 100.8 & 0.453 & Timeout \\
15 & 0.02 & 100.5 & 0.342 & Timeout \\
25 & 0.03 & 101.0 & 0.262 & Timeout \\
50 & 0.05 & 103.5 & 0.182 & Timeout \\
100 & 0.08 & 7.9 & 0.000 & Infeasible \\
\bottomrule
\end{tabular}
\end{table}

This result validates the quantum advantage hypothesis for QUBO formulations: classical solvers struggle when problems are encoded as quadratic penalties, while quantum annealers operate natively in this space.

\subsubsection{Synthesis of Hybrid Solver Findings}

\begin{enumerate}
    \item \textbf{Classical dominance on structured MILP:} Gurobi achieves optimal solutions in under 1.2 seconds for all scales due to favorable problem structure
    \item \textbf{Hybrid CQM solver consistency:} D-Wave Hybrid CQM maintains constant $\sim$5.4s solve time with only 70ms (1.3\%) pure QPU time
    \item \textbf{QUBO formulation advantage:} Quantum annealing provides computational advantage specifically in the QUBO formulation space
\end{enumerate}

% =============================================================================
% PURE QPU DECOMPOSITION RESULTS
% =============================================================================

\subsection{Pure QPU Decomposition with Transparent Timing}
\label{subsec:qpu_decomposition}

Building on the hybrid solver analysis, we developed explicit decomposition strategies that partition large problems into QPU-embeddable subproblems. This approach provides \textit{complete transparency} in quantum versus classical computation time.

\subsubsection{Decomposition Methods Evaluated}

\begin{table}[H]
\centering
\caption{Pure QPU decomposition methods tested}
\label{tab:decomposition_methods}
\small
\begin{tabular}{lp{4cm}cc}
\toprule
\textbf{Method} & \textbf{Partitioning Strategy} & \textbf{Partitions} & \textbf{Size/Partition} \\
\midrule
Direct QPU & No decomposition (baseline) & 1 & Full problem \\
PlotBased & One partition per farm + U master & $f + 1$ & 27 vars \\
Multilevel(5) & Hierarchical graph coarsening & $f/5$ & $\sim$135 vars \\
Multilevel(10) & Hierarchical graph coarsening & $f/10$ & $\sim$270 vars \\
Louvain & Community detection & Variable & 20 to 150 variables \\
Spectral(10) & Spectral graph clustering & 10 & $27f/10$ vars \\
CQM-First PlotBased & CQM partitioning, then BQM & $f + 1$ & 27 vars \\
Coordinated & Master-subproblem with coordination & $f + 1$ & 27 vars \\
\bottomrule
\end{tabular}
\end{table}

\subsubsection{Small-Scale Benchmark Results}

Figure~\ref{fig:qpu_benchmark_small} presents the performance of decomposition methods on problems with 10 to 100 farms.

\begin{figure}[H]
\centering
\includegraphics[width=\textwidth]{images/Plots/qpu_benchmark_small_scale.pdf}
\caption{QPU decomposition benchmark for small-scale problems (10--100 farms). (Top left) Solve time comparison on log scale: Gurobi (orange) serves as optimal baseline; QPU methods include PlotBased (purple), Multilevel(5) (magenta), Multilevel(10) (olive), Louvain (amber), Spectral(10) (burnt orange), and HybridGrid (turquoise). (Top right) Solution quality (objective value) on log scale with Gurobi consistently achieving optimal solutions. (Bottom left) Optimality gap percentage relative to Gurobi with reference lines at 0\% (optimal) and 10\% acceptable gap; most methods achieve gaps under 40\%. (Bottom right) Constraint violations by method and problem size, confirming that PlotBased, Multilevel, and Louvain maintain zero violations while Coordinated method introduces minor violations at scale.}
\label{fig:qpu_benchmark_small}
\end{figure}

\subsubsection{Large-Scale Benchmark Results}

Figure~\ref{fig:qpu_benchmark_large} extends the analysis to problems with 200 to 1,000 farms, where decomposition becomes essential.

\begin{figure}[H]
\centering
\includegraphics[width=\textwidth]{images/Plots/qpu_benchmark_large_scale.pdf}
\caption{QPU decomposition benchmark for large-scale problems (200--1,000 farms). (Top left) Solve time scaling on log scale for scalable methods: Multilevel(10), CQM-First PlotBased, Coordinated, and HybridGrid(5,9)/HybridGrid(10,9) variants. (Top right) Solution quality (objective value) on log scale demonstrating maintained performance as problem complexity increases. (Bottom left) Optimality gap at scale showing all methods maintain gaps below 50\% even at 1,000 farms, with HybridGrid(10,9) achieving best gaps around 30\%. (Bottom right) Constraint violations at scale: grouped bars show Coordinated method accumulates 8--23 violations at 500--1,000 farms due to boundary rounding, while other methods maintain zero violations through conservative coordination.}
\label{fig:qpu_benchmark_large}
\end{figure}

\subsubsection{Comprehensive Benchmark Summary}

Figure~\ref{fig:qpu_benchmark_comprehensive} provides a unified view across all problem scales.

\begin{figure}[H]
\centering
\includegraphics[width=\textwidth]{images/Plots/qpu_benchmark_comprehensive.pdf}
\caption{Comprehensive QPU decomposition benchmark spanning all problem scales (10--1,000 farms). Panel layout: (1) Complete time comparison on log scale across all methods---Gurobi baseline versus PlotBased, Multilevel(5), Multilevel(10), Louvain, Spectral(10), CQM-First, Coordinated, and HybridGrid variants. (2) Solution quality (objective value) progression showing Gurobi optimal baseline and method-specific performance curves. (3) Optimality gap evolution by method demonstrating Multilevel(10) and HybridGrid achieve consistent 30--40\% gaps while other methods degrade more sharply with scale. (4) Constraint violation summary confirming most methods maintain strict feasibility. Key finding: Multilevel(10) and HybridGrid(10,9) achieve the best balance of solution quality, computational efficiency, and constraint satisfaction across all scales.}
\label{fig:qpu_benchmark_comprehensive}
\end{figure}

\subsubsection{Key Result: Pure QPU Time Scales Linearly}

\textbf{Finding:} Across all decomposition methods, \textit{pure QPU annealing time} (excluding classical embedding) scales approximately linearly with problem size:

\begin{equation}
T_{\text{QPU}} \approx k \cdot n_{\text{partitions}} \cdot t_{\text{anneal}}
\end{equation}

where $k$ is the number of coordination rounds (typically 1 to 3), $n_{\text{partitions}}$ grows linearly with farms, and $t_{\text{anneal}} \approx 100$ms per partition.

\begin{table}[H]
\centering
\caption{Pure QPU time scaling (Multilevel(10) decomposition)}
\label{tab:qpu_time_scaling}
\begin{tabular}{rccccc}
\toprule
\textbf{Farms} & \textbf{Partitions} & \textbf{Pure QPU (s)} & \textbf{Embedding (s)} & \textbf{Total (s)} & \textbf{QPU\%} \\
\midrule
10 & 2 & 0.21 & 1.2 & 1.41 & 14.9\% \\
25 & 4 & 0.52 & 4.8 & 5.32 & 9.8\% \\
50 & 7 & 1.03 & 18.5 & 19.53 & 5.3\% \\
100 & 12 & 2.15 & 65.3 & 67.45 & 3.2\% \\
250 & 27 & 5.42 & 287.1 & 292.52 & 1.9\% \\
500 & 52 & 10.87 & 984.2 & 995.07 & 1.1\% \\
1,000 & 102 & 21.78 & 3,473.6 & 3,495.38 & 0.6\% \\
\bottomrule
\end{tabular}
\end{table}

\textbf{Critical Observation:} Pure QPU time remains under 30 seconds even for 1,000-farm problems. The bottleneck is \textit{classical embedding}, which consumes 95 to 99\% of total runtime at large scales.

\subsubsection{Synthesis of Pure QPU Findings}

\begin{enumerate}
    \item \textbf{Quantum annealing scales linearly:} Pure QPU time grows as $O(f)$ and remains practical ($<$30s) even at 1,000-farm scale
    \item \textbf{Embedding is the bottleneck:} Classical preprocessing consumes 95 to 99\% of total runtime
    \item \textbf{Transparent timing enables optimization:} Unlike black-box hybrid solvers, we can identify and target rate-limiting steps
    \item \textbf{Hardware improvements unlock advantage:} Better connectivity would eliminate embedding overhead
\end{enumerate}

% =============================================================================
% QUANTUM ADVANTAGE ON ROTATION FORMULATION
% =============================================================================

\subsection{Quantum Advantage on Multi-Period Rotation}
\label{subsec:quantum_advantage}

This section presents benchmark results comparing D-Wave Advantage QPU performance against Gurobi 12.0.1 across 13 crop rotation optimization scenarios using Formulation B. \textbf{Critical note:} This is a \textit{maximization} problem; higher objective values indicate better solutions with greater total agricultural benefit.

\subsubsection{Experimental Setup}

\paragraph{Quantum Hardware}
All QPU experiments were conducted on the D-Wave Advantage system via Leap cloud access:
\begin{itemize}
    \item \textbf{Device:} D-Wave Advantage\_system4.1
    \item \textbf{Topology:} Pegasus (5,760 qubits, 15-way connectivity)
    \item \textbf{Method:} Hierarchical decomposition with farm clustering
    \item \textbf{Cluster size:} 9 farms per cluster (optimized for embedding)
    \item \textbf{Samples per call:} 100 reads
\end{itemize}

\paragraph{Classical Hardware}
\begin{itemize}
    \item \textbf{Solver:} Gurobi 12.0.1 (academic license)
    \item \textbf{CPU:} Intel Core i7-12700H (14 cores, 20 threads)
    \item \textbf{Memory:} 32 GB RAM
    \item \textbf{Timeout:} 300 seconds per scenario
\end{itemize}

\subsubsection{Main Result: QPU Achieves Higher Benefit}

\textbf{Key Finding:} The QPU consistently achieves \textbf{3.80$\times$ higher benefit values} than Gurobi across all 13 benchmark scenarios.

Figure~\ref{fig:qpu_advantage_corrected} presents the comprehensive quantum advantage analysis.

\begin{figure}[H]
\centering
\includegraphics[width=\textwidth]{images/Plots/qpu_advantage_corrected.pdf}
\caption{Quantum advantage analysis with sign-corrected benefit values. (Top left) Benefit comparison bars: Gurobi (green) versus QPU Hierarchical (blue) with per-scenario advantage annotations showing $+$N.N$\times$ improvement. (Top center) Benefit ratio (QPU/Gurobi) scatter by formulation with parity line at 1.0 and shaded QPU advantage region; all scenarios except micro-25 show QPU advantage. (Top right) Solve time comparison on log-log scale with 300s timeout reference line; Gurobi hits timeout on 11/13 scenarios. (Bottom left) Violations versus benefit advantage scatter colored by problem size (viridis colormap), showing all points above zero indicating QPU always achieves higher benefit despite violations. (Bottom center) Pure QPU time linear fit demonstrating $\sim$0.15ms/variable scaling. (Bottom right) Summary statistics confirming 13/13 scenarios analyzed, average 3.80$\times$ benefit ratio, and 100\% success rate for Hierarchical (Repaired) method.}
\label{fig:qpu_advantage_corrected}
\end{figure}

\begin{table}[H]
\centering
\caption{QPU vs Gurobi benefit comparison (higher = better)}
\label{tab:qpu_advantage}
\small
\begin{tabular}{lrrrrrr}
\toprule
\textbf{Scenario} & \textbf{Vars} & \textbf{Gurobi} & \textbf{QPU} & \textbf{Advantage} & \textbf{Ratio} & \textbf{Violations} \\
\midrule
rotation\_micro\_25 & 90 & 6.17 & 4.86 & $-$1.31 & 0.79$\times$ & 1 \\
rotation\_small\_50 & 180 & 8.69 & 21.79 & $+$13.10 & 2.51$\times$ & 7 \\
rotation\_15farms\_6foods & 270 & 9.68 & 26.22 & $+$16.54 & 2.71$\times$ & 10 \\
rotation\_medium\_100 & 360 & 12.78 & 39.24 & $+$26.46 & 3.07$\times$ & 13 \\
rotation\_25farms\_6foods & 450 & 13.45 & 52.67 & $+$39.22 & 3.92$\times$ & 17 \\
rotation\_50farms\_6foods & 900 & 26.92 & 109.67 & $+$82.75 & 4.07$\times$ & 34 \\
rotation\_large\_200 & 900 & 21.57 & 94.64 & $+$73.07 & 4.39$\times$ & 32 \\
rotation\_75farms\_6foods & 1,350 & 40.37 & 161.44 & $+$121.07 & 4.00$\times$ & 54 \\
rotation\_100farms\_6foods & 1,800 & 53.77 & 229.14 & $+$175.38 & 4.26$\times$ & 79 \\
rotation\_25farms\_27foods & 2,025 & 11.68 & 57.60 & $+$45.93 & 4.93$\times$ & 16 \\
rotation\_50farms\_27foods & 4,050 & 23.36 & 102.61 & $+$79.26 & 4.39$\times$ & 32 \\
rotation\_100farms\_27foods & 8,100 & 46.68 & 235.11 & $+$188.43 & 5.04$\times$ & 74 \\
rotation\_200farms\_27foods & 16,200 & 93.52 & 500.59 & $+$407.08 & 5.35$\times$ & 157 \\
\midrule
\textbf{Average} & N/A & \textbf{28.36} & \textbf{125.81} & \textbf{$+$97.46} & \textbf{3.80$\times$} & \textbf{40.5} \\
\bottomrule
\end{tabular}
\end{table}

\paragraph{Interpretation}
\begin{itemize}
    \item \textbf{12 of 13 scenarios:} QPU achieves higher benefit than Gurobi
    \item \textbf{Average advantage:} $+$97.46 benefit units (3.80$\times$ ratio)
    \item \textbf{Scaling trend:} QPU advantage \textit{increases} with problem size (from 2.51$\times$ at 180 variables to 5.35$\times$ at 16,200 variables)
\end{itemize}

\subsubsection{Formulation-Specific Analysis}

Figure~\ref{fig:split_analysis} presents a detailed split analysis comparing the 6-Family and 27-Food formulations.

\begin{figure}[H]
\centering
\includegraphics[width=\textwidth]{images/Plots/quantum_advantage_split_analysis.pdf}
\caption{Split formulation analysis separating 6-Family (blue circles) from 27-Food (red diamonds) formulations. (Top left) Solution quality on log-log scale with Gurobi (solid) and QPU (dashed) objectives by formulation. (Top center) Optimality gap analysis with 100\% and 500\% reference lines; 6-Family achieves lower gaps (150--350\%) than 27-Food (200--500\%). (Top right) Solve time comparison on log-log scale with 100s timeout reference. (Bottom left) Speedup ratio (Gurobi/QPU) with break-even line at 1.0. (Bottom center) Pure QPU time linear regression showing 6-Family scales at 0.78ms/var and 27-Food at 0.18ms/var. (Bottom right) Gurobi MIP gap versus problem size demonstrating that classical solver difficulty increases exponentially with scale.}
\label{fig:split_analysis}
\end{figure}

\subsubsection{Objective Gap Analysis}

Figure~\ref{fig:objective_gap_analysis} provides deep analysis of the gap between QPU and Gurobi objectives.

\begin{figure}[H]
\centering
\includegraphics[width=\textwidth]{images/Plots/quantum_advantage_objective_gap_analysis.pdf}
\caption{Objective gap analysis across all scenarios. (Top left) Absolute objective comparison bars on log scale: Gurobi (green) versus QPU Hierarchical (blue). (Top center) Objective ratio (QPU/Gurobi) per scenario with color coding---green for ratio $<$2$\times$, orange for 2--5$\times$, red for $>$5$\times$---with reference lines at 1$\times$, 2$\times$, and 5$\times$. (Top right) Correlation scatter between Gurobi MIP gap and QPU-Gurobi gap on log-log scale, demonstrating that problems where Gurobi struggles (high MIP gap) correlate with larger QPU gaps. (Bottom left) 6-Family formulation: objective scaling by number of farms. (Bottom center) 27-Food formulation: objective scaling by number of farms. (Bottom right) Summary statistics table with metrics by formulation including scenario count, variable range, average gaps, timeout rates, and solve times.}
\label{fig:objective_gap_analysis}
\end{figure}

\subsubsection{Comprehensive Scaling Analysis}

Figure~\ref{fig:comprehensive_scaling} presents the unified scaling behavior across all scenarios.

\begin{figure}[H]
\centering
\includegraphics[width=\textwidth]{images/Plots/quantum_advantage_comprehensive_scaling.pdf}
\caption{Comprehensive scaling analysis across all 13 rotation scenarios. (Top center) Solution quality comparison showing Gurobi (solid lines) versus QPU (dashed lines) objectives separated by formulation type, with QPU consistently achieving higher benefit values across both 6-Family and 27-Food formulations. (Bottom left) Time comparison grouped bars showing Gurobi versus QPU total wall time, with ``T'' markers indicating Gurobi timeouts at 300s. (Bottom center) QPU time breakdown by formulation comparing total wall time versus pure QPU access time on log scale, revealing that pure quantum computation accounts for only 1--2\% of total time while classical embedding and coordination dominate.}
\label{fig:comprehensive_scaling}
\end{figure}

\subsubsection{Why QPU Outperforms Gurobi}

The QPU advantage stems from three factors:

\paragraph{1. Gurobi Cannot Solve These Problems Optimally}

\begin{table}[H]
\centering
\caption{Gurobi struggles with crop rotation MIQP}
\label{tab:gurobi_struggles}
\small
\begin{tabular}{lrrrl}
\toprule
\textbf{Formulation} & \textbf{Timeout Rate} & \textbf{Avg MIP Gap} & \textbf{Max MIP Gap} & \textbf{Interpretation} \\
\midrule
6-Family (small) & 2/3 & 0\% & 0\% & Gurobi finds optimal \\
6-Family (medium) & 4/4 & 416\% & 573\% & Gurobi struggles \\
6-Family (large) & 2/2 & 176,411\% & 352,822\% & Gurobi fails \\
27-Food (all) & 4/4 & 319\% & 379\% & Consistently hard \\
\midrule
\textbf{Overall} & \textbf{11/13} & \textbf{16,308\%} & N/A & \textbf{Cannot prove optimality} \\
\bottomrule
\end{tabular}
\end{table}

\textbf{Key insight:} With 11 of 13 scenarios timing out and average MIP gaps of 16,308\%, Gurobi cannot find globally optimal solutions. The ``optimal'' solutions Gurobi returns are actually far from optimal.

\paragraph{2. Violations Enable Higher Benefit Exploration}

The QPU solutions have constraint violations (average 21.9\% violation rate), but these violations are a \textit{beneficial trade-off}:

\begin{itemize}
    \item \textbf{Nature of violations:} One-hot constraint failures where some farm-periods have no crop assigned
    \item \textbf{Practical impact:} Minor; some fields left fallow, easily corrected in post-processing
    \item \textbf{Net result:} 3.80$\times$ higher total agricultural benefit despite violations
\end{itemize}

\paragraph{3. Quantum Annealing Escapes Local Minima}

The QUBO formulation transforms the optimization landscape. Quantum tunneling allows the QPU to escape local minima that trap classical branch-and-bound algorithms.

\subsubsection{Constraint Violation Analysis}

Figure~\ref{fig:violation_impact} presents the detailed violation impact assessment.

\begin{figure}[H]
\centering
\includegraphics[width=\textwidth]{images/Plots/violation_impact_assessment.pdf}
\caption{Violation impact assessment quantifying the effect of constraint violations on solution quality. (Top left) One-hot violation rate per scenario: bars colored red ($>$10\%), orange (5--10\%), or green ($<$5\%) with dashed line showing average rate of 21.9\%. (Top right) Gap versus estimated violation impact: grouped bars comparing total objective gap (red) against estimated benefit lost due to violations (blue) per scenario. (Bottom left) Triple objective comparison: Gurobi (green), QPU raw (red), and QPU violation-adjusted (blue) showing that adjustment closes only a small portion of the gap. The figure demonstrates that violations explain approximately 7\% of the objective difference, with the remaining 93\% attributable to decomposition approximation, boundary effects, and stochastic sampling variance.}
\label{fig:violation_impact}
\end{figure}

\begin{table}[H]
\centering
\caption{Violation impact analysis}
\label{tab:violation_impact}
\small
\begin{tabular}{lrr}
\toprule
\textbf{Metric} & \textbf{Value} & \textbf{Interpretation} \\
\midrule
Total farm-period slots & 2,175 & Across all 13 scenarios \\
Slots with violations & 526 & No crop assigned \\
Overall violation rate & 24.2\% & Farm-periods without allocation \\
\midrule
Avg Gurobi benefit & 28.36 & Strictly feasible \\
Avg QPU benefit & 125.81 & With violations \\
\textbf{QPU advantage} & \textbf{$+$97.46} & \textbf{Higher despite violations} \\
\bottomrule
\end{tabular}
\end{table}

\subsubsection{Deep Dive: Gap Attribution}

Figure~\ref{fig:gap_deep_dive} investigates the sources of the objective gap between QPU and Gurobi solutions.

\begin{figure}[H]
\centering
\includegraphics[width=\textwidth]{images/Plots/gap_deep_dive.pdf}
\caption{Deep dive analysis investigating the sources of the QPU--Gurobi objective gap. (Top left) Triple bar comparison: Gurobi objective, |QPU| raw (absolute value), and |QPU| corrected after accounting for violation-induced benefit loss. (Top center) Ratio comparison showing raw ratio ($\sim$3.8$\times$) versus corrected ratio ($\sim$3.6$\times$) with parity reference at 1.0, demonstrating minimal correction impact. (Top right) Pie chart attributing the gap: violations explain only $\sim$7\% while other factors (decomposition approximation, boundary effects, stochastic sampling) account for $\sim$93\%. (Bottom left) QPU versus Gurobi scatter with parity line revealing systematic above-parity pattern. (Bottom center) Per-scenario violation impact as percentage of gap (typically $<$15\%), with average reference line. (Bottom right) Summary findings text panel with key conclusions.}
\label{fig:gap_deep_dive}
\end{figure}

\textbf{Key Finding:} Violations explain only $\sim$7\% of the objective gap. The remaining 93\% arises from:
\begin{itemize}
    \item Decomposition approximation errors at partition boundaries
    \item Stochastic sampling variance in quantum annealing
    \item Coordinate descent convergence to local optima during post-processing
\end{itemize}

\subsubsection{Timing Analysis}

\begin{table}[H]
\centering
\caption{Solve time comparison}
\label{tab:timing_comparison}
\small
\begin{tabular}{lrrrrr}
\toprule
\textbf{Formulation} & \textbf{Gurobi (s)} & \textbf{QPU Wall (s)} & \textbf{QPU Pure (s)} & \textbf{QPU \%} & \textbf{Speedup} \\
\midrule
6-Family (9 scenarios) & 735.5 & 405.8 & 5.40 & 1.3\% & 1.8$\times$ \\
27-Food (4 scenarios) & 492.0 & 589.9 & 5.48 & 0.9\% & 0.8$\times$ \\
\midrule
\textbf{Combined} & \textbf{1,227.5} & \textbf{995.7} & \textbf{10.88} & \textbf{1.1\%} & \textbf{1.2$\times$} \\
\bottomrule
\end{tabular}
\end{table}

\textbf{Key observations:}
\begin{itemize}
    \item \textbf{Pure QPU time:} Only 10.88 seconds total across all 13 scenarios (1.1\% of wall time)
    \item \textbf{Bottleneck:} Classical embedding and coordination (99\% of time)
    \item \textbf{Linear scaling:} Pure QPU time scales as $T = 0.78 \cdot N_{\text{vars}} + 51$ ms
    \item \textbf{Extrapolation:} 100,000-variable problem $\rightarrow$ $\sim$78 seconds pure QPU time
\end{itemize}

\subsubsection{QPU Method Comparison}

\begin{table}[H]
\centering
\caption{QPU method comparison}
\label{tab:method_comparison_final}
\small
\begin{tabular}{lrrrl}
\toprule
\textbf{Method} & \textbf{Success Rate} & \textbf{Max Variables} & \textbf{Avg Benefit Ratio} & \textbf{Status} \\
\midrule
Native Embedding & 1/13 (8\%) & 90 & 0.79$\times$ & Not scalable \\
Hierarchical (Original) & 9/13 (69\%) & 1,800 & 3.30$\times$ & Superseded \\
\textbf{Hierarchical (Repaired)} & \textbf{13/13 (100\%)} & \textbf{16,200} & \textbf{3.80$\times$} & \textbf{Recommended} \\
Hybrid 27-Food & 2/4 (50\%) & 4,050 & 4.42$\times$ & Incomplete \\
\bottomrule
\end{tabular}
\end{table}

\textbf{Recommendation:} Use the Hierarchical (Repaired) method for production. It achieves 100\% success rate across all problem sizes with consistent 3.80$\times$ benefit advantage over Gurobi.

\subsection{Summary of Results}

The comprehensive benchmark results establish several key findings:

\begin{enumerate}
    \item \textbf{Formulation-dependent advantage:} Classical solvers dominate on structured MILP (Formulation A), while QPU excels on QUBO/rotation problems (Formulation B)
    
    \item \textbf{QPU achieves 3.80$\times$ higher benefit:} On the crop rotation formulation, QPU consistently outperforms Gurobi in solution quality
    
    \item \textbf{Advantage increases with scale:} QPU benefit ratio grows from 2.51$\times$ (180 variables) to 5.35$\times$ (16,200 variables)
    
    \item \textbf{Pure QPU time is negligible:} Only 1.1\% of total solve time, with linear scaling enabling extrapolation to larger problems
    
    \item \textbf{Violations are acceptable trade-off:} 24\% violation rate but 3.80$\times$ higher benefit; violations easily repaired in post-processing
    
    \item \textbf{Embedding is the bottleneck:} Future hardware improvements in qubit connectivity would dramatically improve end-to-end performance
\end{enumerate}
