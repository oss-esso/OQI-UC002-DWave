\documentclass{article}
\usepackage{graphicx} % Required for inserting images
\usepackage{amsmath}
\usepackage{amssymb}
\usepackage{hyperref}
\usepackage{cite}
\usepackage{geometry}
\usepackage{xcolor}
\geometry{margin=1in}

\title{Quantum-Enhanced Conservation: Optimizing Protected Area Networks with QAOA and Gate-Based Quantum Algorithms}
\author{Edo}
\date{November 2025}

\begin{document}

\maketitle

\begin{abstract}
This document presents a comprehensive quantum computing approach to solving the reserve design problem in conservation biology using gate-based quantum algorithms. We demonstrate how the biodiversity conservation planning challenge can be reformulated as a K-SAT problem and solved using QAOA (Quantum Approximate Optimization Algorithm) and Grover-based quantum search on universal quantum computers. Leveraging recent theoretical results showing nearly quartic quantum speedups for planted inference problems with exponential space savings (Schmidhuber et al., 2024), and evidence that QAOA can match or exceed classical SAT solvers on hard random instances (Boulebnane \& Montanaro, 2022), this work bridges sustainable development goals with cutting-edge quantum optimization, providing a pathway from classical constraint programming through SAT encoding to quantum-enhanced decision-making for protecting Earth's ecosystems.
\end{abstract}

\tableofcontents
\newpage

\section{SDG Alignment}

\textbf{Primary SDG:}
\begin{itemize}
\item \textbf{Goal 15: Life on Land} - Protect, restore and promote sustainable use of terrestrial ecosystems, sustainably manage forests, combat desertification, and halt and reverse land degradation and halt biodiversity loss.
\end{itemize}

\textbf{Supporting SDGs:}
\begin{itemize}
\item \textbf{Goal 13: Climate Action} - Forests and protected areas serve as critical carbon sinks and buffer zones against climate change impacts.
\item \textbf{Goal 14: Life Below Water} - Coastal and marine-terrestrial interface protected areas support aquatic biodiversity.
\item \textbf{Goal 11: Sustainable Cities} - Strategic reserve design around human settlements balances conservation with sustainable urban development.
\end{itemize}

\section{Short Summary}

This project develops a quantum computing framework for optimizing biodiversity conservation decisions. Specifically, we address the \textit{reserve design problem}: selecting an optimal network of protected areas (planning units) that maximizes species representation while minimizing costs and ensuring spatial connectivity. We demonstrate a complete pipeline from mathematical formulation through K-SAT encoding to gate-based quantum algorithms (QAOA, Grover's algorithm), with classical SAT solvers providing baseline performance benchmarks.

\subsection*{Key Objectives:}
\begin{enumerate}
\item Formulate the reserve design problem as a constrained optimization challenge
\item Develop lossless conversion to Boolean satisfiability (K-SAT)
\item Implement classical SAT solver baseline (PySAT, Z3)
\item Design QAOA circuit implementations for gate-based quantum computers
\item Validate quantum approaches against real-world conservation scenarios
\item Demonstrate computational advantages for large-scale planning problems
\end{enumerate}

\section{Description of the Problem and Context}

\subsection{The Global Biodiversity Crisis}

\textit{What is the societal challenge and why is it critical?}

The world is facing an unprecedented biodiversity crisis. According to the United Nations Environment Programme:

\begin{itemize}
\item \textbf{1 million species} are threatened with extinction, many within decades - more than ever before in human history
\item \textbf{100 million hectares} of forest lost between 2000-2020 (31\% of Earth's land surface impacted)
\item \textbf{3.2 billion people} affected by land degradation globally
\item \textbf{90\% of deforestation} driven by agricultural expansion
\item \textbf{38\% of tree species} face extinction due to habitat loss, climate change, and overexploitation
\end{itemize}

This rapid loss of biodiversity undermines ecosystem services worth over \textbf{half of global GDP}, threatens food security, accelerates climate change, and increases pandemic risk through wildlife-human interface disruption.

Furthermore, a 2025 horizon scan of emerging biological conservation issues highlights novel challenges this project could help address, including \cite{sutherland2025}:

\begin{itemize}
\item \textbf{Accelerated Antarctic Changes:} Destabilization of Antarctic sea ice and accelerated melting of the Thwaites glacier, with profound impacts on global biodiversity.
\item \textbf{Novel Technological Impacts:} Metal and non-metal organic frameworks, extraction of rare earth elements from macroalgae, synthetic gene drives in plants, and low-emission cement.
\item \textbf{Water Quality and Quantity:} The compounded effects of deteriorating water quality and quantity on both human and natural systems.
\item \textbf{Novel Pollutants:} Risks from near-surface ozone and PFAS contamination.
\end{itemize}

\subsection{Conservation Planning Challenge}

\textit{Who and where is the affected population?}

Conservation planners worldwide face the critical challenge of designing protected area networks with:

\begin{itemize}
\item \textbf{Geographic Scope:} From local (individual national parks) to continental (trans-boundary corridors)
\item \textbf{Stakeholders:} Government agencies, NGOs (WWF, Conservation International), indigenous communities, private landowners
\item \textbf{Resources:} Limited budgets (e.g., \$10M-\$1B for major conservation programs)
\item \textbf{Biodiversity Hotspots:} Amazon rainforest, Congo Basin, Coral Triangle, Madagascar, etc.
\item \textbf{Constraints:} Political boundaries, land ownership, economic development pressures, social equity
\end{itemize}

\textbf{Current Impact:} The Kunming-Montreal Global Biodiversity Framework (2022) commits nations to protecting 30\% of Earth's land and oceans by 2030 ("30x30" target), requiring systematic planning for hundreds of thousands of new protected areas.

\textbf{Quantum Computing Opportunity:} As outlined in Babbush et al.'s "Grand Challenge of Quantum Applications" \cite{babbush2024}, demonstrating practical quantum advantage on real-world optimization problems remains a central challenge for the field. Conservation planning represents an ideal testbed: it combines scientific importance, computational hardness, and natural problem structure amenable to quantum approaches.

\subsection{The Reserve Design Problem}

The scientific formulation asks: \textit{Given a landscape divided into planning units, each containing different species at different costs, which sites should be protected to ensure all target species are adequately represented while minimizing total cost and maintaining spatial connectivity?}

This is formally known as the \textbf{Minimum Set Cover Problem with Connectivity Constraints}, proven to be NP-hard.

\section{Computational Challenge}

\subsection{Mathematical Formulation}

\textit{How is the societal challenge connected to a computational problem? What is the mathematical description?}

The reserve design problem can be formulated as a Mixed-Integer Linear Program (MILP):

\textbf{Given:}
\begin{itemize}
\item $S = \{1, 2, \ldots, n\}$: Set of $n$ planning units (sites)
\item $P = \{1, 2, \ldots, m\}$: Set of $m$ species (features)
\item $c_i$: Cost of protecting site $i$ (land acquisition, management)
\item $r_{ij} \in \{0,1\}$: Presence of species $j$ in site $i$
\item $t_j$: Target representation for species $j$ (e.g., $\geq 3$ sites)
\item $B$: Total budget constraint
\item $G = (S, E)$: Adjacency graph where $(i,j) \in E$ if sites are neighbors
\end{itemize}

\textbf{Decision Variables:}
\begin{itemize}
\item $x_i \in \{0,1\}$: Binary variable, 1 if site $i$ is selected
\item $y_j \in \{0,1\}$: Binary variable, 1 if species $j$ target is met
\item $z_{ij} \in \{0,1\}$: Connectivity variable for edge $(i,j)$
\end{itemize}

\textbf{Objective:}
\begin{equation}
\min \sum_{i=1}^{n} c_i x_i
\end{equation}

\textbf{Subject to:}
\begin{align}
\sum_{i=1}^{n} r_{ij} x_i &\geq t_j \quad \forall j \in P \tag{Species representation} \\
\sum_{i=1}^{n} c_i x_i &\leq B \tag{Budget constraint} \\
z_{ij} &= x_i \land x_j \quad \forall (i,j) \in E \tag{Connectivity} \\
\text{Selected sites} &\text{ form connected subgraph} \tag{Compactness}
\end{align}

\subsection{Current Classical Approaches}

\textit{How is the challenge currently approached with existing technologies?}

\textbf{Classical Solution Methods:}

\begin{enumerate}
\item \textbf{Greedy Heuristics (1980s-present):}
   \begin{itemize}
   \item Tools: Marxan, Zonation, ConsNet
   \item Method: Iteratively add sites with best cost-effectiveness ratio
   \item Speed: Fast (seconds to minutes)
   \item Quality: 10-30\% suboptimal, no optimality guarantee
   \end{itemize}

\item \textbf{Integer Linear Programming (2000s-present):}
   \begin{itemize}
   \item Solvers: Gurobi, CPLEX, COIN-OR
   \item Method: Branch-and-bound with cutting planes
   \item Speed: Minutes to hours for $n < 1000$
   \item Quality: Optimal (when solvable within time limits)
   \end{itemize}

\item \textbf{Constraint Programming (2010s-present):}
   \begin{itemize}
   \item Tools: Choco, Gecode, Google OR-Tools
   \item Method: Propagation + search with global constraints
   \item Speed: Comparable to ILP
   \item Quality: Optimal (with complete search)
   \end{itemize}

\item \textbf{Metaheuristics:}
   \begin{itemize}
   \item Simulated annealing, genetic algorithms, tabu search
   \item Speed: Fast but unpredictable
   \item Quality: No optimality guarantee
   \end{itemize}
\end{enumerate}

\subsection{Computational Bottlenecks}

\textit{What is difficult to model and what are current limitations?}

\textbf{Scalability Crisis:}
\begin{itemize}
\item \textbf{Problem Size:} Real-world instances have $n = 10^4$-$10^6$ planning units
\item \textbf{Complexity:} $O(2^n)$ solution space grows exponentially
\item \textbf{Time Limits:} ILP solvers timeout beyond $n \approx 5000$ sites with connectivity
\item \textbf{Memory:} Large-scale problems exceed RAM (>100GB for $n = 10^5$)
\end{itemize}

\textbf{Specific Challenges:}
\begin{enumerate}
\item \textbf{Connectivity Constraints:} Adding spatial connectivity increases complexity from Set Cover (NP-complete) to Steiner Tree (NP-hard with poor approximation ratios)

\item \textbf{Multiple Objectives:} Real planning requires balancing:
   \begin{itemize}
   \item Species representation vs. cost
   \item Compactness vs. connectivity
   \item Present species vs. future climate projections
   \item Conservation value vs. social/political feasibility
   \end{itemize}

\item \textbf{Uncertainty:} Species occurrence data is incomplete, costs are estimates, climate change impacts are probabilistic

\item \textbf{Dynamic Planning:} Need to adapt plans as new data arrives or conditions change
\end{enumerate}

\textbf{Current Classical Limitations:}
\begin{itemize}
\item Large instances ($n > 10,000$) require hours to days even with heuristics
\item No guarantee of finding optimal solution within practical time
\item Cannot efficiently handle very large problem instances for continental-scale planning
\item Branch-and-bound explores exponentially growing search trees
\end{itemize}

\section{Potential Impact of Quantum Solution}

\subsection{Quantum Approach: From Constraint Programming to K-SAT to Quantum Gate Algorithms}

\textit{From the classical computing approach, which part could be tackled with quantum computing?}

Our approach leverages gate-based quantum algorithms specifically designed for SAT solving:

\textbf{Transformation Pipeline:}
\begin{enumerate}
\item \textbf{Constraint Programming Formulation:} Express reserve design with global constraints (cardinality, connectivity, budget)

\item \textbf{K-SAT Encoding:} Convert constraints to Conjunctive Normal Form (CNF):
   \begin{itemize}
   \item Species representation: AtLeast-K constraints using sequential counter encoding
   \item Budget: Pseudo-Boolean constraints using binary adder circuits
   \item Connectivity: AND/OR gates via Tseitin transformation
   \item Objective: Binary search on cost bounds
   \end{itemize}

\item \textbf{Quantum SAT Solving:} Apply quantum algorithms to the CNF formula:
   \begin{itemize}
   \item QAOA (Quantum Approximate Optimization Algorithm) for k-SAT instances
   \item Grover's algorithm with amplitude amplification for solution search
   \item Quantum walk-based algorithms for structured SAT instances
   \item Hybrid classical-quantum approaches
   \end{itemize}
\end{enumerate}

\subsection{Quantum Algorithms and Methods}

\textit{What type of quantum algorithms and methods would be used and why?}

\textbf{Primary Approach: QAOA for k-SAT and Quantum SAT Algorithms}

\begin{enumerate}
\item \textbf{QAOA (Quantum Approximate Optimization Algorithm):}
   \begin{itemize}
   \item Gate-based variational quantum algorithm designed for combinatorial optimization
   \item Phase separator: Apply problem Hamiltonian $U_P(\gamma) = e^{-i\gamma H_P}$ where $H_P$ encodes SAT clauses as a sum of clause penalty terms
   \item Mixer: Apply $U_M(\beta) = e^{-i\beta H_M}$ where $H_M = \sum_i X_i$ explores the solution space
   \item Parameterized circuit: $|\psi(\gamma, \beta)\rangle = U_M(\beta_p)U_P(\gamma_p) \cdots U_M(\beta_1)U_P(\gamma_1)|+\rangle^{\otimes n}$
   \item Classical optimization of parameters $(\gamma, \beta)$ to maximize solution probability
   \item Depth $p$ controls approximation quality vs. circuit complexity tradeoff
   \item \textbf{Key Result (Boulebnane \& Montanaro, 2022):} For random 8-SAT at the satisfiability threshold, QAOA with approximately 14 ansatz layers matches the performance of the highly optimized classical WalkSATlm solver, and is predicted to outperform it with larger depth. This provides concrete evidence for quantum advantage on hard k-SAT instances \cite{boulebnane2022}
   \item Conservation planning problems, when encoded as k-SAT, exhibit similar structural properties to these hard random instances, making QAOA a promising approach
   \end{itemize}

\item \textbf{Grover-Based SAT Solving:}
   \begin{itemize}
   \item Quantum search for satisfying assignments
   \item Quadratic speedup: $O(\sqrt{2^n})$ vs. classical $O(2^n)$ for unstructured search
   \item Oracle marks satisfying assignments: $O:|x\rangle \to (-1)^{f(x)}|x\rangle$
   \item Amplitude amplification increases probability of finding solutions
   \item Multiple solutions: Modified Grover for counting and sampling
   \end{itemize}

\item \textbf{Quantum Speedups for Planted Inference (Recent Results):}
   \begin{itemize}
   \item \textbf{Nearly quartic speedups demonstrated} for planted inference problems \cite{schmidhuber2024}
   \item Planted k-SAT/k-XOR: instances with hidden satisfying assignments (planted solutions) embedded in a structured constraint satisfaction problem
   \item Conservation planning naturally exhibits "planted" structure: an optimal conservation plan represents a highly structured solution (satisfying biological, spatial, and budgetary constraints) hidden within the vast combinatorial search space
   \item \textbf{Speedup mechanism:} Quantum algorithm combines the Kikuchi Method (for marginal inference on factor graphs) with quantum phase estimation and amplitude amplification
   \item \textbf{Theoretical foundation:} Algorithm achieves runtime $\tilde{O}(n^{3/4})$ compared to classical $\tilde{O}(n)$ for planted k-SAT inference, demonstrating nearly quartic advantage
   \item \textbf{Space efficiency:} Quantum algorithm requires only $O(\log n)$ qubits, achieving exponential space savings over the classical Kikuchi method which requires polynomial space in $n^l$
   \item This result provides a concrete theoretical foundation for expecting quantum advantage on conservation planning when formulated as k-SAT
   \end{itemize}

\item \textbf{Hybrid Quantum-Classical Approaches:}
   \begin{itemize}
   \item \textbf{QAOA + classical optimizer:} Variational parameter optimization (COBYLA, BFGS)
   \item \textbf{Recursive QAOA:} Solve smaller sub-problems, fix variables, iterate
   \item \textbf{Quantum walks:} Explore solution space via continuous-time quantum walks on constraint graphs
   \item \textbf{VQE-style approaches:} Adapt Variational Quantum Eigensolver techniques for SAT
   \end{itemize}
\end{enumerate}

\textbf{Why Gate-Based Quantum Algorithms for k-SAT?}

\begin{itemize}
\item \textbf{Proven Speedups:} Grover's algorithm provides provable quadratic speedup ($O(\sqrt{2^n})$ vs $O(2^n)$) for unstructured search
\item \textbf{Structural Exploitation:} QAOA and quantum walks exploit problem structure (clause patterns, planted solutions, locality)
\item \textbf{Recent Theoretical Advances:} Nearly quartic speedups demonstrated for planted k-SAT instances, with exponential space savings \cite{schmidhuber2024}
\item \textbf{Empirical Evidence:} QAOA shown to match classical WalkSATlm on hard random 8-SAT instances with moderate circuit depth \cite{boulebnane2022}
\item \textbf{Natural Encoding:} K-SAT clauses map directly to quantum Hamiltonians via Pauli-Z operators and tensor products
\item \textbf{Universal Platforms:} Compatible with all gate-based quantum computers (IBM Quantum, Google Sycamore, IonQ, Rigetti, etc.)
\item \textbf{Error Mitigation:} Active research on noise-resilient QAOA implementations using zero-noise extrapolation and probabilistic error cancellation
\item \textbf{Ecological Structure:} Conservation problems have inherent structure (spatial autocorrelation, species co-occurrence, hierarchical constraints) that quantum algorithms can exploit through the planted solution framework
\item \textbf{Scalability Path:} Clear roadmap from NISQ demonstrations to fault-tolerant quantum advantage as hardware improves
\end{itemize}

\subsection{Projected Benefits Over Classical Approaches}

\textit{What are the projected expected benefits (speedup, accuracy, etc.)?}

\textbf{Potential Quantum Advantages:}

\begin{enumerate}
\item \textbf{Computational Speedup:}
   \begin{itemize}
   \item \textbf{Hypothesis:} Quadratic to polynomial speedup for structured problems
   \item \textbf{Target:} Solve $n = 10,000$ site problems in minutes vs. hours
   \item \textbf{Mechanism:} Quantum tunneling through energy barriers vs. thermal activation
   \item \textbf{Evidence:} Recent results on planted k-SAT show quartic speedups (Schmidhuber et al., 2024)
   \end{itemize}

\item \textbf{Solution Quality:}
   \begin{itemize}
   \item Access to broader solution landscape through quantum superposition
   \item Multiple near-optimal solutions (important for stakeholder decision-making)
   \item Better escape from local minima compared to simulated annealing
   \end{itemize}

\item \textbf{Scalability:}
   \begin{itemize}
   \item Hybrid quantum-classical approaches handle $n > 100,000$ via decomposition
   \item Parallel processing: 1000s of quantum samples in seconds
   \item Reduced memory requirements vs. classical branch-and-bound
   \end{itemize}

\item \textbf{Practical Conservation Impact:}
   \begin{itemize}
   \item \textbf{Faster planning cycles:} Respond to changing conditions (fires, development)
   \item \textbf{Larger scales:} Continental-level planning becomes feasible
   \item \textbf{Scenario analysis:} Explore many "what-if" scenarios rapidly
   \item \textbf{Adaptive management:} Real-time optimization as monitoring data arrives
   \end{itemize}
\end{enumerate}

\textbf{Quantitative Performance Targets (Based on Recent Results):}
\begin{itemize}
\item \textbf{Planted k-SAT instances:} Nearly quartic speedup ($\tilde{O}(n^{3/4})$ vs $\tilde{O}(n)$) demonstrated theoretically \cite{schmidhuber2024}
\item \textbf{Hard random k-SAT:} QAOA with $p \approx 14$ layers matches classical WalkSATlm on 8-SAT at satisfiability threshold \cite{boulebnane2022}
\item \textbf{Grover-based search:} Provable quadratic speedup ($O(\sqrt{2^n})$ vs $O(2^n)$) for unstructured search
\item \textbf{Space complexity:} Exponential savings: $O(\log n)$ qubits vs polynomial classical memory for planted inference \cite{schmidhuber2024}
\item \textbf{NISQ demonstrations:} Target $n = 20$-50 variables on current hardware (50-150 physical qubits with error mitigation)
\item \textbf{Near-term advantage:} $n = 50$-100 variables via hybrid quantum-classical decomposition strategies
\item \textbf{Fault-tolerant era:} $n > 500$ variables with full quantum advantage requires logical qubits with error correction
\end{itemize}

\subsection{Scalability and Resource Advantages}

\textbf{Space Efficiency:}
\begin{itemize}
\item \textbf{Exponential Memory Reduction:} The quantum algorithm for planted inference requires only $O(\log n)$ qubits, an exponential space saving compared to the classical Kikuchi method which requires storing probability distributions over $n^l$ configurations \cite{schmidhuber2024}
\item \textbf{Practical Advantage:} This memory advantage is particularly crucial for large-scale conservation planning where classical methods may become memory-bound before computation-bound (e.g., $n = 10,000$ sites would require terabytes classically but only $\approx 14$ qubits for quantum inference)
\item \textbf{NISQ Compatibility:} The logarithmic qubit requirement makes planted inference algorithms feasible on current near-term quantum devices (e.g., IBM's 127-qubit systems can handle $n \approx 2^{100}$ in principle)
\end{itemize}


\textbf{Note on Qubit Requirements:}
\begin{itemize}
\item \textbf{Direct QAOA/Grover Encoding:} Requires $n$ qubits for $n$ SAT variables plus ancillas for clause evaluation ($\sim 1.5$-$3n$ total)
\item \textbf{Planted Inference Framework:} Schmidhuber et al.'s approach requires only $O(\log n)$ qubits by computing marginal probabilities via quantum phase estimation rather than storing full joint distributions \cite{schmidhuber2024}
\item \textbf{Trade-off:} The planted inference approach requires classical preprocessing (constructing the Kikuchi factor graph) but provides dramatic space savings for the quantum component
\item \textbf{Hybrid Strategy:} Combine direct QAOA for small sub-problems with planted inference framework for global coordination
\end{itemize}


\subsection{Implementation Timeline}

\textit{What would be the timeline for implementation (NISQ vs. FTQC)?}

\textbf{Phase 1: NISQ Era (Current - 2027) - \textcolor{green}{CURRENTLY IMPLEMENTED}}

\begin{enumerate}
\item \textbf{Proof of Concept (Completed):}
   \begin{itemize}
   \item[\checkmark] Classical K-SAT encoding implemented (700+ lines Python)
   \item[\checkmark] Integration with 6 SAT solvers (Glucose, MiniSat, Z3, etc.)
   \item[\checkmark] Mathematical correctness proofs (LaTeX document, 30+ pages)
   \item[\checkmark] Validation on random and grid instances (up to $n = 100$ sites)
   \item[\checkmark] Performance benchmarking vs. constraint programming
   \end{itemize}

\item \textbf{QAOA Implementation (Next 6-12 months):}
   \begin{itemize}
   \item Implement k-SAT to QAOA circuit compilation
   \item Develop parameterized quantum circuits for SAT clauses
   \item Test on IBM Quantum / Amazon Braket / Azure Quantum platforms
   \item Implement classical parameter optimization (COBYLA, BFGS)
   \item Compare QAOA vs. classical SAT solver performance
   \item Explore hybrid quantum-classical decomposition strategies
   \end{itemize}

\item \textbf{Real-World Validation (12-24 months):}
   \begin{itemize}
   \item Partner with conservation organizations (e.g., The Nature Conservancy)
   \item Test on actual reserve design datasets (Costa Rica, Madagascar, etc.)
   \item Benchmark against Marxan, Zonation operational tools
   \item Publish peer-reviewed results
   \end{itemize}
\end{enumerate}

\textbf{Phase 2: Near-Term Quantum Advantage (2027-2030)}

\begin{itemize}
\item 10,000+ qubit systems with improved coherence
\item Error mitigation techniques mature
\item Routine quantum advantage demonstrations on specific problem classes
\item Commercial deployment for conservation agencies
\end{itemize}

\textbf{Phase 3: Fault-Tolerant Era (2030+)}

\begin{itemize}
\item Logical qubits with full error correction
\item Gate-model quantum algorithms (QAOA, Grover) surpass annealing
\item Quantum optimization as standard tool in computational ecology
\end{itemize}

\subsection{Quantum Resource Requirements}

\textit{What quantum resources would be required for a proof of concept?}

\textbf{Hardware Requirements:}

\begin{enumerate}
\item \textbf{Minimum (NISQ Proof of Concept):}
   \begin{itemize}
   \item \textbf{Platform:} IBM Quantum (127+ qubits, e.g., ibm\_sherbrooke), IonQ (32+ qubits), or Rigetti (80+ qubits)
   \item \textbf{Access:} Free tier (IBM Quantum, Amazon Braket free credits) or \$0.30-3/shot commercial
   \item \textbf{Problem Size:} $n = 20$-50 variables (sites), $m = 5$-10 species
   \item \textbf{Qubits Required:} $n$ qubits for SAT variables + ancillas for clause evaluation ($\sim$2-3$n$ total)
   \item \textbf{Circuit Depth:} $p = 1$-5 QAOA layers ($\sim$10-50 gates deep with current connectivity)
   \item \textbf{Shots:} 1000-10,000 measurements per parameter setting
   \item \textbf{Classical Optimization:} 50-200 iterations of parameter optimization
   \item \textbf{Wall-Clock Time:} 10 minutes - 2 hours (including queue time)
   \end{itemize}

\item \textbf{Intermediate (Near-Term Advantage Demonstration):}
   \begin{itemize}
   \item \textbf{Problem Size:} $n = 50$-100 variables with 200-500 clauses
   \item \textbf{Qubits:} 100-200 logical qubits (physical qubits with error correction)
   \item \textbf{Approach:} Recursive QAOA with problem decomposition
   \item \textbf{Circuit Depth:} $p = 5$-10 layers with hardware-efficient gates
   \item \textbf{Error Mitigation:} Zero-noise extrapolation, probabilistic error cancellation
   \item \textbf{Total Time:} 1-5 hours including multiple quantum-classical iterations
   \end{itemize}

\item \textbf{Advanced (Fault-Tolerant Era):}
   \begin{itemize}
   \item \textbf{Problem Size:} $n = 1000$+ variables (full conservation planning scales)
   \item \textbf{Qubits:} 1000-10,000 logical qubits with T-gate count $\sim 10^6$-$10^9$
   \item \textbf{Approach:} Grover's algorithm or amplitude amplification with deep circuits
   \item \textbf{Circuit Depth:} $O(\sqrt{2^n})$ for Grover, polynomial for QAOA
   \item \textbf{Total Time:} Minutes to hours depending on quantum advantage magnitude
   \end{itemize}
\end{enumerate}

\textbf{Software Requirements:}

\begin{itemize}
\item \textbf{Quantum Frameworks:} Qiskit (IBM), Cirq (Google), PennyLane, or Amazon Braket SDK
\item \textbf{Classical Baseline:} PySAT, Z3 for validation
\item \textbf{Encoding Tools:} Our custom K-SAT encoder (already implemented)
\item \textbf{QAOA Implementation:} Parameterized quantum circuits for SAT
\item \textbf{Classical Optimization:} SciPy (COBYLA, BFGS), NumPy for parameter tuning
\item \textbf{Visualization:} NetworkX, Matplotlib for solution analysis
\item \textbf{GIS Integration:} QGIS, GeoPandas for real-world data
\end{itemize}

\subsection{Available Datasets}

\textit{What datasets are available to this project?}

\textbf{Synthetic Datasets (Currently Used):}
\begin{enumerate}
\item \textbf{Random Instances:}
   \begin{itemize}
   \item Parameterized by $n$, $m$, budget fraction, species distribution
   \item Costs: uniform or power-law distribution
   \item Species presence: random with density parameter
   \item Adjacency: random geometric graphs or lattices
   \end{itemize}

\item \textbf{Grid Instances:}
   \begin{itemize}
   \item 2D grid topology (e.g., 10$\times$10, 20$\times$20)
   \item Clustered species distributions (mimics real biogeography)
   \item Useful for visualization and algorithm development
   \end{itemize}
\end{enumerate}

\textbf{Real-World Datasets (Available for Integration):}

\begin{enumerate}
\item \textbf{Protected Area Databases:}
   \begin{itemize}
   \item World Database on Protected Areas (WDPA)
   \item National GAP Analysis datasets (US)
   \item European Environment Agency conservation data
   \end{itemize}

\item \textbf{Species Occurrence Data:}
   \begin{itemize}
   \item GBIF (Global Biodiversity Information Facility): 2+ billion records
   \item eBird: 1+ billion bird observations
   \item IUCN Red List: 150,000+ assessed species
   \item Regional biodiversity atlases
   \end{itemize}

\item \textbf{Spatial Data:}
   \begin{itemize}
   \item Land cover: Copernicus Global Land Service, MODIS
   \item Elevation: SRTM, ASTER GDEM
   \item Administrative boundaries: GADM
   \item Land cost proxies: Property value maps, agricultural land prices
   \end{itemize}

\item \textbf{Benchmark Instances:}
   \begin{itemize}
   \item Conservation planning literature datasets (Marxan case studies)
   \item Systematic conservation planning benchmarks
   \item Published reserve design problems with known solutions
   \end{itemize}
\end{enumerate}

\textbf{Data Preprocessing Pipeline:}
\begin{enumerate}
\item Rasterize landscape to planning units (typical: 1-100 km$^2$ cells)
\item Overlay species range maps and occurrence points
\item Estimate protection costs (land value + management)
\item Build adjacency graph from spatial topology
\item Set representation targets (e.g., 10-30\% of species range)
\item Calculate budget as fraction of total landscape cost
\end{enumerate}

\subsection{Proof of Concept Completion Plan}

\textit{Describe how you envisage completing a proof of concept for this project.}

\textbf{Stage 1: Classical Baseline (COMPLETED)}

\begin{enumerate}
\item[\checkmark] \textbf{Mathematical Formulation:}
   \begin{itemize}
   \item Complete MILP formulation with connectivity constraints
   \item LaTeX documentation with correctness proofs (30+ pages)
   \item Complexity analysis and encoding size bounds
   \end{itemize}

\item[\checkmark] \textbf{K-SAT Encoding Implementation:}
   \begin{itemize}
   \item Python modules: ReserveDesignInstance, SATEncoder, SATSolver
   \item Multiple encoding strategies (sequential counter, binary adder, totalizer)
   \item Integration with 6 SAT solvers via PySAT
   \item Comprehensive test suite (6 unit tests)
   \end{itemize}

\item[\checkmark] \textbf{Validation:}
   \begin{itemize}
   \item Synthetic instances: $n = 10$-100 sites, $m = 3$-20 species
   \item Grid instances: 4$\times$4, 5$\times$5 with visualization
   \item Solution verification: feasibility, optimality (via binary search)
   \item Performance benchmarking: encoding time, solving time, memory
   \end{itemize}
\end{enumerate}

\textbf{Stage 2: QAOA Circuit Design (Next 3 months)}

\begin{enumerate}
\item \textbf{SAT-to-QAOA Mapping:}
   \begin{itemize}
   \item Implement clause-to-Hamiltonian conversion
   \item Design phase separator circuits: $U_P(\gamma) = \prod_c e^{-i\gamma_c H_c}$ where $H_c$ is clause Hamiltonian
   \item Implement mixer operations: $U_M(\beta) = \prod_i e^{-i\beta X_i}$
   \item Optimize gate count and circuit depth for NISQ devices
   \end{itemize}

\item \textbf{Classical Parameter Optimization:}
   \begin{itemize}
   \item Implement variational optimization loop
   \item Test multiple optimizers: COBYLA, BFGS, ADAM, SPSA
   \item Develop warm-start strategies from classical solutions
   \item Implement parameter transfer across problem instances
   \end{itemize}

\item \textbf{Classical QAOA Simulation:}
   \begin{itemize}
   \item Simulate QAOA circuits classically (up to $n \approx 20$ qubits)
   \item Verify correctness of quantum circuit compilation
   \item Benchmark against classical SAT solvers
   \item Identify optimal QAOA depth $p$ for problem class
   \end{itemize}
\end{enumerate}

\textbf{Stage 3: Quantum Hardware Execution (Months 4-6)}

\begin{enumerate}
\item \textbf{Quantum Platform Setup:}
   \begin{itemize}
   \item Install Qiskit / Cirq / PennyLane
   \item Obtain cloud quantum access (IBM Quantum, Amazon Braket, Azure Quantum)
   \item Configure API connections and authentication
   \end{itemize}

\item \textbf{QAOA Deployment:}
   \begin{itemize}
   \item Transpile circuits to hardware-native gates
   \item Implement qubit mapping and routing for device topology
   \item Apply error mitigation techniques (measurement error mitigation, zero-noise extrapolation)
   \item Handle device-specific constraints (gate fidelities, connectivity limits)
   \end{itemize}

\item \textbf{Quantum SAT Experiments:}
   \begin{itemize}
   \item Solve small instances ($n = 10$-30) directly on quantum hardware
   \item Vary QAOA depth $p$: 1-10 layers
   \item Collect 1000-10000 shots per parameter setting
   \item Analyze solution quality distribution and success probabilities
   \item Compare with Grover-based implementations
   \end{itemize}

\item \textbf{Hybrid Algorithm Development:}
   \begin{itemize}
   \item Recursive QAOA: solve sub-problems, fix variables, iterate
   \item Quantum-classical feedback loops
   \item Problem decomposition strategies
   \item Warm-starting from classical heuristics
   \end{itemize}
\end{enumerate}

\textbf{Stage 4: Performance Evaluation (Months 7-9)}

\begin{enumerate}
\item \textbf{Comprehensive Benchmarking:}
   \begin{itemize}
   \item Problem sizes: $n \in \{20, 50, 100, 200, 500\}$
   \item Metrics: Solution quality, time-to-solution, success rate
   \item Baselines: Classical SAT, ILP, greedy heuristics
   \item Statistical analysis: 50-100 runs per configuration
   \end{itemize}

\item \textbf{Quantum Advantage Analysis:}
   \begin{itemize}
   \item Identify problem regimes where quantum excels
   \item Quantify speedup factors
   \item Characterize scaling behavior
   \item Error analysis: embedding errors, thermal noise, control errors
   \end{itemize}

\item \textbf{Conservation-Relevant Metrics:}
   \begin{itemize}
   \item Species representation quality
   \item Spatial compactness of solutions
   \item Cost-effectiveness ratio
   \item Diversity of near-optimal solutions
   \end{itemize}
\end{enumerate}

\textbf{Stage 5: Real-World Validation (Months 10-12)}

\begin{enumerate}
\item \textbf{Case Study Selection:}
   \begin{itemize}
   \item Small scale: Local watershed ($n \approx 100$)
   \item Medium scale: County/province ($n \approx 1000$)
   \item Data sources: GBIF + WDPA + regional land value data
   \end{itemize}

\item \textbf{Stakeholder Engagement:}
   \begin{itemize}
   \item Consult conservation practitioners on practical constraints
   \item Incorporate domain knowledge (e.g., existing protected areas, political boundaries)
   \item Validate solutions with ecological expertise
   \end{itemize}

\item \textbf{Comparison with Operational Tools:}
   \begin{itemize}
   \item Marxan (most widely used globally)
   \item Zonation (European standard)
   \item Custom ILP/CP solutions
   \item Document advantages and limitations
   \end{itemize}
\end{enumerate}

\textbf{Stage 6: Dissemination (Ongoing)}

\begin{enumerate}
\item \textbf{Scientific Publication:}
   \begin{itemize}
   \item Target journals: Nature/Science (if strong quantum advantage), or specialized (Quantum, Conservation Biology)
   \item Paper structure: Introduction, Methods (K-SAT + QUBO), Results, Discussion
   \item Supplementary materials: Code repository, datasets, full proofs
   \end{itemize}

\item \textbf{Open-Source Release:}
   \begin{itemize}
   \item GitHub repository with MIT/Apache license
   \item Documentation: API reference, tutorials, examples
   \item Docker containers for reproducibility
   \item Jupyter notebooks for educational use
   \end{itemize}

\item \textbf{Community Engagement:}
   \begin{itemize}
   \item Workshops at conservation conferences (SCB, ICCB)
   \item Quantum computing conferences (APS March Meeting, QIP)
   \item Webinars for practitioners
   \item Policy briefs for UN, IUCN
   \end{itemize}
\end{enumerate}

\textbf{Success Criteria:}

\begin{enumerate}
\item \textbf{Technical:}
   \begin{itemize}
   \item QAOA solutions match or exceed classical quality on planted k-SAT instances
   \item Demonstrated speedup consistent with theoretical predictions (quadratic for Grover, quartic for planted instances)
   \item Successful execution on $n \geq 50$ qubits with NISQ hardware
   \item Error-mitigated results show quantum advantage signal
   \end{itemize}

\item \textbf{Scientific:}
   \begin{itemize}
   \item Publication in peer-reviewed quantum computing or optimization journal
   \item Experimental demonstration of quantum speedup for conservation-relevant problem class
   \item Contribution to understanding of QAOA performance on real-world structured SAT
   \item Presentation at quantum computing or conservation conferences
   \end{itemize}

\item \textbf{Practical:}
   \begin{itemize}
   \item Interest from at least one conservation organization for pilot studies
   \item Open-source implementation used by quantum computing community
   \item Integration pathway defined for future fault-tolerant systems
   \end{itemize}
\end{enumerate}

\section{Technical Implementation Details}

\subsection{Current Implementation Status}

Our proof-of-concept implementation includes:

\begin{enumerate}
\item \textbf{Problem Representation} (\texttt{reserve\_design\_instance.py}):
   \begin{itemize}
   \item Object-oriented design for problem instances
   \item Random and grid instance generators
   \item Solution validation and feasibility checking
   \item Connectivity verification algorithms
   \end{itemize}

\item \textbf{SAT Encoding} (\texttt{sat\_encoder.py}):
   \begin{itemize}
   \item Site selection variables
   \item Species representation (AtLeast-K cardinality constraints)
   \item Budget constraints (pseudo-Boolean encoding)
   \item Connectivity (AND gate encoding)
   \item Multiple encoding strategies with complexity analysis
   \end{itemize}

\item \textbf{Classical Solving} (\texttt{sat\_solver.py}):
   \begin{itemize}
   \item Unified interface to 6 SAT solvers
   \item Feasibility checking
   \item Optimization via binary search
   \item Detailed performance statistics
   \end{itemize}

\item \textbf{Examples and Tests} (\texttt{examples.py}, \texttt{test\_ksat.py}):
   \begin{itemize}
   \item 5 comprehensive usage examples
   \item 6 unit test cases covering all functionality
   \item Validation of encoding correctness
   \end{itemize}
\end{enumerate}

\subsection{Encoding Complexity Analysis}

\textbf{CNF Size:}
\begin{itemize}
\item \textbf{Variables:} $O(n + m + nm + nB)$ where terms represent site, species, presence, and budget encoding variables
\item \textbf{Clauses:} $O(nm + nB + |E|)$ for sequential counter encoding
\item \textbf{Alternative:} $O(nm \log B)$ with binary encoding for large budgets
\item \textbf{k-SAT Properties:} Typical clause width $k = 3$-5 for cardinality constraints, enabling efficient QAOA circuit construction
\end{itemize}

\textbf{Quantum Circuit Requirements:}
\begin{itemize}
\item \textbf{Qubits (QAOA):} $n$ logical qubits for SAT variables + $O(k \cdot C)$ ancillas for $C$ clauses of width $k$
\item \textbf{Qubits (Planted Inference):} $O(\log n)$ qubits via Kikuchi marginal computation \cite{schmidhuber2024}
\item \textbf{Circuit Depth (QAOA):} $O(pC)$ for $p$ layers and $C$ clauses, with depth-optimized parallelization
\item \textbf{Gate Count:} $O(pnC)$ total gates dominated by clause Hamiltonian phase separators
\end{itemize}

\section{References}

\begin{thebibliography}{99}

\bibitem{justiniano2018}
Justiniano-Albarracin, X., Birnbaum, P., \& Lorca, X. (2018).
\textit{Unifying reserve design strategies with graph theory and constraint programming}.
International Conference on Principles and Practice of Constraint Programming.

\bibitem{schmidhuber2024}
Schmidhuber, A., O'Donnell, R., Kothari, R., \& Babbush, R. (2024).
\textit{Quartic quantum speedups for planted inference}.
arXiv:2406.19378v2.

\bibitem{boulebnane2022}
Boulebnane, S., \& Montanaro, A. (2022).
\textit{Solving boolean satisfiability problems with the quantum approximate optimization algorithm}.
arXiv:2208.06909.

\bibitem{babbush2024}
Babbush, R., King, R., Boixo, S., Huggins, W., Khattar, T., Low, G. H., McClean, J. R., O'Brien, T., \& Rubin, N. C. (2024).
\textit{The Grand Challenge of Quantum Applications}.
arXiv:2511.09124.

\bibitem{sutherland2025}
Sutherland, W. J., Brotherton, P. N. M., Butterworth, H. M., Clarke, S. J., Davies, T. E., Doar, N., \ldots \& Thornton, A. (2025).
\textit{A horizon scan of biological conservation issues for 2025}.
Trends in Ecology \& Evolution, 40(1), 80-89.

\bibitem{farhi2014}
Farhi, E., Goldstone, J., \& Gutmann, S. (2014).
\textit{A Quantum Approximate Optimization Algorithm}.
arXiv:1411.4028.

\bibitem{grover1996}
Grover, L. K. (1996).
\textit{A fast quantum mechanical algorithm for database search}.
Proceedings of the twenty-eighth annual ACM symposium on Theory of computing, 212-219.

\bibitem{qiskit}
Qiskit Development Team (2024).
\textit{Qiskit: An Open-source Framework for Quantum Computing}.
https://qiskit.org/

\bibitem{ibmquantum}
IBM Quantum (2024).
\textit{IBM Quantum Computing Platform}.
https://quantum.ibm.com/

\bibitem{un2022}
United Nations (2022).
\textit{Kunming-Montreal Global Biodiversity Framework}.
Convention on Biological Diversity.

\bibitem{unsdg15}
United Nations Sustainable Development Goals.
\textit{Goal 15: Life on Land}.
https://sdgs.un.org/goals/goal15

\bibitem{pysat}
Ignatiev, A., Morgado, A., \& Marques-Silva, J. (2018).
\textit{PySAT: A Python Toolkit for Prototyping with SAT Oracles}.
International Conference on Theory and Applications of Satisfiability Testing.

\bibitem{marxan}
Ball, I. R., Possingham, H. P., \& Watts, M. (2009).
\textit{Marxan and relatives: Software for spatial conservation prioritisation}.
Spatial conservation prioritisation: Quantitative methods and computational tools, 185-195.

\bibitem{biere2009}
Biere, A., Heule, M., \& van Maaren, H. (2009).
\textit{Handbook of Satisfiability}.
IOS Press.

\bibitem{sinz2005}
Sinz, C. (2005).
\textit{Towards an optimal CNF encoding of boolean cardinality constraints}.
International Conference on Principles and Practice of Constraint Programming.

\end{thebibliography}



\end{document}
