\documentclass{article}
\usepackage{graphicx} % Required for inserting images

\title{outline template}
\author{Edo }
\date{November 2025}

\begin{document}

\maketitle


\section{SDG}

\textit{List the relevant SDGs}

\section{Short Summary}

\textit{Description of the scope and key objectives of the project}

\section{Description of the problem and context}


\textit{

\begin{itemize}

\item What is the societal challenge and why is it critical?
\item Who and where is the affected population?

\end{itemize}}

\section{Computational Challenge}

\textit{\begin{itemize}
    \item How is the societal challenge connected to a computational problem? What is the
mathematical description of the problem?
\item How is the challenge currently approached with existing technologies?
\item How is this type of computational problem solved on classical computers and why
are computational models useful?
\item What type of computational methods are used?
\item What is difficult to model and what are the current limitations/bottlenecks of the best
in class classical computing approaches?
\end{itemize}}

\section{Potential Impact of Quantum Solution}

\textit{\begin{itemize}
    \item From the classical computing approach, which part could be tackled with quantum
computing?
\item What type of quantum algorithms and methods would be used and why?
\item What are the projected expected benefits over classical approaches (speedup,
accuracy etc.)?
\item What would be the timeline for implementation (by classifying if a proof of concept
could be done with NISQ or FTQC devices)?
\item What quantum resources would be required to run a proof of concept (specify e.g.
type of quantum hardware, number of qubits, depth of circuits, computation time
needed (if possible), ...)
\item What datasets are available to this project? (please also specify other known /
necessary initial conditions for the model)
\item Describe how you envisage to complete a proof of concept for this project.
\end{itemize}}


\section{References}


\end{document}
