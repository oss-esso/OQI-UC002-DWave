\documentclass[11pt,a4paper]{article}
\usepackage[utf8]{inputenc}
\usepackage{amsmath}
\usepackage{amssymb}
\usepackage{geometry}
\usepackage{hyperref}
\usepackage{listings}
\usepackage{xcolor}

\geometry{margin=1in}

\lstset{
    language=Python,
    basicstyle=\ttfamily\small,
    keywordstyle=\color{blue},
    commentstyle=\color{gray},
    stringstyle=\color{red},
    showstringspaces=false,
    breaklines=true
}

	itle{Solver Documentation: Even-grid (Binary) and Uneven-farm (Continuous) Formulations\\
\large \texttt{solver\_runner\_BINARY.py} and Related Runners}
\author{Generated from codebase analysis}
\date{\today}

\begin{document}
\maketitle

\begin{abstract}
This document describes two complementary formulations used in the repository for agricultural land allocation:
\begin{itemize}
    \item an even-grid binary formulation (plots, binary assignment variables) designed to be compatible with BQM/BQUBO and quantum annealers; and
    \item an uneven-farm continuous formulation (realistic farm areas, continuous area allocation) identical in structure to the continuous solver in \texttt{solver\_runner.py}.
\end{itemize}
The document presents the mathematical models, constraints, solver choices, differences and similarities between the formulations, and integration notes for the codebase.
\end{abstract}

\section{Overview}

We support two ways to represent the same total land area $A_{\text{total}}$:
\begin{enumerate}
    \item \textbf{Even-grid (Binary)}: divide $A_{\text{total}}$ into $|P|$ equal-area plots. Decision variables are binary assignment variables $X_{p,c} \in \{0,1\}$. This formulation is suitable for conversion to BQM/BQUBO and for use with quantum annealers.
    \item \textbf{Uneven-farm (Continuous)}: represent land as $|F|$ farms with realistic, varying areas $A_f$. Decision variables are continuous area allocations $A_{f,c} \ge 0$. This formulation is solved by classical LP/MIP solvers and is the same as in \texttt{solver\_runner.py}.
\end{enumerate}

Both methods model the same high-level objectives and constraints (nutritional value, environmental impact, food-group diversity, min/max area per crop), but differ in their decision variable types and discretization.

\section{Notation}
Sets and indices:
\begin{itemize}
    \item $P$ — set of plots (even-grid, binary formulation), index $p$.
    \item $F$ — set of farms (uneven distribution, continuous formulation), index $f$.
    \item $C$ — set of crops, index $c$.
    \item $G$ — set of food groups, index $g$; $C_g \subseteq C$ are crops in group $g$.
\end{itemize}

Parameters (shared): weights $w_k$, attributes $\phi_{c,k}$, and area bounds $s_c^{\min}, s_c^{\max}$ for each crop. Total area is $A_{\text{total}}$.

\section{Even-grid (Binary) Formulation}

\subsection{Parameters}
\begin{align}
    a_p &= A_{\text{total}}/|P| && \text{area of each equal plot } p \in P
\end{align}

\subsection{Decision variables}
\begin{align}
    X_{p,c} &\in \{0,1\} && \forall p \in P, c \in C
\end{align}
where $X_{p,c}=1$ iff crop $c$ is assigned to plot $p$.

\subsection{Objective}
Define a per-crop benefit coefficient
\[ B_c = \sum_k w_k \phi_{c,k} \]
The area-weighted objective (normalized by total area) is
\[ \max \; \frac{1}{A_{\text{total}}} \sum_{p\in P} \sum_{c\in C} a_p \, B_c \, X_{p,c}. \]

\subsection{Constraints}
\begin{itemize}
    \item Single crop per plot: $\sum_{c\in C} X_{p,c} \le 1 \; \forall p\in P$.
    \item Crop area bounds: $\sum_{p\in P} a_p X_{p,c} \in [s_c^{\min}, s_c^{\max}] \; \forall c\in C$.
    \item Food-group diversity: introduce binary indicators $Y_c \in\{0,1\}$ with
        \[ Y_c \le \sum_{p\in P} X_{p,c}, \qquad \sum_{c\in C_g} Y_c \in [n_g^{\min}, n_g^{\max}] \; \forall g\in G. \]
\end{itemize}

This formulation is amenable to conversion to a CQM/BQM and then to a hybrid quantum/classical solve (via D-Wave APIs) or to classical MILP solvers (PuLP/Gurobi).

\section{Uneven-farm (Continuous) Formulation}

\subsection{Parameters}
Farms have varying areas $A_f > 0$ satisfying $\sum_{f\in F} A_f = A_{\text{total}}$.

\subsection{Decision variables}
Continuous allocation variables:
\[ A_{f,c} \ge 0, \qquad A_{f,c} \le A_f, \quad \forall f\in F, c\in C. \]

\subsection{Objective}
Same per-crop coefficient $B_c$ as above. Objective:
\[ \max \; \frac{1}{A_{\text{total}}} \sum_{f\in F} \sum_{c\in C} B_c \; A_{f,c}. \]

\subsection{Constraints}
\begin{itemize}
    \item Farm capacity: $\sum_{c\in C} A_{f,c} \le A_f \; \forall f\in F$.
    \item Crop area bounds: $\sum_{f\in F} A_{f,c} \in [s_c^{\min}, s_c^{\max}] \; \forall c\in C$.
    \item Food-group diversity: use indicator variables $Y_c\in\{0,1\}$ and big-M linking
        \[ M\, Y_c \ge \sum_{f\in F} A_{f,c}, \qquad \sum_{c\in C_g} Y_c \in [n_g^{\min}, n_g^{\max}]. \]
    \item Non-negativity: $A_{f,c} \ge 0$.
\end{itemize}

This formulation is a linear program (or MILP when indicators are required) and matches the continuous model implemented in \texttt{solver\_runner.py}.

\section{Differences and Similarities}
\begin{itemize}
    \item Both formulations maximize the same high-level objective expressed via $B_c$ and enforce the same global area and crop constraints; they differ only in the variable types and discretization.
    \item Binary formulation: combinatorial decisions, suitable for BQM conversion and quantum annealing; objective terms are area-weighted but written as sums over discrete plots.
    \item Continuous formulation: flexible fractional allocations per farm; typically yields higher relaxation quality and is solved by LP/MIP solvers.
\end{itemize}

\section{Implementation Notes (code integration)}
\begin{itemize}
    \item The repository contains both \texttt{solver\_runner\_BINARY.py} (which supports selecting the land method) and \texttt{solver\_runner.py} (continuous solver implementation). The binary runner will call the continuous solver when \texttt{--land-method uneven\_distribution} is selected, and use the binary/CQM flow for \texttt{even\_grid}.
    \item PuLP is used for classical solves (CBC or Gurobi if configured). D-Wave hybrid/Leap samplers are only used for the binary formulation.
    \item Output conventions: binary runs save \texttt{cqm\_binary\_...} and \texttt{pulp\_binary\_...}; continuous runs save \texttt{cqm\_continuous\_...} and \texttt{pulp\_continuous\_...}.
\end{itemize}

\section{Command-line Examples}
\begin{lstlisting}
# Binary formulation (even grid, quantum-capable)
python solver_runner_BINARY.py --land-method even_grid --n-units 25 --total-land 100.0

# Continuous formulation (uneven farms, classical LP)
python solver_runner_BINARY.py --land-method uneven_distribution --n-units 15 --total-land 100.0
\end{lstlisting}

\section{Quick Compilation / Preview}
To preview the document locally (if you have pdflatex installed):
\begin{verbatim}
pdflatex solver_runner_BINARY_documentation.tex
\end{verbatim}

\section{Conclusion}
This cleaned documentation clarifies the intended modeling choices: use binary, even-grid discretization when you want a quantum-compatible representation; use continuous, uneven-farm modeling when you want faithful, classical LP solutions. Both views are provided so you can compare solution quality and computational behaviour.

\appendix
\section{Relevant Files}
\begin{itemize}
    \item \texttt{solver\_runner\_BINARY.py} — runner that handles both land methods and dispatches to the appropriate formulation.
    \item \texttt{solver\_runner.py} — continuous formulation implementation (area variables per farm).
    \item \texttt{patch\_sampler.py}, \texttt{farm\_sampler.py} — land generation utilities used in the grid-refinement analysis.
\end{itemize}

\section{Dependencies}
\begin{lstlisting}
python packages:
- pulp
- tqdm
- dimod (optional, for CQM/BQM conversion)
- dwave-system (optional, for D-Wave sampling)
\end{lstlisting}

\end{document}
\begin{align}
    A_f &\in \mathbb{R}^+ \quad \text{Area of farm } f \in F \text{ (variable size)} \\
    w_k &\in [0,1] \quad \text{Weight for objective component } k \\
    \phi_{c,k} &\in [0,1] \quad \text{Value of crop } c \text{ for attribute } k \\
    s_c^{\min} &\in \mathbb{R}^+ \quad \text{Minimum area required for crop } c \\
    s_c^{\max} &\in \mathbb{R}^+ \quad \text{Maximum area allowed for crop } c
\end{align}

where $k \in \{\text{nutritional}, \text{nutrient\_density}, \text{environmental}, \text{affordability}, \text{sustainability}\}$.

For food group constraints (applicable to both representations):
\begin{align}
    C_g &\subseteq C \quad \text{Crops belonging to group } g \in G \\
    n_g^{\min} &\in \mathbb{Z}^+ \quad \text{Minimum number of different crops required from group } g \\
    n_g^{\max} &\in \mathbb{Z}^+ \quad \text{Maximum number of different crops allowed from group } g
\end{align}

\subsection{Decision Variables}

\subsubsection{For Even Grid Discretization (Binary)}
Binary assignment variables:
\begin{equation}
    X_{p,c} \in \{0, 1\} \quad \forall p \in P, c \in C
\end{equation}
where:
\begin{equation}
    X_{p,c} = \begin{cases}
        1 & \text{if crop } c \text{ is assigned to plot } p \\
        0 & \text{otherwise}
    \end{cases}
\end{equation}

\subsubsection{For Uneven Farm Distribution (Continuous)}
Continuous area variables:
\begin{equation}
    A_{f,c} \in [0, A_f] \quad \forall f \in F, c \in C
\end{equation}
where $A_{f,c}$ represents the area (in hectares) of crop $c$ grown on farm $f$, subject to the farm capacity constraint:
\begin{equation}
    \sum_{c \in C} A_{f,c} \leq A_f \quad \forall f \in F
\end{equation}
    \end{cases}
\end{equation}

\subsection{Objective Function}

\subsubsection{For Even Grid Discretization (Binary)}
The objective maximizes the total area-weighted value across all plot assignments:
\begin{equation}
    \max \quad \frac{1}{A_{\text{total}}} \sum_{p \in P} \sum_{c \in C} a_p \cdot B_c \cdot X_{p,c}
\end{equation}

\subsubsection{For Uneven Farm Distribution (Continuous)}
The objective maximizes the total value across all farm areas (identical to solver\_runner.py):
\begin{equation}
    \max \quad \frac{1}{A_{\text{total}}} \sum_{f \in F} \sum_{c \in C} B_c \cdot A_{f,c}
\end{equation}

where the benefit coefficient $B_c$ for each crop is computed as:
\begin{multline}
    B_c = w_{\text{nutritional}} \cdot \phi_{c,\text{nutritional}} + w_{\text{nutrient\_density}} \cdot \phi_{c,\text{nutrient\_density}} \\
    - w_{\text{environmental}} \cdot \phi_{c,\text{environmental}} + w_{\text{affordability}} \cdot \phi_{c,\text{affordability}} \\
    + w_{\text{sustainability}} \cdot \phi_{c,\text{sustainability}}
\end{multline>

Note: The environmental impact is subtracted (negative weight) as lower impact is preferred.

\subsection{Constraints}

\subsection{Constraints}

\subsubsection{For Even Grid Discretization (Binary)}

\textbf{Single Crop per Plot:}
Each plot can be assigned to at most one crop:
\begin{equation}
    \sum_{c \in C} X_{p,c} \leq 1 \quad \forall p \in P
\end{equation}

\textbf{Crop Area Constraints:}
Total area assigned to each crop:
\begin{align}
    \sum_{p \in P} X_{p,c} \times a_p &\geq s_c^{\min} \quad \forall c \in C \\
    \sum_{p \in P} X_{p,c} \times a_p &\leq s_c^{\max} \quad \forall c \in C
\end{align}

\subsubsection{For Uneven Farm Distribution (Continuous)}

\textbf{Farm Capacity:}
Total area used on each farm cannot exceed farm size:
\begin{equation}
    \sum_{c \in C} A_{f,c} \leq A_f \quad \forall f \in F
\end{equation}

\textbf{Crop Area Constraints:}
Total area assigned to each crop:
\begin{align}
    \sum_{f \in F} A_{f,c} &\geq s_c^{\min} \quad \forall c \in C \\
    \sum_{f \in F} A_{f,c} &\leq s_c^{\max} \quad \forall c \in C
\end{align>

\textbf{Non-negativity:}
\begin{equation}
    A_{f,c} \geq 0 \quad \forall f \in F, c \in C
\end{equation}

\subsubsection{Crop Area Constraints}

Each crop must be assigned a total area between its minimum and maximum bounds. For both representation methods, the total area assigned to crop $c$ is:

\begin{equation}
\subsubsection{Food Group Diversity Constraints (Both Formulations)}

To ensure dietary diversity, both formulations use indicator variables:
\begin{equation}
    Y_c = \begin{cases}
        1 & \text{if crop } c \text{ is planted} \\
        0 & \text{otherwise}
    \end{cases}
\end{equation>

\textbf{For Binary Formulation:}
\begin{equation}
    Y_c \leq \sum_{p \in P} X_{p,c} \quad \forall c \in C
\end{equation}

\textbf{For Continuous Formulation:}
\begin{equation>
    M \cdot Y_c \geq \sum_{f \in F} A_{f,c} \quad \forall c \in C
\end{equation}
where $M$ is a large constant.

\textbf{Food Group Constraints (Both):}
\begin{align}
    \sum_{c \in C_g} Y_c &\geq n_g^{\min} \quad \forall g \in G \\
    \sum_{c \in C_g} Y_c &\leq n_g^{\max} \quad \forall g \in G
\end{align>
        0 & \text{otherwise}
    \end{cases}
\end{equation}

These are linked to the land unit assignments via:

\begin{equation}
    Y_c \leq \sum_{u \in U} X_{u,c} \quad \forall c \in C
\end{equation}

Then, for each food group:

\begin{align}
    \sum_{c \in C_g} Y_c &\geq n_g^{\min} \quad \forall g \in G \quad \text{(if specified)} \\
    \sum_{c \in C_g} Y_c &\leq n_g^{\max} \quad \forall g \in G \quad \text{(if specified)}
\end{align}
\end{align}

These constraints ensure that a minimum/maximum number of different crop types from each food group are selected.

\section{Implementation Details}

\subsection{CQM Construction Algorithm}

The \texttt{create\_cqm()} function builds the Constrained Quadratic Model:

\begin{algorithm}[H]
\caption{CQM Construction for Binary Formulation}
\begin{algorithmic}[1]
\STATE Initialize empty CQM
\STATE Extract parameters: $P$, $C$, $G$, weights, constraints
\STATE Calculate total operations for progress tracking
\FOR{each plot $p \in P$}
    \FOR{each crop $c \in C$}
        \STATE Create binary variable $X_{p,c}$
    \ENDFOR
\ENDFOR
\FOR{each crop $c \in C$}
    \STATE Create binary indicator variable $Y_c$
\ENDFOR
\STATE Initialize objective $= 0$
\FOR{each plot $p \in P$}
    \FOR{each crop $c \in C$}
        \STATE Compute $B_c$ from weighted attributes
        \STATE Add $B_c \cdot X_{p,c}$ to objective
    \ENDFOR
\ENDFOR
\STATE Normalize objective by $|P| \times |C|$
\STATE Set CQM objective to $-\text{objective}$ (for minimization)
\FOR{each plot $p \in P$}
    \STATE Add constraint: $\sum_{c \in C} X_{p,c} \leq 1$ (single crop per plot)
\ENDFOR
\FOR{each crop $c \in C$}
    \STATE Add constraint: $\sum_{p \in P} X_{p,c} \geq \lceil s_c^{\min} \rceil$ (minimum area)
    \STATE Add constraint: $\sum_{p \in P} X_{p,c} \leq \lfloor s_c^{\max} \rfloor$ (maximum area)
    \STATE Add constraint: $Y_c \leq \sum_{p \in P} X_{p,c}$ (crop selection indicator)
\ENDFOR
\IF{food group constraints exist}
    \FOR{each group $g \in G$}
        \IF{minimum constraint exists}
            \STATE Add constraint: $\sum_{c \in C_g} Y_c \geq n_g^{\min}$
        \ENDIF
        \IF{maximum constraint exists}
            \STATE Add constraint: $\sum_{c \in C_g} Y_c \leq n_g^{\max}$
        \ENDIF
    \ENDFOR
\ENDIF
\RETURN CQM, variables, constraint metadata
\end{algorithmic}
\end{algorithm}

\subsection{PuLP Solver Implementation}

The \texttt{solve\_with\_pulp()} function creates a classical MILP formulation:

\begin{itemize}
    \item \textbf{Variable Creation}: Creates binary variables $X_{u,c}$ for each land unit-crop pair and indicator variables $Y_c$ for crop selection
    \item \textbf{Objective Construction}: Builds the area-weighted objective with normalization
    \item \textbf{Constraint Addition}: Adds all constraints using PuLP's API:
    \begin{itemize}
        \item Single crop per land unit constraints
        \item Minimum and maximum area constraints per crop
        \item Crop selection indicator linking
        \item Food group diversity constraints
    \end{itemize}
    \item \textbf{Solver Invocation}: Uses CBC (COIN-OR Branch and Cut) solver
    \item \textbf{Solution Extraction}: Extracts binary values for all land unit-crop assignments
\end{itemize}

The normalization factor ensures objective values are comparable across different problem sizes and land distributions.

\subsection{D-Wave Quantum Solver Implementation}

The \texttt{solve\_with\_dwave()} function enables quantum annealing:

\begin{algorithm}[H]
\caption{D-Wave Quantum Solving via BQM Conversion}
\begin{algorithmic}[1]
\STATE Start BQM conversion timer
\STATE Convert CQM to BQM using \texttt{cqm\_to\_bqm()}
\STATE Record BQM conversion time
\STATE Report BQM statistics (variables, interactions)
\STATE Initialize LeapHybridBQMSampler with API token
\STATE Submit BQM to D-Wave cloud service
\STATE Retrieve sampleset with solutions
\STATE Extract timing information:
\begin{itemize}
    \item Hybrid solve time (total wall-clock time)
    \item QPU access time (actual quantum processor time)
\end{itemize}
\STATE Convert timing from microseconds to seconds
\RETURN sampleset, hybrid\_time, qpu\_time, conversion\_time, invert
\end{algorithmic}
\end{algorithm}

\subsubsection{CQM to BQM Conversion}

The conversion process discretizes any remaining continuous aspects and transforms the constrained problem into an unconstrained binary problem by:

\begin{enumerate}
    \item Converting all constraints to penalty terms
    \item Adjusting the objective function to include penalties
    \item Creating an inversion function to map BQM solutions back to CQM variables
\end{enumerate}

\section{Key Design Features}

\subsection{Pure Binary Formulation}

Unlike hybrid formulations that mix continuous and binary variables, this solver uses exclusively binary variables:

\begin{itemize}
    \item \textbf{Advantage 1}: Direct compatibility with quantum annealers (no discretization needed)
    \item \textbf{Advantage 2}: Clearer interpretation (each variable = crop assignment to a specific land unit)
    \item \textbf{Advantage 3}: Simpler constraint structure (linear constraints only)
    \item \textbf{Advantage 4}: Flexible land representation (equal plots or variable farms)
    \item \textbf{Trade-off}: Less flexibility than continuous formulations; resolution limited by discretization level
\end{itemize>

\subsection{Normalization Strategy}

The objective is normalized by $|P| \times |C|$ to ensure:

\begin{equation}
    \text{Objective} = \frac{\sum_{p,c} B_c \cdot X_{p,c}}{|P| \times |C|}
\end{equation}

This normalization provides:
\begin{itemize}
    \item Scale-independent comparison across problem sizes
    \item Bounded objective values (approximately $[0, \max(B_c)]$)
    \item Compatibility with continuous formulation metrics
\end{itemize}

\subsection{Progress Tracking}

The implementation uses \texttt{tqdm} progress bars to track:

\begin{align}
    \text{Total Operations} = |P| \times |C| + |C| + |P| \times |C| + |P| + 3 \times |C| + 2 \times |G|
\end{align}

where the terms represent:
\begin{enumerate}
    \item Variable creation operations for $X_{p,c}$
    \item Variable creation operations for $Y_c$
    \item Objective term additions
    \item Single crop per plot constraints
    \item Crop area constraints (min, max, and indicator linking)
    \item Food group constraints (min and max)
\end{enumerate}

\section{Comparison with Continuous Formulation}

\begin{table}[h]
\centering
\begin{tabular}{|l|l|l|}
\hline
\textbf{Aspect} & \textbf{Even Grid (Binary)} & \textbf{Uneven Farms (Continuous)} \\
\hline
Variables & $X_{p,c} \in \{0,1\}$ & $A_{f,c} \in [0, A_f]$ \\
Land Representation & Equal-sized plots & Variable-sized farms \\
Optimization Type & Discrete binary (MILP) & Continuous linear (LP) \\
Quantum Compatibility & Native (BQUBO) & Not compatible \\
Solver & Binary CQM + PuLP + D-Wave & Continuous CQM + PuLP \\
Formulation & Plot assignment & Area allocation \\
Constraints & Single crop per plot & Farm capacity limits \\
\hline
\end{tabular}
\caption{Comparison between even grid (binary) and uneven farm (continuous) formulations}
\end{table}

\section{Usage and Integration}

\subsection{Main Function Flow}

\begin{enumerate}
    \item Load scenario data (farms, crops, food groups, configuration)
    \item Generate land data based on selected method (even grid or uneven farms)
    \item Create appropriate formulation:
    \begin{itemize}
        \item Even Grid: Binary CQM with plot assignment variables
        \item Uneven Farms: Continuous CQM with area allocation variables
    \end{itemize}
    \item Solve with PuLP (MILP for binary, LP for continuous)
    \item Solve with D-Wave (only for binary formulation)
    \item Save all results and create run manifest
\end{enumerate}

\subsection{Command Line Interface}

\begin{lstlisting}
# Binary formulation (even grid)
python solver_runner_BINARY.py --land-method even_grid --n-units 25 --total-land 100.0

# Continuous formulation (uneven farms) 
python solver_runner_BINARY.py --land-method uneven_distribution --n-units 15 --total-land 100.0
\end{lstlisting}

\subsection{Output Files}

The solver generates different files based on the formulation:

\textbf{For Even Grid (Binary):}
\begin{itemize}
    \item \texttt{cqm\_binary\_[scenario]\_[timestamp].cqm}: Binary CQM model
    \item \texttt{pulp\_binary\_[scenario]\_[timestamp].json}: Classical MILP solution
    \item \texttt{dwave\_bqubo\_[scenario]\_[timestamp].json}: Quantum solution
    \item \texttt{constraints\_[scenario]\_[timestamp].json}: Constraint metadata
\end{itemize}

\textbf{For Uneven Farms (Continuous):}
\begin{itemize}
    \item \texttt{cqm\_continuous\_[scenario]\_[timestamp].cqm}: Continuous CQM model
    \item \texttt{pulp\_continuous\_[scenario]\_[timestamp].json}: Classical LP solution
    \item \texttt{constraints\_[scenario]\_[timestamp].json}: Constraint metadata
\end{itemize}

\section{Computational Complexity}

\subsection{Problem Size}

\begin{align}
    \text{Variables} &= |P| \times |C| + |C| \quad \text{($X_{p,c}$ and $Y_c$ variables)} \\
    \text{Constraints} &= |P| + 3 \times |C| + 2 \times |G| \\
    \text{Objective Terms} &= |P| \times |C|
\end{align}

where:
\begin{itemize}
    \item $|P|$ constraints for single crop per plot
    \item $|C|$ constraints for minimum area per crop
    \item $|C|$ constraints for maximum area per crop
    \item $|C|$ constraints for crop selection indicators
    \item $2 \times |G|$ constraints for food group diversity (min and max)
\end{itemize}

\subsection{Scaling Characteristics}

The binary formulation scales as:
\begin{itemize}
    \item \textbf{Variables}: $O(|P| \times |C|)$ - linear in both dimensions
    \item \textbf{Constraints}: $O(|P| + |C| + |G|)$ - dominated by plot count for large problems
    \item \textbf{Objective Evaluation}: $O(|P| \times |C|)$ - single pass computation
\end{itemize}

For quantum solvers, the BQM conversion may add quadratic interaction terms, resulting in:
\begin{equation}
    \text{BQM Interactions} \leq \binom{|P| \times |C|}{2} = O((|P| \times |C|)^2)
\end{equation}

\subsection{Discretization Impact}

The discretization level directly affects problem size. For both representation methods:

\textbf{Even Grid Discretization:}
\begin{align}
    \text{Fine grid (many plots):} \quad &|U| \uparrow \quad \rightarrow \quad \text{More variables, better resolution} \\
    \text{Coarse grid (few plots):} \quad &|U| \downarrow \quad \rightarrow \quad \text{Fewer variables, lower resolution}
\end{align}

\textbf{Uneven Farm Distribution:}
\begin{align}
    \text{Many small farms:} \quad &|U| \uparrow \quad \rightarrow \quad \text{More variables, fine-grained decisions} \\
    \text{Few large farms:} \quad &|U| \downarrow \quad \rightarrow \quad \text{Fewer variables, coarse decisions}
\end{align}

\section{Conclusion}

This solver provides a comprehensive framework for comparing discrete binary optimization with continuous linear programming on agricultural land allocation problems. By supporting both formulations within a single implementation, it enables direct empirical comparison between quantum-compatible discrete approaches and classical continuous methods.

\textbf{Key advantages of the dual formulation approach:}
\begin{itemize}
    \item \textbf{Direct comparison}: Same agricultural problem solved with different optimization paradigms
    \item \textbf{Quantum readiness}: Binary formulation native to D-Wave quantum annealers (BQUBO)
    \item \textbf{Classical optimality}: Continuous formulation leverages mature linear programming solvers
    \item \textbf{Flexible land representation}: Even grids (theoretical) vs realistic farms (practical)
    \item \textbf{Benchmarking capability}: Quantifies performance trade-offs between discrete and continuous approaches
\end{itemize}

\textbf{Research applications:}
\begin{itemize}
    \item Evaluating quantum advantage for combinatorial optimization in agriculture
    \item Understanding discretization trade-offs in land allocation planning
    \item Comparing solution quality between MILP and quantum annealing approaches
    \item Assessing scalability of quantum methods vs classical methods
\end{itemize>

The solver architecture enables systematic study of how land representation choices and optimization paradigms affect agricultural decision-making, providing valuable insights for both quantum computing research and agricultural optimization practice.

\appendix

\section{Code Structure}

\begin{itemize}
    \item \textbf{create\_cqm()}: Builds the Constrained Quadratic Model
    \item \textbf{solve\_with\_pulp()}: Classical MILP solver using CBC
    \item \textbf{solve\_with\_dwave()}: Quantum solver using Leap HybridBQM
    \item \textbf{main()}: Orchestrates the complete solving pipeline
\end{itemize}

\section{Dependencies}

\begin{lstlisting}
dimod                  # D-Wave's modeling library
dwave.system           # D-Wave quantum samplers
pulp                   # Classical optimization
tqdm                   # Progress tracking
\end{lstlisting}

\end{document}
